\documentclass[11pt]{article}%
\usepackage[T1]{fontenc}%
\usepackage[utf8]{inputenc}%
\usepackage{lmodern}%
\usepackage{textcomp}%
\usepackage{lastpage}%
\usepackage{tikz}%
\usepackage{fontspec}%
\usepackage{polyglossia}%
\usepackage[paperwidth=1176pt,paperheight=1320pt,margin=0pt]{geometry}%
%
\setmainlanguage{urdu}%
\setotherlanguage{english}%
\newfontfamily\arabicfont[Script=Arabic,Path=/home/vivek/Pager/fonts/urdu/Paragraph/]{Mikhak-Thin}%
\newfontfamily\headerfont[Script=Arabic,Path=/home/vivek/Pager/fonts/urdu/Header/]{Handjet-Black}%
\newfontfamily\paragraphfont[Script=Arabic,Path=/home/vivek/Pager/fonts/urdu/Paragraph/]{Mikhak-Thin}%

    \makeatletter
    \newcommand{\zettaHuge}{\@setfontsize\zettaHuge{200}{220}}
    \newcommand{\exaHuge}{\@setfontsize\exaHuge{165}{180}}
    \newcommand{\petaHuge}{\@setfontsize\petaHuge{135}{150}}
    \newcommand{\teraHuge}{\@setfontsize\teraHuge{110}{120}}
    \newcommand{\gigaHuge}{\@setfontsize\gigaHuge{90}{100}}
    \newcommand{\megaHuge}{\@setfontsize\megaHuge{75}{85}}
    \newcommand{\superHuge}{\@setfontsize\superHuge{62}{70}}
    \newcommand{\verylarge}{\@setfontsize\verylarge{37}{42}}
    \newcommand{\veryLarge}{\@setfontsize\veryLarge{43}{49}}
    \newcommand{\veryHuge}{\@setfontsize\veryHuge{62}{70}}
    \newcommand{\alphaa}{\@setfontsize\alphaa{60}{66}}
    \newcommand{\betaa}{\@setfontsize\betaa{57}{63}}
    \newcommand{\gammaa}{\@setfontsize\gammaa{55}{61}}
    \newcommand{\deltaa}{\@setfontsize\deltaa{53}{59}}
    \newcommand{\epsilona}{\@setfontsize\epsilona{51}{57}}
    \newcommand{\zetaa}{\@setfontsize\zetaa{47}{53}}
    \newcommand{\etaa}{\@setfontsize\etaa{45}{51}}
    \newcommand{\iotaa}{\@setfontsize\iotaa{41}{47}}
    \newcommand{\kappaa}{\@setfontsize\kappaa{39}{45}}
    \newcommand{\lambdaa}{\@setfontsize\lambdaa{35}{41}}
    \newcommand{\mua}{\@setfontsize\mua{33}{39}}
    \newcommand{\nua}{\@setfontsize\nua{31}{37}}
    \newcommand{\xia}{\@setfontsize\xia{29}{35}}
    \newcommand{\pia}{\@setfontsize\pia{27}{33}}
    \newcommand{\rhoa}{\@setfontsize\rhoa{24}{30}}
    \newcommand{\sigmaa}{\@setfontsize\sigmaa{22}{28}}
    \newcommand{\taua}{\@setfontsize\taua{18}{24}}
    \newcommand{\upsilona}{\@setfontsize\upsilona{16}{22}}
    \newcommand{\phia}{\@setfontsize\phia{15}{20}}
    \newcommand{\chia}{\@setfontsize\chia{13}{18}}
    \newcommand{\psia}{\@setfontsize\psia{11}{16}}
    \newcommand{\omegaa}{\@setfontsize\omegaa{6}{7}}
    \newcommand{\oomegaa}{\@setfontsize\oomegaa{4}{5}}
    \newcommand{\ooomegaa}{\@setfontsize\ooomegaa{3}{4}}
    \newcommand{\oooomegaaa}{\@setfontsize\oooomegaaa{2}{3}}
    \makeatother
    %
%
\begin{document}%
\normalsize%
\begin{center}%
\begin{tikzpicture}[x=1pt, y=1pt]%
\node[anchor=south west, inner sep=0pt] at (0,0) {\includegraphics[width=1176pt,height=1320pt]{/home/vivek/Pager/images_val/mg_bn_000212_0.png}};%
\tikzset{headertext/.style={font=\headerfont, text=black}}%
\tikzset{paragraphtext/.style={font=\paragraphfont, text=black}}%
\node[paragraphtext, anchor=north west, text width=309.0pt, align=justify] at (67.0,615.5){\setlength{\baselineskip}{19.8pt} \par
\taua {\hspace{2em}کی حمایت کرنے والی اہم تحریریں کیں\linebreak}};%
\node[paragraphtext, anchor=north west, text width=309.0pt, align=justify] at (67.0,597.5){\setlength{\baselineskip}{19.8pt} \par
\taua {پاک و ہند میں انسانی زندگی کے ابتدائی\linebreak}};%
\node[paragraphtext, anchor=north west, text width=309.0pt, align=justify] at (67.0,579.5){\setlength{\baselineskip}{19.8pt} \par
\taua {علاقہ شیرانی میں آباد ہو گیا۔ اپنے کارکنوں\linebreak}};%
\node[paragraphtext, anchor=north west, text width=309.0pt, align=justify] at (67.0,561.5){\setlength{\baselineskip}{19.8pt} \par
\taua {قوانین بنائے اور عدالتی فیصلے صادر کیے۔\linebreak}};%
\node[paragraphtext, anchor=north west, text width=309.0pt, align=justify] at (67.0,543.5){\setlength{\baselineskip}{19.8pt} \par
\taua {۔ بابا عالم سیاہ کی نظم آٹا کے چند اشعار\linebreak}};%
\node[paragraphtext, anchor=north west, text width=309.0pt, align=justify] at (67.0,525.5){\setlength{\baselineskip}{19.8pt} \par
\taua {مہاراشٹر کے ضلع بیڈ کے پاٹودا تعلقہ میں\linebreak}};%
\node[paragraphtext, anchor=north west, text width=309.0pt, align=justify] at (67.0,507.5){\setlength{\baselineskip}{19.8pt} \par
\taua {برطانیہ ، آسٹریا ، ہنگری ، روس سمیت\linebreak}};%
\node[paragraphtext, anchor=north west, text width=309.0pt, align=justify] at (67.0,489.5){\setlength{\baselineskip}{19.8pt} \par
\taua {ریتواڑی نظام کے لاگو ہونے کا باعث بنے۔\linebreak}};%
\node[paragraphtext, anchor=north west, text width=312.0pt, align=justify] at (67.0,1048.5){\setlength{\baselineskip}{19.8pt} \par
\taua {\hspace{2em}کھرے قرض کی مثالوں میں آپ کا آپ\linebreak}};%
\node[paragraphtext, anchor=north west, text width=312.0pt, align=justify] at (67.0,1030.5){\setlength{\baselineskip}{19.8pt} \par
\taua {لوگ ایمان لے آئے ۔ آج بھی تلاوت قرآن مع\linebreak}};%
\node[paragraphtext, anchor=north west, text width=312.0pt, align=justify] at (67.0,1012.5){\setlength{\baselineskip}{19.8pt} \par
\taua {کے شمال وسطی علاقے میں واقع ایک چهوٹا\linebreak}};%
\node[paragraphtext, anchor=north west, text width=312.0pt, align=justify] at (67.0,994.5){\setlength{\baselineskip}{19.8pt} \par
\taua {مختلف قلعے،تاریخی\linebreak}};%
\node[paragraphtext, anchor=north west, text width=312.0pt, align=justify] at (67.0,976.5){\setlength{\baselineskip}{19.8pt} \par
\taua {سے یہ تفسیر بھری پڑی ہے۔ شانِ نزول بھی\linebreak}};%
\node[paragraphtext, anchor=north west, text width=312.0pt, align=justify] at (67.0,958.5){\setlength{\baselineskip}{19.8pt} \par
\taua {بعد وہ نواب بشیر دولہ کے معتمد ،مشیر کے\linebreak}};%
\node[paragraphtext, anchor=north west, text width=312.0pt, align=justify] at (67.0,940.5){\setlength{\baselineskip}{19.8pt} \par
\taua {آئی چوبیس انیس سو بہتر سے کام کرتی ہے\linebreak}};%
\node[paragraphtext, anchor=north west, text width=312.0pt, align=justify] at (67.0,922.5){\setlength{\baselineskip}{19.8pt} \par
\taua {لیے سبسڈی اور ٹیرف پر یورپی یونین اور\linebreak}};%
\node[paragraphtext, anchor=north west, text width=312.0pt, align=justify] at (67.0,904.5){\setlength{\baselineskip}{19.8pt} \par
\taua {بہ طور وزیر اعلیٰ پنجاب منتخب ہونے میں\linebreak}};%
\node[paragraphtext, anchor=north west, text width=312.0pt, align=justify] at (67.0,886.5){\setlength{\baselineskip}{19.8pt} \par
\taua {آندھرا پردیش، مدھیہ پردیش سے کشمیر،\linebreak}};%
\node[paragraphtext, anchor=north west, text width=312.0pt, align=justify] at (67.0,868.5){\setlength{\baselineskip}{19.8pt} \par
\taua {سو چھبیس ہجری نو سو سترہجری فقہ حنفی\linebreak}};%
\node[paragraphtext, anchor=north west, text width=312.0pt, align=justify] at (67.0,850.5){\setlength{\baselineskip}{19.8pt} \par
\taua {اکثر مخالف کے بادشاہ کے خلاف حملے سے\linebreak}};%
\node[paragraphtext, anchor=north west, text width=312.0pt, align=justify] at (67.0,832.5){\setlength{\baselineskip}{19.8pt} \par
\taua {والی کئی سڑکیں شامل ہیں۔ پنڈ دادن خان\linebreak}};%
\node[paragraphtext, anchor=north west, text width=312.0pt, align=justify] at (67.0,814.5){\setlength{\baselineskip}{19.8pt} \par
\taua {گھر پر ہی والد گرامی اور برادر اکبر مولانا\linebreak}};%
\node[paragraphtext, anchor=north west, text width=312.0pt, align=justify] at (67.0,796.5){\setlength{\baselineskip}{19.8pt} \par
\taua {کے سنگم پر واقع ہیں ۔ پامیر تاجکستان،\linebreak}};%
\node[paragraphtext, anchor=north west, text width=312.0pt, align=justify] at (67.0,778.5){\setlength{\baselineskip}{19.8pt} \par
\taua {پکڑ لیا جب ان کا ہیلی کاپٹر معمول کی تربیتی\linebreak}};%
\node[paragraphtext, anchor=north west, text width=312.0pt, align=justify] at (67.0,760.5){\setlength{\baselineskip}{19.8pt} \par
\taua {جناب مولانا فیصل احمد ندوی بھٹکلی کی\linebreak}};%
\node[paragraphtext, anchor=north west, text width=312.0pt, align=justify] at (67.0,742.5){\setlength{\baselineskip}{19.8pt} \par
\taua {چرکسی سرزمین کے مغربی سرے پر تمنیو\linebreak}};%
\node[paragraphtext, anchor=north west, text width=312.0pt, align=justify] at (67.0,724.5){\setlength{\baselineskip}{19.8pt} \par
\taua {بارود کا کارخانہ قائم کر دیا ایک روز کسی\linebreak}};%
\node[paragraphtext, anchor=north west, text width=312.0pt, align=justify] at (67.0,706.5){\setlength{\baselineskip}{19.8pt} \par
\taua {کے نتیجے میں مصر میں نپولین بوناپارٹ کی\linebreak}};%
\node[paragraphtext, anchor=north west, text width=312.0pt, align=justify] at (67.0,688.5){\setlength{\baselineskip}{19.8pt} \par
\taua {ٹرین کو روک کر سب سے پہلے ہندو\linebreak}};%
\node[paragraphtext, anchor=north west, text width=312.0pt, align=justify] at (67.0,670.5){\setlength{\baselineskip}{19.8pt} \par
\taua {اڈتیس آٹھایس یعنی اٹھسٹھ میں، تقریباً دس\linebreak}};%
\node[paragraphtext, anchor=north west, text width=312.0pt, align=justify] at (67.0,652.5){\setlength{\baselineskip}{19.8pt} \par
\taua {سبائی ایک اہم جین مرکز تھا اور جین مندر\linebreak}};%
\node[headertext, anchor=north west, text width=159.0pt, align=left] at (60.0,1269.5){\setlength{\baselineskip}{19.8pt} \par
\taua {لوٹ مار ان کا خاصہ\linebreak}};%
\node[headertext, anchor=north west, text width=159.0pt, align=left] at (60.0,1251.5){\setlength{\baselineskip}{19.8pt} \par
\taua {ہے کہ امام ابویوسف\linebreak}};%
\node[headertext, anchor=north west, text width=159.0pt, align=left] at (60.0,1233.5){\setlength{\baselineskip}{19.8pt} \par
\taua {تیونسی مورخ تھے ۔\linebreak}};%
\node[headertext, anchor=north west, text width=159.0pt, align=left] at (60.0,1215.5){\setlength{\baselineskip}{19.8pt} \par
\taua {تیلی داڑہ دہلی سے\linebreak}};%
\node[headertext, anchor=north west, text width=159.0pt, align=left] at (60.0,1197.5){\setlength{\baselineskip}{19.8pt} \par
\taua {کے تحت ایک محفوظ\linebreak}};%
\node[headertext, anchor=north west, text width=159.0pt, align=left] at (60.0,1179.5){\setlength{\baselineskip}{19.8pt} \par
\taua {تاریخی اہمیت کا پتہ\linebreak}};%
\node[paragraphtext, anchor=north west, text width=314.0pt, align=justify] at (403.0,594.5){\setlength{\baselineskip}{19.8pt} \par
\taua {\hspace{2em}کی ٹوکریوں میں پھینکا جاتا تھا۔ نے\linebreak}};%
\node[paragraphtext, anchor=north west, text width=314.0pt, align=justify] at (403.0,576.5){\setlength{\baselineskip}{19.8pt} \par
\taua {کے جھنڈے پر سبز رنگ کو ہی جگہ دی ۔\linebreak}};%
\node[paragraphtext, anchor=north west, text width=314.0pt, align=justify] at (403.0,558.5){\setlength{\baselineskip}{19.8pt} \par
\taua {پر حاوی ہونا نہایت ہی دشوار مسئلہ ہے ۔ لہذا\linebreak}};%
\node[paragraphtext, anchor=north west, text width=314.0pt, align=justify] at (403.0,540.5){\setlength{\baselineskip}{19.8pt} \par
\taua {پہلے ہندو مسافروں کی شناخت کی گئی ۔\linebreak}};%
\node[paragraphtext, anchor=north west, text width=314.0pt, align=justify] at (403.0,522.5){\setlength{\baselineskip}{19.8pt} \par
\taua {ازاد كشمير پاكستان كا ایک خوبصورت گاوں\linebreak}};%
\node[paragraphtext, anchor=north west, text width=314.0pt, align=justify] at (403.0,504.5){\setlength{\baselineskip}{19.8pt} \par
\taua {انہوں نے وصیت کی کہ اگر وہ جنگ ًمیں\linebreak}};%
\node[paragraphtext, anchor=north west, text width=314.0pt, align=justify] at (403.0,486.5){\setlength{\baselineskip}{19.8pt} \par
\taua {فلموں کے شریک مصنف ہیں فلم میں اداکارہ\linebreak}};%
\node[paragraphtext, anchor=north west, text width=314.0pt, align=justify] at (403.0,468.5){\setlength{\baselineskip}{19.8pt} \par
\taua {یہ غلط فہمی اور سوء ظنی قائم تھی کہ یہ\linebreak}};%
\node[paragraphtext, anchor=north west, text width=314.0pt, align=justify] at (403.0,450.5){\setlength{\baselineskip}{19.8pt} \par
\taua {کہ اسے گرفتار کر لیا گیا ۔ وہ تین چار برس\linebreak}};%
\node[headertext, anchor=north west, text width=586.0pt, align=left, text=black] at (107.0,1123.5){\setlength{\baselineskip}{40.7pt} \par
\verylarge {عصری تعلیم کا سلسلہ منقطع کردیا ۔ والد صاحب\linebreak}};%
\node[paragraphtext, anchor=north west, text width=339.0pt, align=justify] at (390.0,1048.5){\setlength{\baselineskip}{19.8pt} \par
\taua {\hspace{2em}ذہبی کی تحقیق کے مطابق ہارون کی\linebreak}};%
\node[paragraphtext, anchor=north west, text width=339.0pt, align=justify] at (390.0,1030.5){\setlength{\baselineskip}{19.8pt} \par
\taua {اور جس کا عقد ایک ناگا شہزادی کے ساتھ ہوا تھا۔\linebreak}};%
\node[paragraphtext, anchor=north west, text width=339.0pt, align=justify] at (390.0,1012.5){\setlength{\baselineskip}{19.8pt} \par
\taua {کہ ملگھاٹ ٹائیگر ریزرو جنگلات کے آس پاس ،\linebreak}};%
\node[paragraphtext, anchor=north west, text width=339.0pt, align=justify] at (390.0,994.5){\setlength{\baselineskip}{19.8pt} \par
\taua {پانچ نفوس پر مشتمل ہے۔ اور ہوئیتینن کا کل\linebreak}};%
\node[paragraphtext, anchor=north west, text width=339.0pt, align=justify] at (390.0,976.5){\setlength{\baselineskip}{19.8pt} \par
\taua {ڈول ھے۔ یہ نام قدیم رکارڈ میں آتا ھے چھ سو\linebreak}};%
\node[paragraphtext, anchor=north west, text width=339.0pt, align=justify] at (390.0,958.5){\setlength{\baselineskip}{19.8pt} \par
\taua {دور سے ہی نظر آجاتا ہے ۔ سلاگر کے معنی ہیں\linebreak}};%
\node[paragraphtext, anchor=north west, text width=339.0pt, align=justify] at (390.0,940.5){\setlength{\baselineskip}{19.8pt} \par
\taua {خٹک پشتو خټک پختون قبیلہ خٹک قبائل\linebreak}};%
\node[paragraphtext, anchor=north west, text width=326.0pt, align=justify] at (394.0,905.5){\setlength{\baselineskip}{19.8pt} \par
\taua {\hspace{2em}لکھنئو میں تھا پاکستان نے یہ میچ اننگز\linebreak}};%
\node[paragraphtext, anchor=north west, text width=326.0pt, align=justify] at (394.0,887.5){\setlength{\baselineskip}{19.8pt} \par
\taua {ہجا ہے سِنۨہ اور انگریزی میں اس کی حرف\linebreak}};%
\node[paragraphtext, anchor=north west, text width=326.0pt, align=justify] at (394.0,869.5){\setlength{\baselineskip}{19.8pt} \par
\taua {نبرد آزما ہو کر انہیں قابو میں کیا۔ ستارہ سو\linebreak}};%
\node[paragraphtext, anchor=north west, text width=332.0pt, align=justify] at (392.0,711.5){\setlength{\baselineskip}{19.8pt} \par
\taua {\hspace{2em}مانا جاتا ہے کہ آپ پاتال بھونیشور میں کر\linebreak}};%
\node[paragraphtext, anchor=north west, text width=332.0pt, align=justify] at (392.0,693.5){\setlength{\baselineskip}{19.8pt} \par
\taua {کرتا تھا ۔ اور لگتا ہے کہ یہ دیوار امویوں نے بنائی\linebreak}};%
\node[paragraphtext, anchor=north west, text width=332.0pt, align=justify] at (392.0,675.5){\setlength{\baselineskip}{19.8pt} \par
\taua {کو بھی حاصل ہوچکا تھا ۔ لیکن اہل علم وخیر\linebreak}};%
\node[paragraphtext, anchor=north west, text width=330.0pt, align=justify] at (401.0,647.5){\setlength{\baselineskip}{19.8pt} \par
\taua {\hspace{2em}اسے بادشاہ تسلیم کرنے سے انکار کر دیا ان\linebreak}};%
\node[paragraphtext, anchor=north west, text width=330.0pt, align=justify] at (401.0,629.5){\setlength{\baselineskip}{19.8pt} \par
\taua {انسان تھے۔ فقری بازی اور لطیفہ گوئی میں ان\linebreak}};%
\node[paragraphtext, anchor=north west, text width=321.0pt, align=justify] at (395.0,422.5){\setlength{\baselineskip}{19.8pt} \par
\taua {\hspace{2em}پیپلزپارٹی سے وابستگی رکھتے ہیں ۔ اور\linebreak}};%
\node[paragraphtext, anchor=north west, text width=321.0pt, align=justify] at (395.0,404.5){\setlength{\baselineskip}{19.8pt} \par
\taua {کے عجائب گھر کی انتظامیہ کے حوالے کر دیا\linebreak}};%
\node[paragraphtext, anchor=north west, text width=321.0pt, align=justify] at (395.0,386.5){\setlength{\baselineskip}{19.8pt} \par
\taua {نام نہ ہونے کی وجہ سے جدید محققین نے اس\linebreak}};%
\node[paragraphtext, anchor=north west, text width=321.0pt, align=justify] at (395.0,368.5){\setlength{\baselineskip}{19.8pt} \par
\taua {کی وجہ سے اکثر اوقات موت واقع ہوجاتی ہے ۔\linebreak}};%
\node[paragraphtext, anchor=north west, text width=321.0pt, align=justify] at (395.0,350.5){\setlength{\baselineskip}{19.8pt} \par
\taua {ہوتا ہے ۔ ناجی خان ناجی نہ صرف انجمن ترقی\linebreak}};%
\node[paragraphtext, anchor=north west, text width=321.0pt, align=justify] at (395.0,332.5){\setlength{\baselineskip}{19.8pt} \par
\taua {قبیلہ خٹک کا سردار اور مغلوں کی جانب سے\linebreak}};%
\node[paragraphtext, anchor=north west, text width=321.0pt, align=justify] at (395.0,314.5){\setlength{\baselineskip}{19.8pt} \par
\taua {تھا، اور تاریخی شواہد یہ ثابت کرتے ہیں کہ\linebreak}};%
\node[paragraphtext, anchor=north west, text width=324.0pt, align=justify] at (57.0,457.5){\setlength{\baselineskip}{19.8pt} \par
\taua {\hspace{2em}خلاف حفاظت کے طور پر بنائی گئی ایسا\linebreak}};%
\node[paragraphtext, anchor=north west, text width=324.0pt, align=justify] at (57.0,439.5){\setlength{\baselineskip}{19.8pt} \par
\taua {ہوا کی طرف اشارہ کرتی ہے۔ اِس سوال کا جواب\linebreak}};%
\node[paragraphtext, anchor=north west, text width=324.0pt, align=justify] at (57.0,421.5){\setlength{\baselineskip}{19.8pt} \par
\taua {شروع ہوتے ہں اور بعد میں رات کو ختم ہوتیں\linebreak}};%
\node[paragraphtext, anchor=north west, text width=324.0pt, align=justify] at (57.0,403.5){\setlength{\baselineskip}{19.8pt} \par
\taua {ہوجاتی ہیں۔ یہی وہ چیز ہے جسے مغربی\linebreak}};%
\node[paragraphtext, anchor=north west, text width=324.0pt, align=justify] at (57.0,385.5){\setlength{\baselineskip}{19.8pt} \par
\taua {پیدائش اٹھارہ سو پچہتر وفات ساتھ جولائی\linebreak}};%
\node[paragraphtext, anchor=north west, text width=324.0pt, align=justify] at (57.0,367.5){\setlength{\baselineskip}{19.8pt} \par
\taua {بڑی تیزی سے پهیل رہا تها ۔ بازطینی اور جنبش\linebreak}};%
\node[paragraphtext, anchor=north west, text width=324.0pt, align=justify] at (57.0,349.5){\setlength{\baselineskip}{19.8pt} \par
\taua {لب و حلق سے نکلتی ہیں دوسرے حروف بھی\linebreak}};%
\node[paragraphtext, anchor=north west, text width=324.0pt, align=justify] at (57.0,331.5){\setlength{\baselineskip}{19.8pt} \par
\taua {کو متحد کریں، افغانستان، شام، اردن، کویت،\linebreak}};%
\node[paragraphtext, anchor=north west, text width=324.0pt, align=justify] at (57.0,313.5){\setlength{\baselineskip}{19.8pt} \par
\taua {لیکن قوم کے لیے اپنے والد کی قربانی کی\linebreak}};%
\node[paragraphtext, anchor=north west, text width=324.0pt, align=justify] at (57.0,295.5){\setlength{\baselineskip}{19.8pt} \par
\taua {آدمی کی اصطلاح انیس سو انتالیس میں\linebreak}};%
\node[paragraphtext, anchor=north west, text width=324.0pt, align=justify] at (57.0,277.5){\setlength{\baselineskip}{19.8pt} \par
\taua {کا خیال رکھا ہے۔ امام نضرؒ کی مذکورہ بالا\linebreak}};%
\node[paragraphtext, anchor=north west, text width=324.0pt, align=justify] at (57.0,259.5){\setlength{\baselineskip}{19.8pt} \par
\taua {تھے۔ ہاشم علی خان عباسی پیدإش اٹھارہ\linebreak}};%
\node[paragraphtext, anchor=north west, text width=324.0pt, align=justify] at (57.0,241.5){\setlength{\baselineskip}{19.8pt} \par
\taua {لیکن ان کا مسلک شیعہ ہونے کی وجہ سے\linebreak}};%
\node[paragraphtext, anchor=north west, text width=324.0pt, align=justify] at (57.0,223.5){\setlength{\baselineskip}{19.8pt} \par
\taua {کاریگروں کے گروہوں کی بڑھتی ہوئی طاقت کی\linebreak}};%
\node[paragraphtext, anchor=north west, text width=324.0pt, align=justify] at (57.0,205.5){\setlength{\baselineskip}{19.8pt} \par
\taua {تازہ ہوا کا ایک دلنشین جھونکا ہے ۔ آپؒ کے\linebreak}};%
\node[paragraphtext, anchor=north west, text width=324.0pt, align=justify] at (57.0,187.5){\setlength{\baselineskip}{19.8pt} \par
\taua {میانمار، نیپال، لائوس، تائیوان، ویٹنام، برونائی،\linebreak}};%
\node[paragraphtext, anchor=north west, text width=324.0pt, align=justify] at (57.0,169.5){\setlength{\baselineskip}{19.8pt} \par
\taua {کی عمر تقریباً اسی سال تھی جب آپ کا وصال\linebreak}};%
\node[paragraphtext, anchor=north west, text width=324.0pt, align=justify] at (57.0,151.5){\setlength{\baselineskip}{19.8pt} \par
\taua {پھر ایک قرارداد پاس ہوٸی جس کے تحت آل\linebreak}};%
\node[paragraphtext, anchor=north west, text width=324.0pt, align=justify] at (57.0,133.5){\setlength{\baselineskip}{19.8pt} \par
\taua {تک ان سالوں میں اموی خلافت کے ذریعہ\linebreak}};%
\node[paragraphtext, anchor=north west, text width=324.0pt, align=justify] at (57.0,115.5){\setlength{\baselineskip}{19.8pt} \par
\taua {ہوئے ہیں اور انسانی زندگی کا سب سے پہلا\linebreak}};%
\node[paragraphtext, anchor=north west, text width=165.0pt, align=left] at (759.0,331.5){\setlength{\baselineskip}{19.8pt} \par
\taua {تفصیل دی ہے ۔ اس کی\linebreak}};%
\node[paragraphtext, anchor=north west, text width=165.0pt, align=left] at (759.0,313.5){\setlength{\baselineskip}{19.8pt} \par
\taua {گولو رومنی زبان سے\linebreak}};%
\node[paragraphtext, anchor=north west, text width=165.0pt, align=left] at (759.0,295.5){\setlength{\baselineskip}{19.8pt} \par
\taua {کی توثیق بہت سے علما\linebreak}};%
\node[paragraphtext, anchor=north west, text width=223.0pt, align=left] at (872.0,198.5){\setlength{\baselineskip}{19.8pt} \par
\taua {ان چار شعبوں کے ذیل میں\linebreak}};%
\node[paragraphtext, anchor=north west, text width=223.0pt, align=left] at (872.0,180.5){\setlength{\baselineskip}{19.8pt} \par
\taua {مسموع آواز سے پہلے کی\linebreak}};%
\node[paragraphtext, anchor=north west, text width=223.0pt, align=left] at (872.0,162.5){\setlength{\baselineskip}{19.8pt} \par
\taua {دیا گیا۔ پانی میں گندھک ہوتی\linebreak}};%
\node[paragraphtext, anchor=north west, text width=223.0pt, align=left] at (872.0,144.5){\setlength{\baselineskip}{19.8pt} \par
\taua {پھر عمرہ یا حج کی نیت کریں\linebreak}};%
\node[paragraphtext, anchor=north west, text width=223.0pt, align=left] at (872.0,126.5){\setlength{\baselineskip}{19.8pt} \par
\taua {یگانۂروزگار،صاحبِ ورع واتقاء\linebreak}};%
\end{tikzpicture}%
\end{center}%
\end{document}