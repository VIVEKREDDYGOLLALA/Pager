\documentclass[11pt]{article}%
\usepackage[T1]{fontenc}%
\usepackage[utf8]{inputenc}%
\usepackage{lmodern}%
\usepackage{textcomp}%
\usepackage{lastpage}%
\usepackage{tikz}%
\usepackage{fontspec}%
\usepackage{polyglossia}%
\usepackage[paperwidth=1070pt,paperheight=1637pt,margin=0pt]{geometry}%
%
\setmainlanguage{urdu}%
\setotherlanguage{english}%
\newfontfamily\arabicfont[Script=Arabic,Path=/home/vivek/Pager/fonts/urdu/Paragraph/]{Lemonada-SemiBold}%
\newfontfamily\headerfont[Script=Arabic,Path=/home/vivek/Pager/fonts/urdu/Header/]{Estedad-ExtraBold}%
\newfontfamily\paragraphfont[Script=Arabic,Path=/home/vivek/Pager/fonts/urdu/Paragraph/]{Lemonada-SemiBold}%

    \makeatletter
    \newcommand{\zettaHuge}{\@setfontsize\zettaHuge{200}{220}}
    \newcommand{\exaHuge}{\@setfontsize\exaHuge{165}{180}}
    \newcommand{\petaHuge}{\@setfontsize\petaHuge{135}{150}}
    \newcommand{\teraHuge}{\@setfontsize\teraHuge{110}{120}}
    \newcommand{\gigaHuge}{\@setfontsize\gigaHuge{90}{100}}
    \newcommand{\megaHuge}{\@setfontsize\megaHuge{75}{85}}
    \newcommand{\superHuge}{\@setfontsize\superHuge{62}{70}}
    \newcommand{\verylarge}{\@setfontsize\verylarge{37}{42}}
    \newcommand{\veryLarge}{\@setfontsize\veryLarge{43}{49}}
    \newcommand{\veryHuge}{\@setfontsize\veryHuge{62}{70}}
    \newcommand{\alphaa}{\@setfontsize\alphaa{60}{66}}
    \newcommand{\betaa}{\@setfontsize\betaa{57}{63}}
    \newcommand{\gammaa}{\@setfontsize\gammaa{55}{61}}
    \newcommand{\deltaa}{\@setfontsize\deltaa{53}{59}}
    \newcommand{\epsilona}{\@setfontsize\epsilona{51}{57}}
    \newcommand{\zetaa}{\@setfontsize\zetaa{47}{53}}
    \newcommand{\etaa}{\@setfontsize\etaa{45}{51}}
    \newcommand{\iotaa}{\@setfontsize\iotaa{41}{47}}
    \newcommand{\kappaa}{\@setfontsize\kappaa{39}{45}}
    \newcommand{\lambdaa}{\@setfontsize\lambdaa{35}{41}}
    \newcommand{\mua}{\@setfontsize\mua{33}{39}}
    \newcommand{\nua}{\@setfontsize\nua{31}{37}}
    \newcommand{\xia}{\@setfontsize\xia{29}{35}}
    \newcommand{\pia}{\@setfontsize\pia{27}{33}}
    \newcommand{\rhoa}{\@setfontsize\rhoa{24}{30}}
    \newcommand{\sigmaa}{\@setfontsize\sigmaa{22}{28}}
    \newcommand{\taua}{\@setfontsize\taua{18}{24}}
    \newcommand{\upsilona}{\@setfontsize\upsilona{16}{22}}
    \newcommand{\phia}{\@setfontsize\phia{15}{20}}
    \newcommand{\chia}{\@setfontsize\chia{13}{18}}
    \newcommand{\psia}{\@setfontsize\psia{11}{16}}
    \newcommand{\omegaa}{\@setfontsize\omegaa{6}{7}}
    \newcommand{\oomegaa}{\@setfontsize\oomegaa{4}{5}}
    \newcommand{\ooomegaa}{\@setfontsize\ooomegaa{3}{4}}
    \newcommand{\oooomegaaa}{\@setfontsize\oooomegaaa{2}{3}}
    \makeatother
    %
%
\begin{document}%
\normalsize%
\begin{center}%
\begin{tikzpicture}[x=1pt, y=1pt]%
\node[anchor=south west, inner sep=0pt] at (0,0) {\includegraphics[width=1070pt,height=1637pt]{/home/vivek/Pager/images_val/mg_hi_000307_0.png}};%
\tikzset{headertext/.style={font=\headerfont, text=black}}%
\tikzset{paragraphtext/.style={font=\paragraphfont, text=black}}%
\node[paragraphtext, anchor=north west, text width=866.0pt, align=justify] at (84.0,919.5){\setlength{\baselineskip}{47.300000000000004pt} \par
\veryLarge {منمودی غار ہندوستان کے شہر جنر\linebreak}};%
\node[paragraphtext, anchor=north west, text width=866.0pt, align=justify] at (84.0,876.0){\setlength{\baselineskip}{47.300000000000004pt} \par
\veryLarge {وغیرہ پر بھی اس فن کو استعمال\linebreak}};%
\node[paragraphtext, anchor=north west, text width=866.0pt, align=justify] at (84.0,832.5){\setlength{\baselineskip}{47.300000000000004pt} \par
\veryLarge {بولی ہے۔ ناپید ہونے کے دہانے پر ہے\linebreak}};%
\node[paragraphtext, anchor=north west, text width=866.0pt, align=justify] at (84.0,789.0){\setlength{\baselineskip}{47.300000000000004pt} \par
\veryLarge {مديرية القبيطة یمن کا ایک فہرست\linebreak}};%
\node[paragraphtext, anchor=north west, text width=866.0pt, align=justify] at (84.0,745.5){\setlength{\baselineskip}{47.300000000000004pt} \par
\veryLarge {خاں رازی ؔسترہویں صدی کے اوائل\linebreak}};%
\node[paragraphtext, anchor=north west, text width=866.0pt, align=justify] at (84.0,702.0){\setlength{\baselineskip}{47.300000000000004pt} \par
\veryLarge {ان کا قبضہ آج کے شمالی تامل ناڈو\linebreak}};%
\node[paragraphtext, anchor=north west, text width=866.0pt, align=justify] at (84.0,658.5){\setlength{\baselineskip}{47.300000000000004pt} \par
\veryLarge {نوگانواں تک رہی ہیں ۔ چھولس\linebreak}};%
\node[paragraphtext, anchor=north west, text width=866.0pt, align=justify] at (84.0,615.0){\setlength{\baselineskip}{47.300000000000004pt} \par
\veryLarge {انکار کر دیا ہم مسلمان ہیں اور\linebreak}};%
\node[paragraphtext, anchor=north west, text width=866.0pt, align=justify] at (84.0,571.5){\setlength{\baselineskip}{47.300000000000004pt} \par
\veryLarge {ديتا ہے ۔ جوہر نے اپنے امام کے حکم\linebreak}};%
\node[paragraphtext, anchor=north west, text width=866.0pt, align=justify] at (84.0,528.0){\setlength{\baselineskip}{47.300000000000004pt} \par
\veryLarge {کے مسموم ماحول میں بھی حریت\linebreak}};%
\node[paragraphtext, anchor=north west, text width=866.0pt, align=justify] at (84.0,484.5){\setlength{\baselineskip}{47.300000000000004pt} \par
\veryLarge {ہیں ۔ اس نے اپنی مملکت کو بیس تا\linebreak}};%
\node[paragraphtext, anchor=north west, text width=866.0pt, align=justify] at (84.0,441.0){\setlength{\baselineskip}{47.300000000000004pt} \par
\veryLarge {اپنی تحریروں میں اردو کا مزاق و\linebreak}};%
\node[paragraphtext, anchor=north west, text width=866.0pt, align=justify] at (84.0,397.5){\setlength{\baselineskip}{47.300000000000004pt} \par
\veryLarge {قسم لکڑی کی چھڑیوں کے ساتھ\linebreak}};%
\node[paragraphtext, anchor=north west, text width=866.0pt, align=justify] at (84.0,354.0){\setlength{\baselineskip}{47.300000000000004pt} \par
\veryLarge {سو پچہتر ہجری میں لاڑ\linebreak}};%
\node[paragraphtext, anchor=north west, text width=866.0pt, align=justify] at (84.0,310.5){\setlength{\baselineskip}{47.300000000000004pt} \par
\veryLarge {ناکام واپس لوٹے لاہور پر قبضہ کرنے\linebreak}};%
\node[paragraphtext, anchor=north west, text width=866.0pt, align=justify] at (84.0,267.0){\setlength{\baselineskip}{47.300000000000004pt} \par
\veryLarge {اسے سرعام زندہ جلا دیا۔ بيلجئيم كي\linebreak}};%
\node[paragraphtext, anchor=north west, text width=866.0pt, align=justify] at (84.0,223.5){\setlength{\baselineskip}{47.300000000000004pt} \par
\veryLarge {ریونیو آفیسر فرائض انجام دیتے رہے ۔\linebreak}};%
\node[paragraphtext, anchor=north west, text width=866.0pt, align=justify] at (84.0,180.0){\setlength{\baselineskip}{47.300000000000004pt} \par
\veryLarge {وفات کے بعد سلطنت میں سیاسی\linebreak}};%
\node[paragraphtext, anchor=north west, text width=865.0pt, align=justify] at (81.0,1523.5){\setlength{\baselineskip}{47.300000000000004pt} \par
\veryLarge {\hspace{2em}لکھا ہے مِنْ بَیْتٍ شریف اسی\linebreak}};%
\node[paragraphtext, anchor=north west, text width=865.0pt, align=justify] at (81.0,1480.0){\setlength{\baselineskip}{47.300000000000004pt} \par
\veryLarge {آباد ہونے کا مشاہدہ کیا ہے، جو\linebreak}};%
\node[paragraphtext, anchor=north west, text width=874.0pt, align=justify] at (76.0,1381.5){\setlength{\baselineskip}{47.300000000000004pt} \par
\veryLarge {کی آواز میں انیس سو ستہر میں\linebreak}};%
\node[paragraphtext, anchor=north west, text width=874.0pt, align=justify] at (76.0,1338.0){\setlength{\baselineskip}{47.300000000000004pt} \par
\veryLarge {کو تباہ و برباد کرتے تھے اور ہر بار\linebreak}};%
\node[paragraphtext, anchor=north west, text width=874.0pt, align=justify] at (76.0,1294.5){\setlength{\baselineskip}{47.300000000000004pt} \par
\veryLarge {کا سامنا کر رہا ہے، خاص طور پر\linebreak}};%
\node[paragraphtext, anchor=north west, text width=874.0pt, align=justify] at (76.0,1251.0){\setlength{\baselineskip}{47.300000000000004pt} \par
\veryLarge {آپ کے والدین نے بھی دیگر لوگوں\linebreak}};%
\node[paragraphtext, anchor=north west, text width=874.0pt, align=justify] at (76.0,1207.5){\setlength{\baselineskip}{47.300000000000004pt} \par
\veryLarge {تھے کہ جمعیت علمائے ہند نے انگریز\linebreak}};%
\node[paragraphtext, anchor=north west, text width=874.0pt, align=justify] at (76.0,1164.0){\setlength{\baselineskip}{47.300000000000004pt} \par
\veryLarge {دو سو سترہ فوجی مارے گئے۔\linebreak}};%
\node[paragraphtext, anchor=north west, text width=874.0pt, align=justify] at (76.0,1120.5){\setlength{\baselineskip}{47.300000000000004pt} \par
\veryLarge {انگریز لڑکیوں کو دہلی میں بھارتی\linebreak}};%
\node[paragraphtext, anchor=north west, text width=874.0pt, align=justify] at (76.0,1077.0){\setlength{\baselineskip}{47.300000000000004pt} \par
\veryLarge {کا ذریعہ تھا ۔ شہنشاہ اکبر نے اسی\linebreak}};%
\node[paragraphtext, anchor=north west, text width=874.0pt, align=justify] at (76.0,1033.5){\setlength{\baselineskip}{47.300000000000004pt} \par
\veryLarge {ہے وہ دس بھائی تھے جن میں پانچ\linebreak}};%
\end{tikzpicture}%
\end{center}%
\end{document}