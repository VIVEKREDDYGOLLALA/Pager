\documentclass[11pt]{article}%
\usepackage[T1]{fontenc}%
\usepackage[utf8]{inputenc}%
\usepackage{lmodern}%
\usepackage{textcomp}%
\usepackage{lastpage}%
\usepackage{tikz}%
\usepackage{fontspec}%
\usepackage{polyglossia}%
\usepackage[paperwidth=1070pt,paperheight=1637pt,margin=0pt]{geometry}%
%
\setmainlanguage{urdu}%
\setotherlanguage{english}%
\newfontfamily\arabicfont[Script=Arabic,Path=/home/vivek/Pager/fonts/urdu/Paragraph/]{Handjet-ExtraLight}%
\newfontfamily\headerfont[Script=Arabic,Path=/home/vivek/Pager/fonts/urdu/Header/]{Mikhak-DS2-Black}%
\newfontfamily\paragraphfont[Script=Arabic,Path=/home/vivek/Pager/fonts/urdu/Paragraph/]{Handjet-ExtraLight}%

    \makeatletter
    \newcommand{\zettaHuge}{\@setfontsize\zettaHuge{200}{220}}
    \newcommand{\exaHuge}{\@setfontsize\exaHuge{165}{180}}
    \newcommand{\petaHuge}{\@setfontsize\petaHuge{135}{150}}
    \newcommand{\teraHuge}{\@setfontsize\teraHuge{110}{120}}
    \newcommand{\gigaHuge}{\@setfontsize\gigaHuge{90}{100}}
    \newcommand{\megaHuge}{\@setfontsize\megaHuge{75}{85}}
    \newcommand{\superHuge}{\@setfontsize\superHuge{62}{70}}
    \newcommand{\verylarge}{\@setfontsize\verylarge{37}{42}}
    \newcommand{\veryLarge}{\@setfontsize\veryLarge{43}{49}}
    \newcommand{\veryHuge}{\@setfontsize\veryHuge{62}{70}}
    \newcommand{\alphaa}{\@setfontsize\alphaa{60}{66}}
    \newcommand{\betaa}{\@setfontsize\betaa{57}{63}}
    \newcommand{\gammaa}{\@setfontsize\gammaa{55}{61}}
    \newcommand{\deltaa}{\@setfontsize\deltaa{53}{59}}
    \newcommand{\epsilona}{\@setfontsize\epsilona{51}{57}}
    \newcommand{\zetaa}{\@setfontsize\zetaa{47}{53}}
    \newcommand{\etaa}{\@setfontsize\etaa{45}{51}}
    \newcommand{\iotaa}{\@setfontsize\iotaa{41}{47}}
    \newcommand{\kappaa}{\@setfontsize\kappaa{39}{45}}
    \newcommand{\lambdaa}{\@setfontsize\lambdaa{35}{41}}
    \newcommand{\mua}{\@setfontsize\mua{33}{39}}
    \newcommand{\nua}{\@setfontsize\nua{31}{37}}
    \newcommand{\xia}{\@setfontsize\xia{29}{35}}
    \newcommand{\pia}{\@setfontsize\pia{27}{33}}
    \newcommand{\rhoa}{\@setfontsize\rhoa{24}{30}}
    \newcommand{\sigmaa}{\@setfontsize\sigmaa{22}{28}}
    \newcommand{\taua}{\@setfontsize\taua{18}{24}}
    \newcommand{\upsilona}{\@setfontsize\upsilona{16}{22}}
    \newcommand{\phia}{\@setfontsize\phia{15}{20}}
    \newcommand{\chia}{\@setfontsize\chia{13}{18}}
    \newcommand{\psia}{\@setfontsize\psia{11}{16}}
    \newcommand{\omegaa}{\@setfontsize\omegaa{6}{7}}
    \newcommand{\oomegaa}{\@setfontsize\oomegaa{4}{5}}
    \newcommand{\ooomegaa}{\@setfontsize\ooomegaa{3}{4}}
    \newcommand{\oooomegaaa}{\@setfontsize\oooomegaaa{2}{3}}
    \makeatother
    %
%
\begin{document}%
\normalsize%
\begin{center}%
\begin{tikzpicture}[x=1pt, y=1pt]%
\node[anchor=south west, inner sep=0pt] at (0,0) {\includegraphics[width=1070pt,height=1637pt]{/home/vivek/Pager/images_val/mg_hi_000426_0.png}};%
\tikzset{headertext/.style={font=\headerfont, text=black}}%
\tikzset{paragraphtext/.style={font=\paragraphfont, text=black}}%
\node[paragraphtext, anchor=north west, text width=854.0pt, align=justify] at (60.0,1533.5){\setlength{\baselineskip}{47.300000000000004pt} \par
\veryLarge {\hspace{2em}مگر سنگھالی زبان کا ترجمہ رہے گیا ۔ جسے بارہ سو ؁میں\linebreak}};%
\node[paragraphtext, anchor=north west, text width=854.0pt, align=justify] at (60.0,1490.0){\setlength{\baselineskip}{47.300000000000004pt} \par
\veryLarge {قائد اعظم، لیکچرز آن اقبال اور ڈاکٹر این میری شمل،ادب اور\linebreak}};%
\node[paragraphtext, anchor=north west, text width=854.0pt, align=justify] at (60.0,1446.5){\setlength{\baselineskip}{47.300000000000004pt} \par
\veryLarge {نے اصل دین کی طرف واپسی کی تحریک چلائی اور فرماتے تھے\linebreak}};%
\node[paragraphtext, anchor=north west, text width=854.0pt, align=justify] at (60.0,1403.0){\setlength{\baselineskip}{47.300000000000004pt} \par
\veryLarge {جاتے ہیں جو مقامی روایات اور عقائد كی عكاسی كرتے ہیں\linebreak}};%
\node[paragraphtext, anchor=north west, text width=854.0pt, align=justify] at (60.0,1359.5){\setlength{\baselineskip}{47.300000000000004pt} \par
\veryLarge {تمام تراجم کیلیے بمنزلہ اساس ہے ۔ اس ترجمہ پر شاہ صاحب کے\linebreak}};%
\node[paragraphtext, anchor=north west, text width=854.0pt, align=justify] at (60.0,1316.0){\setlength{\baselineskip}{47.300000000000004pt} \par
\veryLarge {سے جڑا ہوا تھا۔ مدورائی ہوائی اڈہ ریاست کا واحد کسٹم\linebreak}};%
\node[paragraphtext, anchor=north west, text width=854.0pt, align=justify] at (60.0,1272.5){\setlength{\baselineskip}{47.300000000000004pt} \par
\veryLarge {گیت نگاروں نے اپنے فن کا جادو جگایا ان میں حزیں قادری،وارث\linebreak}};%
\node[paragraphtext, anchor=north west, text width=854.0pt, align=justify] at (60.0,1229.0){\setlength{\baselineskip}{47.300000000000004pt} \par
\veryLarge {اس وقت شہرت ملی جب شاہ مراکش نے ایک فاسی کمپیوٹر\linebreak}};%
\node[paragraphtext, anchor=north west, text width=854.0pt, align=justify] at (60.0,1185.5){\setlength{\baselineskip}{47.300000000000004pt} \par
\veryLarge {میوزیم کرد مۆزەخانەی کەلەپو کورد، عربی متحف التراث الكردي\linebreak}};%
\node[paragraphtext, anchor=north west, text width=848.0pt, align=justify] at (59.0,1096.5){\setlength{\baselineskip}{47.300000000000004pt} \par
\veryLarge {\hspace{2em}پر دی ویڈنگ کریشرز کی طرح ٹریٹمنٹ کے ساتھ چک\linebreak}};%
\node[paragraphtext, anchor=north west, text width=848.0pt, align=justify] at (59.0,1053.0){\setlength{\baselineskip}{47.300000000000004pt} \par
\veryLarge {چین کے شہر نے نۓینگ،صوبہ ہنان سے تھا اور وہ مشرقی ہان\linebreak}};%
\node[paragraphtext, anchor=north west, text width=848.0pt, align=justify] at (59.0,1009.5){\setlength{\baselineskip}{47.300000000000004pt} \par
\veryLarge {افسانے حال کے سیاسی پس منظر میں لکھے گئے ہیں ۔ ان کی\linebreak}};%
\node[paragraphtext, anchor=north west, text width=848.0pt, align=justify] at (59.0,966.0){\setlength{\baselineskip}{47.300000000000004pt} \par
\veryLarge {سے دو ہزار دو کابل اور باقی کے کمانڈر دو ہزار تین سے دو ہزار\linebreak}};%
\node[paragraphtext, anchor=north west, text width=848.0pt, align=justify] at (59.0,922.5){\setlength{\baselineskip}{47.300000000000004pt} \par
\veryLarge {نے سنہ دو ہزار پندرہ میں ایکسپریس اخبار کو انٹرویو میں بتایا\linebreak}};%
\node[paragraphtext, anchor=north west, text width=848.0pt, align=justify] at (59.0,879.0){\setlength{\baselineskip}{47.300000000000004pt} \par
\veryLarge {ایک فورس قاجر محل گارڈ جو کے نظام کہلاتی تھی سویڈش\linebreak}};%
\node[paragraphtext, anchor=north west, text width=848.0pt, align=justify] at (59.0,835.5){\setlength{\baselineskip}{47.300000000000004pt} \par
\veryLarge {میں واحد شہر پنڈدادن خان ہی تھا ۔ پنڈدادن خان بہت آباد\linebreak}};%
\node[paragraphtext, anchor=north west, text width=848.0pt, align=justify] at (59.0,792.0){\setlength{\baselineskip}{47.300000000000004pt} \par
\veryLarge {ہے، پوگو ٹی وی پر نشر ہوئی۔ اس مندر کا نام اسے بنانے والی کے\linebreak}};%
\node[paragraphtext, anchor=north west, text width=848.0pt, align=justify] at (59.0,748.5){\setlength{\baselineskip}{47.300000000000004pt} \par
\veryLarge {ایک بڑی افغان مصنف اور محقق تھیں جنہوں نے وزیر تعلیم کی\linebreak}};%
\node[paragraphtext, anchor=north west, text width=848.0pt, align=justify] at (59.0,705.0){\setlength{\baselineskip}{47.300000000000004pt} \par
\veryLarge {کو زیریں نظر عروج کے مقام پر دیکھتی ہیں۔ اس وقت دور\linebreak}};%
\node[paragraphtext, anchor=north west, text width=848.0pt, align=justify] at (59.0,661.5){\setlength{\baselineskip}{47.300000000000004pt} \par
\veryLarge {کرنے کی وجہ سے اس کا نام مشہور ہوا بیان کیاجاتاہے اسے\linebreak}};%
\node[paragraphtext, anchor=north west, text width=848.0pt, align=justify] at (59.0,618.0){\setlength{\baselineskip}{47.300000000000004pt} \par
\veryLarge {قائم کیں اور قیام کیا اس کے علاوہ شیرانوالہ دروازے کے علاقہ\linebreak}};%
\node[paragraphtext, anchor=north west, text width=848.0pt, align=justify] at (59.0,574.5){\setlength{\baselineskip}{47.300000000000004pt} \par
\veryLarge {اور شہر کی حیثیت سے نئی طرز پر بسایا اور دارالحکومت قرار\linebreak}};%
\node[paragraphtext, anchor=north west, text width=848.0pt, align=justify] at (59.0,531.0){\setlength{\baselineskip}{47.300000000000004pt} \par
\veryLarge {مدد ملتی ہے ۔ اذنا عربی إذنا فلسطین کا ایک رہائشی علاقہ\linebreak}};%
\node[paragraphtext, anchor=north west, text width=848.0pt, align=justify] at (59.0,487.5){\setlength{\baselineskip}{47.300000000000004pt} \par
\veryLarge {بڑی تبدیلی لائی دنیا کی دو کامیاب ترین ممالک میں سے دو ،\linebreak}};%
\node[paragraphtext, anchor=north west, text width=848.0pt, align=justify] at (59.0,444.0){\setlength{\baselineskip}{47.300000000000004pt} \par
\veryLarge {میں سے زیادہ تر پڑوسی خطوں میں گردش کرنے والے اٹک سکے\linebreak}};%
\node[paragraphtext, anchor=north west, text width=848.0pt, align=justify] at (59.0,400.5){\setlength{\baselineskip}{47.300000000000004pt} \par
\veryLarge {زر بن حبیش ہیں اورابو عمرو الشيبانی سے بھی علم حاصل کیا.\linebreak}};%
\node[paragraphtext, anchor=north west, text width=848.0pt, align=justify] at (59.0,357.0){\setlength{\baselineskip}{47.300000000000004pt} \par
\veryLarge {کہا جاتا ہے، دھوکہ دہی کے طور پر بے نقاب ہونے سے پہلے ہی\linebreak}};%
\node[paragraphtext, anchor=north west, text width=848.0pt, align=justify] at (59.0,313.5){\setlength{\baselineskip}{47.300000000000004pt} \par
\veryLarge {مالی امداد کی تھی۔ ویشنو دیوی فرار ہونے سے پہلے\linebreak}};%
\node[paragraphtext, anchor=north west, text width=848.0pt, align=justify] at (59.0,270.0){\setlength{\baselineskip}{47.300000000000004pt} \par
\veryLarge {میں سلطان مصطفی دوم کی سربراہی میں عثمانی فوج\linebreak}};%
\node[paragraphtext, anchor=north west, text width=848.0pt, align=justify] at (59.0,226.5){\setlength{\baselineskip}{47.300000000000004pt} \par
\veryLarge {ہر شعبے اور ہر عمل میں آزادی کی حامی تھی وہ چبیس مئی\linebreak}};%
\node[paragraphtext, anchor=north west, text width=848.0pt, align=justify] at (59.0,183.0){\setlength{\baselineskip}{47.300000000000004pt} \par
\veryLarge {جانے کی بھی اجازت نہیں تھی بادام پہلی چند سندھی\linebreak}};%
\end{tikzpicture}%
\end{center}%
\end{document}