\documentclass[11pt]{article}%
\usepackage[T1]{fontenc}%
\usepackage[utf8]{inputenc}%
\usepackage{lmodern}%
\usepackage{textcomp}%
\usepackage{lastpage}%
\usepackage{tikz}%
\usepackage{fontspec}%
\usepackage{polyglossia}%
\usepackage[paperwidth=1512pt,paperheight=2072pt,margin=0pt]{geometry}%
%
\setmainlanguage{urdu}%
\setotherlanguage{english}%
\newfontfamily\arabicfont[Script=Arabic,Path=/home/vivek/Pager/fonts/urdu/Paragraph/]{Mikhak-Light}%
\newfontfamily\headerfont[Script=Arabic,Path=/home/vivek/Pager/fonts/urdu/Header/]{DGAgnadeen-Extrabold}%
\newfontfamily\paragraphfont[Script=Arabic,Path=/home/vivek/Pager/fonts/urdu/Paragraph/]{Mikhak-Light}%

    \makeatletter
    \newcommand{\zettaHuge}{\@setfontsize\zettaHuge{200}{220}}
    \newcommand{\exaHuge}{\@setfontsize\exaHuge{165}{180}}
    \newcommand{\petaHuge}{\@setfontsize\petaHuge{135}{150}}
    \newcommand{\teraHuge}{\@setfontsize\teraHuge{110}{120}}
    \newcommand{\gigaHuge}{\@setfontsize\gigaHuge{90}{100}}
    \newcommand{\megaHuge}{\@setfontsize\megaHuge{75}{85}}
    \newcommand{\superHuge}{\@setfontsize\superHuge{62}{70}}
    \newcommand{\verylarge}{\@setfontsize\verylarge{37}{42}}
    \newcommand{\veryLarge}{\@setfontsize\veryLarge{43}{49}}
    \newcommand{\veryHuge}{\@setfontsize\veryHuge{62}{70}}
    \newcommand{\alphaa}{\@setfontsize\alphaa{60}{66}}
    \newcommand{\betaa}{\@setfontsize\betaa{57}{63}}
    \newcommand{\gammaa}{\@setfontsize\gammaa{55}{61}}
    \newcommand{\deltaa}{\@setfontsize\deltaa{53}{59}}
    \newcommand{\epsilona}{\@setfontsize\epsilona{51}{57}}
    \newcommand{\zetaa}{\@setfontsize\zetaa{47}{53}}
    \newcommand{\etaa}{\@setfontsize\etaa{45}{51}}
    \newcommand{\iotaa}{\@setfontsize\iotaa{41}{47}}
    \newcommand{\kappaa}{\@setfontsize\kappaa{39}{45}}
    \newcommand{\lambdaa}{\@setfontsize\lambdaa{35}{41}}
    \newcommand{\mua}{\@setfontsize\mua{33}{39}}
    \newcommand{\nua}{\@setfontsize\nua{31}{37}}
    \newcommand{\xia}{\@setfontsize\xia{29}{35}}
    \newcommand{\pia}{\@setfontsize\pia{27}{33}}
    \newcommand{\rhoa}{\@setfontsize\rhoa{24}{30}}
    \newcommand{\sigmaa}{\@setfontsize\sigmaa{22}{28}}
    \newcommand{\taua}{\@setfontsize\taua{18}{24}}
    \newcommand{\upsilona}{\@setfontsize\upsilona{16}{22}}
    \newcommand{\phia}{\@setfontsize\phia{15}{20}}
    \newcommand{\chia}{\@setfontsize\chia{13}{18}}
    \newcommand{\psia}{\@setfontsize\psia{11}{16}}
    \newcommand{\omegaa}{\@setfontsize\omegaa{6}{7}}
    \newcommand{\oomegaa}{\@setfontsize\oomegaa{4}{5}}
    \newcommand{\ooomegaa}{\@setfontsize\ooomegaa{3}{4}}
    \newcommand{\oooomegaaa}{\@setfontsize\oooomegaaa{2}{3}}
    \makeatother
    %
%
\begin{document}%
\normalsize%
\begin{center}%
\begin{tikzpicture}[x=1pt, y=1pt]%
\node[anchor=south west, inner sep=0pt] at (0,0) {\includegraphics[width=1512pt,height=2072pt]{/home/vivek/Pager/images_val/mg_kn_000435_0.png}};%
\tikzset{headertext/.style={font=\headerfont, text=black}}%
\tikzset{paragraphtext/.style={font=\paragraphfont, text=black}}%
\node[paragraphtext, anchor=north west, text width=667.0pt, align=justify] at (54.0,1462.5){\setlength{\baselineskip}{47.300000000000004pt} \par
\veryLarge {\hspace{2em}کہلاتے تھے ۔ جنھوں نے لوگوں\linebreak}};%
\node[paragraphtext, anchor=north west, text width=667.0pt, align=justify] at (54.0,1419.0){\setlength{\baselineskip}{47.300000000000004pt} \par
\veryLarge {آصفیہ رکھ دیا گیا بعد سے حمکرانوں نے\linebreak}};%
\node[paragraphtext, anchor=north west, text width=667.0pt, align=justify] at (54.0,1375.5){\setlength{\baselineskip}{47.300000000000004pt} \par
\veryLarge {حکومت خطرے میں تھی ۔ عوام\linebreak}};%
\node[paragraphtext, anchor=north west, text width=667.0pt, align=justify] at (54.0,1332.0){\setlength{\baselineskip}{47.300000000000004pt} \par
\veryLarge {کے درمیان راستے کو بند کرنے کے\linebreak}};%
\node[paragraphtext, anchor=north west, text width=667.0pt, align=justify] at (54.0,1288.5){\setlength{\baselineskip}{47.300000000000004pt} \par
\veryLarge {حصے میں ترک سلطان حسین خان نے\linebreak}};%
\node[paragraphtext, anchor=north west, text width=667.0pt, align=justify] at (54.0,1245.0){\setlength{\baselineskip}{47.300000000000004pt} \par
\veryLarge {پر ممتاز فیشن ڈیزائنرز اور کپڑوں کے\linebreak}};%
\node[paragraphtext, anchor=north west, text width=667.0pt, align=justify] at (54.0,1201.5){\setlength{\baselineskip}{47.300000000000004pt} \par
\veryLarge {آبشاروں میں ایلیفنٹ فالس، شدتھم\linebreak}};%
\node[paragraphtext, anchor=north west, text width=668.0pt, align=justify] at (50.0,1877.5){\setlength{\baselineskip}{47.300000000000004pt} \par
\veryLarge {\hspace{2em}نے آذربائیجان فتح کیا اور رَے کو\linebreak}};%
\node[paragraphtext, anchor=north west, text width=668.0pt, align=justify] at (50.0,1834.0){\setlength{\baselineskip}{47.300000000000004pt} \par
\veryLarge {تھے ۔ آپ کی والدہ نے بھوپال سے\linebreak}};%
\node[paragraphtext, anchor=north west, text width=668.0pt, align=justify] at (50.0,1790.5){\setlength{\baselineskip}{47.300000000000004pt} \par
\veryLarge {تبصرہ کیا گیا ہے۔ چولوں کا اڈہ کاویری\linebreak}};%
\node[paragraphtext, anchor=north west, text width=668.0pt, align=justify] at (50.0,1747.0){\setlength{\baselineskip}{47.300000000000004pt} \par
\veryLarge {اورعظیم المرتبت وجامع شخصیت\linebreak}};%
\node[paragraphtext, anchor=north west, text width=668.0pt, align=justify] at (50.0,1703.5){\setlength{\baselineskip}{47.300000000000004pt} \par
\veryLarge {صہبائی کے ادبی پرچے انصاری میں\linebreak}};%
\node[paragraphtext, anchor=north west, text width=668.0pt, align=justify] at (50.0,1660.0){\setlength{\baselineskip}{47.300000000000004pt} \par
\veryLarge {کی چیز ہے ۔ سندھ میں ایک جگہ\linebreak}};%
\node[paragraphtext, anchor=north west, text width=668.0pt, align=justify] at (50.0,1616.5){\setlength{\baselineskip}{47.300000000000004pt} \par
\veryLarge {گھر سے گرفتار کیا تھا۔ جس میں انھوں\linebreak}};%
\node[paragraphtext, anchor=north west, text width=668.0pt, align=justify] at (50.0,1573.0){\setlength{\baselineskip}{47.300000000000004pt} \par
\veryLarge {سوری نے اس وقت رکھی جب انہوں نے\linebreak}};%
\node[paragraphtext, anchor=north west, text width=668.0pt, align=justify] at (50.0,1529.5){\setlength{\baselineskip}{47.300000000000004pt} \par
\veryLarge {عآپ کو لوک سبھا انتخابات کے دوران\linebreak}};%
\node[paragraphtext, anchor=north west, text width=660.0pt, align=justify] at (59.0,1970.5){\setlength{\baselineskip}{47.300000000000004pt} \par
\veryLarge {\hspace{2em}پٹنہ پائریٹس کے محمدرضا نے\linebreak}};%
\node[paragraphtext, anchor=north west, text width=660.0pt, align=justify] at (59.0,1927.0){\setlength{\baselineskip}{47.300000000000004pt} \par
\veryLarge {حریت ضمیر سے جینے کی راہ دکھائی\linebreak}};%
\node[paragraphtext, anchor=north west, text width=673.0pt, align=justify] at (733.0,1970.5){\setlength{\baselineskip}{47.300000000000004pt} \par
\veryLarge {\hspace{2em}کا رکن تسلیم کرنا مشکل ہے اور فرض\linebreak}};%
\node[paragraphtext, anchor=north west, text width=673.0pt, align=justify] at (733.0,1927.0){\setlength{\baselineskip}{47.300000000000004pt} \par
\veryLarge {سور صرف سُروں پر مشتمل ہوتا ہے، اور\linebreak}};%
\node[paragraphtext, anchor=north west, text width=673.0pt, align=justify] at (733.0,1883.5){\setlength{\baselineskip}{47.300000000000004pt} \par
\veryLarge {مضامین نہیں ملیں گے جنہیں\linebreak}};%
\node[paragraphtext, anchor=north west, text width=673.0pt, align=justify] at (733.0,1840.0){\setlength{\baselineskip}{47.300000000000004pt} \par
\veryLarge {وی کے آئی اے این ایس نے فلم کو اس\linebreak}};%
\node[headertext, anchor=north west, text width=1252.0pt, align=left] at (69.0,1099.5){\setlength{\baselineskip}{66.0pt} \par
\alphaa {ایک زمانے میں فلمی صنعت میں ان کے نام کا ڈنکا\linebreak}};%
\node[paragraphtext, anchor=north west, text width=912.0pt, align=left] at (347.0,299.5){\setlength{\baselineskip}{47.300000000000004pt} \par
\veryLarge {لوگ اس لفظ کااستعمال اپنےناموں کےساتھ اپنے\linebreak}};%
\node[paragraphtext, anchor=north west, text width=912.0pt, align=left] at (347.0,256.0){\setlength{\baselineskip}{47.300000000000004pt} \par
\veryLarge {اور ساخت میں ہیگزاگنل ہے، ایک ساخت جسے آیی\linebreak}};%
\node[paragraphtext, anchor=north west, text width=912.0pt, align=left] at (347.0,212.5){\setlength{\baselineskip}{47.300000000000004pt} \par
\veryLarge {شستہ مزاح کے لیے معروف ہیں تاحال اسٹیج ڈراموں\linebreak}};%
\node[paragraphtext, anchor=north west, text width=475.0pt, align=justify] at (833.0,870.5){\setlength{\baselineskip}{47.300000000000004pt} \par
\veryLarge {\hspace{2em}کہا گیا ہے کہ کے\linebreak}};%
\node[paragraphtext, anchor=north west, text width=475.0pt, align=justify] at (833.0,827.0){\setlength{\baselineskip}{47.300000000000004pt} \par
\veryLarge {اولاد نرینہ نہ تھی تو ان کے\linebreak}};%
\node[paragraphtext, anchor=north west, text width=475.0pt, align=justify] at (833.0,783.5){\setlength{\baselineskip}{47.300000000000004pt} \par
\veryLarge {دینے کی دعوت دی گئی ۔\linebreak}};%
\node[paragraphtext, anchor=north west, text width=475.0pt, align=justify] at (833.0,740.0){\setlength{\baselineskip}{47.300000000000004pt} \par
\veryLarge {الخيل تھا اس کے قوانین\linebreak}};%
\node[paragraphtext, anchor=north west, text width=475.0pt, align=justify] at (833.0,696.5){\setlength{\baselineskip}{47.300000000000004pt} \par
\veryLarge {کے بعد فرانسیسی افواج کی\linebreak}};%
\node[paragraphtext, anchor=north west, text width=475.0pt, align=justify] at (833.0,653.0){\setlength{\baselineskip}{47.300000000000004pt} \par
\veryLarge {پر کھدی ہوئی سدھ کے\linebreak}};%
\node[paragraphtext, anchor=north west, text width=475.0pt, align=justify] at (833.0,609.5){\setlength{\baselineskip}{47.300000000000004pt} \par
\veryLarge {اپنی سلطنت میں شامل کر\linebreak}};%
\node[paragraphtext, anchor=north west, text width=475.0pt, align=justify] at (833.0,566.0){\setlength{\baselineskip}{47.300000000000004pt} \par
\veryLarge {تھی۔ ایک اقلیتی نظریہ یہ ہے\linebreak}};%
\node[paragraphtext, anchor=north west, text width=475.0pt, align=justify] at (833.0,522.5){\setlength{\baselineskip}{47.300000000000004pt} \par
\veryLarge {منترام ویسوا جو پانچ سال\linebreak}};%
\node[paragraphtext, anchor=north west, text width=464.0pt, align=justify] at (85.0,894.5){\setlength{\baselineskip}{47.300000000000004pt} \par
\veryLarge {\hspace{2em}درد کا گیت لائلپوری\linebreak}};%
\node[paragraphtext, anchor=north west, text width=464.0pt, align=justify] at (85.0,851.0){\setlength{\baselineskip}{47.300000000000004pt} \par
\veryLarge {پانچ کے کشمیر میں آنے\linebreak}};%
\node[paragraphtext, anchor=north west, text width=464.0pt, align=justify] at (85.0,807.5){\setlength{\baselineskip}{47.300000000000004pt} \par
\veryLarge {داس کی سنسکرت تصنیف\linebreak}};%
\node[paragraphtext, anchor=north west, text width=464.0pt, align=justify] at (85.0,764.0){\setlength{\baselineskip}{47.300000000000004pt} \par
\veryLarge {اور مکتب محمدو ؐآل\linebreak}};%
\node[paragraphtext, anchor=north west, text width=464.0pt, align=justify] at (85.0,720.5){\setlength{\baselineskip}{47.300000000000004pt} \par
\veryLarge {جو ان کے قبر پر لکھی گئی\linebreak}};%
\node[paragraphtext, anchor=north west, text width=464.0pt, align=justify] at (85.0,677.0){\setlength{\baselineskip}{47.300000000000004pt} \par
\veryLarge {جمعدار ملازم اکبری کے نام\linebreak}};%
\node[paragraphtext, anchor=north west, text width=464.0pt, align=justify] at (85.0,633.5){\setlength{\baselineskip}{47.300000000000004pt} \par
\veryLarge {اخباروں میں ڈیسک\linebreak}};%
\node[paragraphtext, anchor=north west, text width=464.0pt, align=justify] at (85.0,590.0){\setlength{\baselineskip}{47.300000000000004pt} \par
\veryLarge {پٹنہ سے سورت ،سلطان پور\linebreak}};%
\node[paragraphtext, anchor=north west, text width=464.0pt, align=justify] at (85.0,546.5){\setlength{\baselineskip}{47.300000000000004pt} \par
\veryLarge {کہا یہ ایک ایسی یونیورسٹی\linebreak}};%
\node[paragraphtext, anchor=north west, text width=464.0pt, align=justify] at (85.0,503.0){\setlength{\baselineskip}{47.300000000000004pt} \par
\veryLarge {بعد چودہ سال کے اکبر کو\linebreak}};%
\node[paragraphtext, anchor=north west, text width=464.0pt, align=justify] at (85.0,459.5){\setlength{\baselineskip}{47.300000000000004pt} \par
\veryLarge {نام سے جانا جاتا ہے یہ\linebreak}};%
\node[paragraphtext, anchor=north west, text width=483.0pt, align=left] at (726.0,1265.5){\setlength{\baselineskip}{47.300000000000004pt} \par
\veryLarge {سنایا وہ سننے سے پہلے\linebreak}};%
\node[paragraphtext, anchor=north west, text width=483.0pt, align=left] at (726.0,1222.0){\setlength{\baselineskip}{47.300000000000004pt} \par
\veryLarge {تلفّظ کی پیروی ضروری ہے\linebreak}};%
\node[paragraphtext, anchor=north west, text width=483.0pt, align=left] at (726.0,1178.5){\setlength{\baselineskip}{47.300000000000004pt} \par
\veryLarge {وہ کاپتی ها کا کارکن تھا وہ\linebreak}};%
\node[paragraphtext, anchor=north west, text width=664.0pt, align=justify] at (735.0,1766.5){\setlength{\baselineskip}{47.300000000000004pt} \par
\veryLarge {\hspace{2em}لینے والے مقامی سے\linebreak}};%
\node[paragraphtext, anchor=north west, text width=664.0pt, align=justify] at (735.0,1723.0){\setlength{\baselineskip}{47.300000000000004pt} \par
\veryLarge {ساتھ مفسدین کی سرکوبی کے لیے بھی\linebreak}};%
\node[paragraphtext, anchor=north west, text width=664.0pt, align=justify] at (735.0,1679.5){\setlength{\baselineskip}{47.300000000000004pt} \par
\veryLarge {یافتہ اور متحرک شہر تھا مگر برطانوی\linebreak}};%
\node[paragraphtext, anchor=north west, text width=664.0pt, align=justify] at (735.0,1636.0){\setlength{\baselineskip}{47.300000000000004pt} \par
\veryLarge {بریڈ، چٹنی اور ٹاپنگ کا استعمال\linebreak}};%
\node[paragraphtext, anchor=north west, text width=664.0pt, align=justify] at (735.0,1592.5){\setlength{\baselineskip}{47.300000000000004pt} \par
\veryLarge {کی گڑیاں بھی ہوسکتی ہیں لیکن ایک\linebreak}};%
\node[paragraphtext, anchor=north west, text width=664.0pt, align=justify] at (735.0,1549.0){\setlength{\baselineskip}{47.300000000000004pt} \par
\veryLarge {متاثر ہوئے اور انہیں ریڈیو پاکستان\linebreak}};%
\node[paragraphtext, anchor=north west, text width=664.0pt, align=justify] at (735.0,1505.5){\setlength{\baselineskip}{47.300000000000004pt} \par
\veryLarge {ہوا کی طرف اشارہ کرتی ہے۔ اِس سوال کا\linebreak}};%
\node[paragraphtext, anchor=north west, text width=664.0pt, align=justify] at (735.0,1462.0){\setlength{\baselineskip}{47.300000000000004pt} \par
\veryLarge {عربی السبخة متحدہ عرب امارات کا ایک\linebreak}};%
\node[paragraphtext, anchor=north west, text width=664.0pt, align=justify] at (735.0,1418.5){\setlength{\baselineskip}{47.300000000000004pt} \par
\veryLarge {تھا ۔ وہاڑی کے ضلع بننے کے بعد یہ\linebreak}};%
\node[paragraphtext, anchor=north west, text width=479.0pt, align=justify] at (732.0,1368.5){\setlength{\baselineskip}{47.300000000000004pt} \par
\veryLarge {\hspace{2em}تھا، لیکن شاہ کے\linebreak}};%
\node[paragraphtext, anchor=north west, text width=479.0pt, align=justify] at (732.0,1325.0){\setlength{\baselineskip}{47.300000000000004pt} \par
\veryLarge {بھتیجے کو گرفتارکر لیا، اور\linebreak}};%
\end{tikzpicture}%
\end{center}%
\end{document}