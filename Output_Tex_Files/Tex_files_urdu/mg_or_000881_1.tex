\documentclass[11pt]{article}%
\usepackage[T1]{fontenc}%
\usepackage[utf8]{inputenc}%
\usepackage{lmodern}%
\usepackage{textcomp}%
\usepackage{lastpage}%
\usepackage{tikz}%
\usepackage{fontspec}%
\usepackage{polyglossia}%
\usepackage[paperwidth=595pt,paperheight=842pt,margin=0pt]{geometry}%
%
\setmainlanguage{urdu}%
\setotherlanguage{english}%
\newfontfamily\arabicfont[Script=Arabic,Path=/home/vivek/Pager/fonts/urdu/Paragraph/]{Mikhak-Light}%
\newfontfamily\headerfont[Script=Arabic,Path=/home/vivek/Pager/fonts/urdu/Header/]{Zain-Black}%
\newfontfamily\paragraphfont[Script=Arabic,Path=/home/vivek/Pager/fonts/urdu/Paragraph/]{Mikhak-Light}%

    \makeatletter
    \newcommand{\zettaHuge}{\@setfontsize\zettaHuge{200}{220}}
    \newcommand{\exaHuge}{\@setfontsize\exaHuge{165}{180}}
    \newcommand{\petaHuge}{\@setfontsize\petaHuge{135}{150}}
    \newcommand{\teraHuge}{\@setfontsize\teraHuge{110}{120}}
    \newcommand{\gigaHuge}{\@setfontsize\gigaHuge{90}{100}}
    \newcommand{\megaHuge}{\@setfontsize\megaHuge{75}{85}}
    \newcommand{\superHuge}{\@setfontsize\superHuge{62}{70}}
    \newcommand{\verylarge}{\@setfontsize\verylarge{37}{42}}
    \newcommand{\veryLarge}{\@setfontsize\veryLarge{43}{49}}
    \newcommand{\veryHuge}{\@setfontsize\veryHuge{62}{70}}
    \newcommand{\alphaa}{\@setfontsize\alphaa{60}{66}}
    \newcommand{\betaa}{\@setfontsize\betaa{57}{63}}
    \newcommand{\gammaa}{\@setfontsize\gammaa{55}{61}}
    \newcommand{\deltaa}{\@setfontsize\deltaa{53}{59}}
    \newcommand{\epsilona}{\@setfontsize\epsilona{51}{57}}
    \newcommand{\zetaa}{\@setfontsize\zetaa{47}{53}}
    \newcommand{\etaa}{\@setfontsize\etaa{45}{51}}
    \newcommand{\iotaa}{\@setfontsize\iotaa{41}{47}}
    \newcommand{\kappaa}{\@setfontsize\kappaa{39}{45}}
    \newcommand{\lambdaa}{\@setfontsize\lambdaa{35}{41}}
    \newcommand{\mua}{\@setfontsize\mua{33}{39}}
    \newcommand{\nua}{\@setfontsize\nua{31}{37}}
    \newcommand{\xia}{\@setfontsize\xia{29}{35}}
    \newcommand{\pia}{\@setfontsize\pia{27}{33}}
    \newcommand{\rhoa}{\@setfontsize\rhoa{24}{30}}
    \newcommand{\sigmaa}{\@setfontsize\sigmaa{22}{28}}
    \newcommand{\taua}{\@setfontsize\taua{18}{24}}
    \newcommand{\upsilona}{\@setfontsize\upsilona{16}{22}}
    \newcommand{\phia}{\@setfontsize\phia{15}{20}}
    \newcommand{\chia}{\@setfontsize\chia{13}{18}}
    \newcommand{\psia}{\@setfontsize\psia{11}{16}}
    \newcommand{\omegaa}{\@setfontsize\omegaa{6}{7}}
    \newcommand{\oomegaa}{\@setfontsize\oomegaa{4}{5}}
    \newcommand{\ooomegaa}{\@setfontsize\ooomegaa{3}{4}}
    \newcommand{\oooomegaaa}{\@setfontsize\oooomegaaa{2}{3}}
    \makeatother
    %
%
\begin{document}%
\normalsize%
\begin{center}%
\begin{tikzpicture}[x=1pt, y=1pt]%
\node[anchor=south west, inner sep=0pt] at (0,0) {\includegraphics[width=595pt,height=842pt]{/home/vivek/Pager/images_val/mg_or_000881_1.png}};%
\tikzset{headertext/.style={font=\headerfont, text=black}}%
\tikzset{paragraphtext/.style={font=\paragraphfont, text=black}}%
\node[paragraphtext, anchor=north west, text width=8.0pt, align=left] at (56.0,441.5){\setlength{\baselineskip}{19.8pt} \par
\taua {ا\linebreak}};%
\node[paragraphtext, anchor=north west, text width=138.0pt, align=justify] at (53.0,620.5){\setlength{\baselineskip}{4.4pt} \par
\oomegaa {\hspace{2em}میں، کھیلوں کے مصنف ڈیمن رونیون نے سیلٹر کو رولر ڈربی کے قوانین کو تبدیل پر\linebreak}};%
\node[paragraphtext, anchor=north west, text width=138.0pt, align=justify] at (53.0,602.5){\setlength{\baselineskip}{4.4pt} \par
\oomegaa {میری عمر بھی اس وقت اکیاون برس کے قریب تھی ۔ اس کے باوجود میں اس سیمینار کا\linebreak}};%
\node[paragraphtext, anchor=north west, text width=138.0pt, align=justify] at (53.0,584.5){\setlength{\baselineskip}{4.4pt} \par
\oomegaa {اور حماماں والی گلی قابل ذکر ہیں ۔ آج اتنی صدیاں گزر جانے کے بعد بھی یہ دروازہ\linebreak}};%
\node[paragraphtext, anchor=north west, text width=138.0pt, align=justify] at (53.0,566.5){\setlength{\baselineskip}{4.4pt} \par
\oomegaa {کہیں نہیں پائی جاتی ہیں۔ بہروٹ غاروں کو ایک وراثتی مقام قرار دیا گیا ہے اور یہ جائزہ\linebreak}};%
\node[paragraphtext, anchor=north west, text width=138.0pt, align=justify] at (53.0,548.5){\setlength{\baselineskip}{4.4pt} \par
\oomegaa {ہے ۔ ٹز کی ماپ بارہ سے چودہ سینٹی میٹر اور وزن چھ سے بارہ گرام ہوتا ہے ۔ ٹز کی\linebreak}};%
\node[paragraphtext, anchor=north west, text width=138.0pt, align=justify] at (53.0,530.5){\setlength{\baselineskip}{4.4pt} \par
\oomegaa {ہے۔ قفلہ ضلع عربی مديرية قفلة عذر یمن کا ایک فہرست یمن کے اضلاع جو محافظہ\linebreak}};%
\node[paragraphtext, anchor=north west, text width=138.0pt, align=justify] at (53.0,512.5){\setlength{\baselineskip}{4.4pt} \par
\oomegaa {کے لیے میدان میں اکٹھے ہوتے تھے۔ تروچیراپلی قلعہ ہندوستان کا ایک خستہ حال قلعہ ہے\linebreak}};%
\node[paragraphtext, anchor=north west, text width=138.0pt, align=justify] at (53.0,494.5){\setlength{\baselineskip}{4.4pt} \par
\oomegaa {کی تعمیر کرنے پر سنجیدیگی سے سوچا گیا ۔ یہاں ایک سڑک کی تعمیر کا آغٓاذ ہوا ۔ انیس\linebreak}};%
\node[paragraphtext, anchor=north west, text width=138.0pt, align=justify] at (53.0,476.5){\setlength{\baselineskip}{4.4pt} \par
\oomegaa {کرتا ہے جو ٹاسک مینجمنٹ سے پہلے اور بعد میں کام کرتا ہے ۔ انگریزی زبان کے کورسز\linebreak}};%
\node[headertext, anchor=north west, text width=57.0pt, align=left] at (18.0,784.5){\setlength{\baselineskip}{19.8pt} \par
\taua {ملتی\linebreak}};%
\node[headertext, anchor=north west, text width=57.0pt, align=left] at (18.0,766.5){\setlength{\baselineskip}{19.8pt} \par
\taua {اعتراض\linebreak}};%
\node[headertext, anchor=north west, text width=57.0pt, align=left] at (18.0,748.5){\setlength{\baselineskip}{19.8pt} \par
\taua {کی\linebreak}};%
\node[headertext, anchor=north west, text width=77.0pt, align=left] at (86.0,749.5){\setlength{\baselineskip}{13.200000000000001pt} \par
\large {حاضر رہتے تھے ۔\linebreak}};%
\node[paragraphtext, anchor=north west, text width=80.0pt, align=left] at (83.0,683.5){\setlength{\baselineskip}{16.5pt} \par
\Large {تھے ان دو\linebreak}};%
\node[paragraphtext, anchor=north west, text width=121.0pt, align=left] at (61.0,665.5){\setlength{\baselineskip}{19.8pt} \par
\taua {جس کا نام خلصہ\linebreak}};%
\node[paragraphtext, anchor=north west, text width=81.0pt, align=justify] at (83.0,642.5){\setlength{\baselineskip}{4.4pt} \par
\oomegaa {\hspace{2em}عیسوی وفات اٹھائیس مارچ اٹھارہ سو عیسوی\linebreak}};%
\node[headertext, anchor=north west, text width=58.0pt, align=left] at (204.0,743.5){\setlength{\baselineskip}{13.200000000000001pt} \par
\large {عیسوی میں\linebreak}};%
\node[headertext, anchor=north west, text width=67.0pt, align=left] at (204.0,726.5){\setlength{\baselineskip}{4.4pt} \par
\oomegaa {چھلاکا نے کی ہے اور اسے چیلاکا اور سمیر\linebreak}};%
\node[headertext, anchor=north west, text width=67.0pt, align=left] at (204.0,722.0){\setlength{\baselineskip}{4.4pt} \par
\oomegaa {جائے تو ہوا میں آبی بخارات کو جذب کرنے کی\linebreak}};%
\node[paragraphtext, anchor=north west, text width=70.0pt, align=justify] at (200.0,715.5){\setlength{\baselineskip}{4.4pt} \par
\oomegaa {\hspace{2em}صدی کے ہیں۔ العجیبیہ عربی الجزائر\linebreak}};%
\node[paragraphtext, anchor=north west, text width=70.0pt, align=justify] at (200.0,697.5){\setlength{\baselineskip}{4.4pt} \par
\oomegaa {فوجدار کے کنٹرول میں رہا۔ تشنّج الجضن والے\linebreak}};%
\node[paragraphtext, anchor=north west, text width=70.0pt, align=justify] at (200.0,679.5){\setlength{\baselineskip}{4.4pt} \par
\oomegaa {ہوئے ۔ ہندوستانی کبڈی کی چار بڑی شکلیں\linebreak}};%
\node[paragraphtext, anchor=north west, text width=70.0pt, align=justify] at (200.0,661.5){\setlength{\baselineskip}{4.4pt} \par
\oomegaa {میل ایم آئی چوبیس ہیلی کاپٹر افغانستان میں\linebreak}};%
\node[headertext, anchor=north west, text width=72.0pt, align=left] at (202.0,628.5){\setlength{\baselineskip}{4.4pt} \par
\oomegaa {ہی تحقیقی مقالے شائع کررہا ہے۔ حسین نے انیس\linebreak}};%
\node[headertext, anchor=north west, text width=72.0pt, align=left] at (202.0,624.0){\setlength{\baselineskip}{4.4pt} \par
\oomegaa {کا علاقہ ساندل بار کہلاتا تھا بار، مقامی زبان کا\linebreak}};%
\node[paragraphtext, anchor=north west, text width=88.0pt, align=justify] at (203.0,616.5){\setlength{\baselineskip}{4.4pt} \par
\oomegaa {\hspace{2em}ہوا کی گزرگاہ چورانوے فیصد بڑھ جاتی ہے ۔ انسانی\linebreak}};%
\node[paragraphtext, anchor=north west, text width=88.0pt, align=justify] at (203.0,598.5){\setlength{\baselineskip}{4.4pt} \par
\oomegaa {بھی اس دور کی عظمت کی گواہی دیتی ہیں کنہیا لال ہندی\linebreak}};%
\node[paragraphtext, anchor=north west, text width=113.0pt, align=justify] at (204.0,575.5){\setlength{\baselineskip}{4.4pt} \par
\oomegaa {\hspace{2em}کو ھائی اسکول کے زمانے سے ہی شاعری کا شوق تھا اس نے پہلے سی\linebreak}};%
\node[paragraphtext, anchor=north west, text width=113.0pt, align=justify] at (204.0,557.5){\setlength{\baselineskip}{4.4pt} \par
\oomegaa {یاد کر لیتے۔ قرآن مجید کی تعلیم شروع ہوئی پندرہ پارے حفظ اور پندرہ ناظر\linebreak}};%
\node[paragraphtext, anchor=north west, text width=152.0pt, align=justify] at (201.0,531.5){\setlength{\baselineskip}{4.4pt} \par
\oomegaa {\hspace{2em}اہم ہے۔ کیرلا کے بیک واٹر دنیا کے تین رامسر کنونشن میں درج دلدلوں کی میزبانی کرتا اشٹمڈی\linebreak}};%
\node[paragraphtext, anchor=north west, text width=337.0pt, align=justify] at (201.0,504.5){\setlength{\baselineskip}{4.4pt} \par
\oomegaa {\hspace{2em}نندوربار، دھولیہ خاندیش کے باہریہ ناسک کے کچھ علاقوں باگلان ، مالیگاؤں اور کالون تحصیل اور اورنگ آباد میں بولی جاتی ہے ۔ تحصیلدھرن گاؤں، چوپڑا امل نیر ، ساکری ، سندکھیڑا ،ڈونڈائچا ورواڑے ، شرپور ، ،\linebreak}};%
\node[paragraphtext, anchor=north west, text width=337.0pt, align=justify] at (201.0,486.5){\setlength{\baselineskip}{4.4pt} \par
\oomegaa {میں مغیرہ قبیلہ کے لوگ تجارت میں مشہور تھے ۔ اور اس دور میں سندھ و بلوچستان کے امیر طبقوں میں شمار ہوتے تھے ۔ اسی طرح بھنڈ قبیلہ بھی اونٹوں کی تجارت عراق اور سعودیہ جیسے دوردراز علاقوں تک کرتے\linebreak}};%
\node[headertext, anchor=north west, text width=38.0pt, align=left] at (204.0,461.5){\setlength{\baselineskip}{13.200000000000001pt} \par
\large {خلیفہ\linebreak}};%
\node[headertext, anchor=north west, text width=59.0pt, align=left] at (203.0,445.5){\setlength{\baselineskip}{4.4pt} \par
\oomegaa {کو آزاد کِیا تو انڈونیشیا نے فورا جزیرہ\linebreak}};%
\node[paragraphtext, anchor=north west, text width=263.0pt, align=justify] at (274.0,447.5){\setlength{\baselineskip}{4.4pt} \par
\oomegaa {\hspace{2em}دوسرے کام اپنے ہاتھوں میں لے لیے لیکن جب شاپور بڑا ہوا تواس نے اپنے آپ کو ان کے چنگل سے چھڑایا اور مستحکم عزم وارادہ کے ساتھ ساسانیان کی کو\linebreak}};%
\node[paragraphtext, anchor=north west, text width=58.0pt, align=justify] at (204.0,430.5){\setlength{\baselineskip}{4.4pt} \par
\oomegaa {\hspace{2em}صدی کے وسط میں تعمیر کیا تھا۔\linebreak}};%
\node[headertext, anchor=north west, text width=52.0pt, align=left] at (204.0,416.5){\setlength{\baselineskip}{4.4pt} \par
\oomegaa {تو یہ قبیلہ پشاور کےراستے پختونحوا\linebreak}};%
\node[paragraphtext, anchor=north west, text width=270.0pt, align=justify] at (266.0,417.5){\setlength{\baselineskip}{4.4pt} \par
\oomegaa {\hspace{2em}کرنے کے لیے متعدد کلاسک مارشل آرٹ ایکشن فلمیں دیکھیں جسمانی تربیت کے دوران وہ زخمی بھی ہوئیں اور چوٹیں لگی تھیں ۔ رادھیکا مدن وسن بالا نے فلم لیے\linebreak}};%
\node[paragraphtext, anchor=north west, text width=270.0pt, align=justify] at (266.0,413.0){\setlength{\baselineskip}{4.4pt} \par
\oomegaa {یا چکوک میں تقسیم ہے بازار چک ایک سو ستاسی ای بی میں واقع ہے ۔ بڑی بستی محمد پورہ چک دو سو پینتیس ای بی اور چک دو سو سینتالیس ای بی میں واقع ہے ۔ جبکہ\linebreak}};%
\node[paragraphtext, anchor=north west, text width=336.0pt, align=justify] at (202.0,404.5){\setlength{\baselineskip}{4.4pt} \par
\oomegaa {\hspace{2em}زبان میں څ سے شروع ہونے والے بے شمار الفاظ موجود ہیں جیسے څھوٷ یتیم ، څھار څوق نیاز کھانے والا ، وغیرہ، کھوار اکیڈمی نے یونی کوڈ کنسورشیم کے لیے اس حرف کا یونی کوڈ تیار کیا ہے ۔ چوتھی صدی میں\linebreak}};%
\node[paragraphtext, anchor=north west, text width=336.0pt, align=justify] at (202.0,386.5){\setlength{\baselineskip}{4.4pt} \par
\oomegaa {اردو شاعری کا بڑا ذوق رکھتے تھے ۔ جناح نے مقامی سیاست دانوں کے ساتھ مل کر کام کیا، تاہم لکھنؤ میں انیس سو اڈتیس انیس سو انتالیس کے مدحِ صحابہ فسادات کے دوران لیگ کی طرف سے یکساں سیاسی آواز\linebreak}};%
\node[paragraphtext, anchor=north west, text width=336.0pt, align=justify] at (202.0,368.5){\setlength{\baselineskip}{4.4pt} \par
\oomegaa {ہے ۔ اور کبھی مختلف معانی کے لیے کسی جگہ لفظ کا حقیقی معنی مراد ہوتا ہے ۔ زمانہء جاہلیت میں بعض الفاظ کے معانی عام تھے جبکہ اسلام کی آمد کے بعد وہ الفاظ کسی ایک مفہوم کے لیے خاص ہو گئے مثلاً\linebreak}};%
\node[paragraphtext, anchor=north west, text width=336.0pt, align=justify] at (202.0,350.5){\setlength{\baselineskip}{4.4pt} \par
\oomegaa {سو بانوے میں سٹیو فوسم اور ڈین سٹیل نے رکھی تھی۔ طلاقِ مُغَلَّظہ طلاق مغلظہ مرد کا اپنی بیوی کو ایک ہی نشست میں تین طلاقیں دے دینا ۔ فرنچائزز نے دو سو کھلاڑیوں کو حاصل کرنے کے لیے پنچاس کروڑ سے\linebreak}};%
\end{tikzpicture}%
\end{center}%
\end{document}