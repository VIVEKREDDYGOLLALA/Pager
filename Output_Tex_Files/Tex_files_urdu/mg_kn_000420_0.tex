\documentclass[11pt]{article}%
\usepackage[T1]{fontenc}%
\usepackage[utf8]{inputenc}%
\usepackage{lmodern}%
\usepackage{textcomp}%
\usepackage{lastpage}%
\usepackage{tikz}%
\usepackage{fontspec}%
\usepackage{polyglossia}%
\usepackage[paperwidth=1528pt,paperheight=2084pt,margin=0pt]{geometry}%
%
\setmainlanguage{urdu}%
\setotherlanguage{english}%
\newfontfamily\arabicfont[Script=Arabic,Path=/home/vivek/Pager/fonts/urdu/Paragraph/]{Marhey-VariableFont_wght}%
\newfontfamily\headerfont[Script=Arabic,Path=/home/vivek/Pager/fonts/urdu/Header/]{Estedad-ExtraBold}%
\newfontfamily\paragraphfont[Script=Arabic,Path=/home/vivek/Pager/fonts/urdu/Paragraph/]{Marhey-VariableFont_wght}%

    \makeatletter
    \newcommand{\zettaHuge}{\@setfontsize\zettaHuge{200}{220}}
    \newcommand{\exaHuge}{\@setfontsize\exaHuge{165}{180}}
    \newcommand{\petaHuge}{\@setfontsize\petaHuge{135}{150}}
    \newcommand{\teraHuge}{\@setfontsize\teraHuge{110}{120}}
    \newcommand{\gigaHuge}{\@setfontsize\gigaHuge{90}{100}}
    \newcommand{\megaHuge}{\@setfontsize\megaHuge{75}{85}}
    \newcommand{\superHuge}{\@setfontsize\superHuge{62}{70}}
    \newcommand{\verylarge}{\@setfontsize\verylarge{37}{42}}
    \newcommand{\veryLarge}{\@setfontsize\veryLarge{43}{49}}
    \newcommand{\veryHuge}{\@setfontsize\veryHuge{62}{70}}
    \newcommand{\alphaa}{\@setfontsize\alphaa{60}{66}}
    \newcommand{\betaa}{\@setfontsize\betaa{57}{63}}
    \newcommand{\gammaa}{\@setfontsize\gammaa{55}{61}}
    \newcommand{\deltaa}{\@setfontsize\deltaa{53}{59}}
    \newcommand{\epsilona}{\@setfontsize\epsilona{51}{57}}
    \newcommand{\zetaa}{\@setfontsize\zetaa{47}{53}}
    \newcommand{\etaa}{\@setfontsize\etaa{45}{51}}
    \newcommand{\iotaa}{\@setfontsize\iotaa{41}{47}}
    \newcommand{\kappaa}{\@setfontsize\kappaa{39}{45}}
    \newcommand{\lambdaa}{\@setfontsize\lambdaa{35}{41}}
    \newcommand{\mua}{\@setfontsize\mua{33}{39}}
    \newcommand{\nua}{\@setfontsize\nua{31}{37}}
    \newcommand{\xia}{\@setfontsize\xia{29}{35}}
    \newcommand{\pia}{\@setfontsize\pia{27}{33}}
    \newcommand{\rhoa}{\@setfontsize\rhoa{24}{30}}
    \newcommand{\sigmaa}{\@setfontsize\sigmaa{22}{28}}
    \newcommand{\taua}{\@setfontsize\taua{18}{24}}
    \newcommand{\upsilona}{\@setfontsize\upsilona{16}{22}}
    \newcommand{\phia}{\@setfontsize\phia{15}{20}}
    \newcommand{\chia}{\@setfontsize\chia{13}{18}}
    \newcommand{\psia}{\@setfontsize\psia{11}{16}}
    \newcommand{\omegaa}{\@setfontsize\omegaa{6}{7}}
    \newcommand{\oomegaa}{\@setfontsize\oomegaa{4}{5}}
    \newcommand{\ooomegaa}{\@setfontsize\ooomegaa{3}{4}}
    \newcommand{\oooomegaaa}{\@setfontsize\oooomegaaa{2}{3}}
    \makeatother
    %
%
\begin{document}%
\normalsize%
\begin{center}%
\begin{tikzpicture}[x=1pt, y=1pt]%
\node[anchor=south west, inner sep=0pt] at (0,0) {\includegraphics[width=1528pt,height=2084pt]{/home/vivek/Pager/images_val/mg_kn_000420_0.png}};%
\tikzset{headertext/.style={font=\headerfont, text=black}}%
\tikzset{paragraphtext/.style={font=\paragraphfont, text=black}}%
\node[paragraphtext, anchor=north west, text width=383.0pt, align=justify] at (384.0,1682.5){\setlength{\baselineskip}{47.300000000000004pt} \par
\veryLarge {\hspace{2em}ودیا بالن بھی\linebreak}};%
\node[paragraphtext, anchor=north west, text width=383.0pt, align=justify] at (384.0,1639.0){\setlength{\baselineskip}{47.300000000000004pt} \par
\veryLarge {پر کنکریٹ کی سیڑھیوں\linebreak}};%
\node[paragraphtext, anchor=north west, text width=383.0pt, align=justify] at (384.0,1595.5){\setlength{\baselineskip}{47.300000000000004pt} \par
\veryLarge {گاہ، رائیڈو ہال کے آؤٹ\linebreak}};%
\node[paragraphtext, anchor=north west, text width=383.0pt, align=justify] at (384.0,1552.0){\setlength{\baselineskip}{47.300000000000004pt} \par
\veryLarge {میں ڈوب کر مر گئے۔\linebreak}};%
\node[paragraphtext, anchor=north west, text width=383.0pt, align=justify] at (384.0,1508.5){\setlength{\baselineskip}{47.300000000000004pt} \par
\veryLarge {شبیہ دکھانے کے قابل\linebreak}};%
\node[paragraphtext, anchor=north west, text width=383.0pt, align=justify] at (384.0,1465.0){\setlength{\baselineskip}{47.300000000000004pt} \par
\veryLarge {زبان کا مخصوص حرف\linebreak}};%
\node[paragraphtext, anchor=north west, text width=650.0pt, align=justify] at (116.0,1136.5){\setlength{\baselineskip}{47.300000000000004pt} \par
\veryLarge {\hspace{2em}افراد رہ گئے تھے جب فچیلے کو پر\linebreak}};%
\node[paragraphtext, anchor=north west, text width=650.0pt, align=justify] at (116.0,1093.0){\setlength{\baselineskip}{47.300000000000004pt} \par
\veryLarge {اپنے عمدہ اور یونیک رإٹنگ طرز کے\linebreak}};%
\node[paragraphtext, anchor=north west, text width=650.0pt, align=justify] at (116.0,1049.5){\setlength{\baselineskip}{47.300000000000004pt} \par
\veryLarge {ضلع میں واقع ہے۔ دو ہزار سترہ کی\linebreak}};%
\node[paragraphtext, anchor=north west, text width=650.0pt, align=justify] at (116.0,1006.0){\setlength{\baselineskip}{47.300000000000004pt} \par
\veryLarge {بارہ سو اِکیانوے میں کارا کی گورنری اور\linebreak}};%
\node[paragraphtext, anchor=north west, text width=650.0pt, align=justify] at (116.0,962.5){\setlength{\baselineskip}{47.300000000000004pt} \par
\veryLarge {یادگارعمارات کا تھیٹر، یونیسکو کے\linebreak}};%
\node[paragraphtext, anchor=north west, text width=387.0pt, align=justify] at (384.0,1376.5){\setlength{\baselineskip}{47.300000000000004pt} \par
\veryLarge {\hspace{2em}۔ عام طور بہترین\linebreak}};%
\node[paragraphtext, anchor=north west, text width=387.0pt, align=justify] at (384.0,1333.0){\setlength{\baselineskip}{47.300000000000004pt} \par
\veryLarge {ساتھ آپؑ کی قبر مطہر\linebreak}};%
\node[paragraphtext, anchor=north west, text width=387.0pt, align=justify] at (384.0,1289.5){\setlength{\baselineskip}{47.300000000000004pt} \par
\veryLarge {احمد زئی افغانستان\linebreak}};%
\node[paragraphtext, anchor=north west, text width=659.0pt, align=justify] at (109.0,660.5){\setlength{\baselineskip}{47.300000000000004pt} \par
\veryLarge {\hspace{2em}پترا سے جڑا ہوا تھا۔ مدورائی اڈہ\linebreak}};%
\node[paragraphtext, anchor=north west, text width=659.0pt, align=justify] at (109.0,617.0){\setlength{\baselineskip}{47.300000000000004pt} \par
\veryLarge {اورنگ زیب کی موت سے آٹھ سال قبل،\linebreak}};%
\node[paragraphtext, anchor=north west, text width=659.0pt, align=justify] at (109.0,573.5){\setlength{\baselineskip}{47.300000000000004pt} \par
\veryLarge {ہوا انسان، مبارک انسان جو کائنات کو\linebreak}};%
\node[paragraphtext, anchor=north west, text width=659.0pt, align=justify] at (109.0,530.0){\setlength{\baselineskip}{47.300000000000004pt} \par
\veryLarge {کدام متوفی سنہ ایک سو پچپن ہجری یا\linebreak}};%
\node[paragraphtext, anchor=north west, text width=663.0pt, align=justify] at (792.0,1536.5){\setlength{\baselineskip}{47.300000000000004pt} \par
\veryLarge {\hspace{2em}ویس ٖفیڈریکو لیلوئرایک جرمن کینیڈا\linebreak}};%
\node[paragraphtext, anchor=north west, text width=663.0pt, align=justify] at (792.0,1493.0){\setlength{\baselineskip}{47.300000000000004pt} \par
\veryLarge {پہلے چودا سال گزارے۔ ؐمیاں کاشف\linebreak}};%
\node[paragraphtext, anchor=north west, text width=663.0pt, align=justify] at (792.0,1449.5){\setlength{\baselineskip}{47.300000000000004pt} \par
\veryLarge {چلا ہے کہ کم عمر ماؤں کی بہنیں کم ہی\linebreak}};%
\node[paragraphtext, anchor=north west, text width=655.0pt, align=justify] at (112.0,282.5){\setlength{\baselineskip}{47.300000000000004pt} \par
\veryLarge {\hspace{2em}واقع ہے ا يڈمز ٹاؤن شپ، كاؤنٹي،\linebreak}};%
\node[paragraphtext, anchor=north west, text width=655.0pt, align=justify] at (112.0,239.0){\setlength{\baselineskip}{47.300000000000004pt} \par
\veryLarge {ٹورنامنٹ کے اپنے پہلے میچ میں وہ\linebreak}};%
\node[paragraphtext, anchor=north west, text width=655.0pt, align=justify] at (112.0,195.5){\setlength{\baselineskip}{47.300000000000004pt} \par
\veryLarge {کر رہے ہیں۔ ایک پامیر زبان ہے جو صوبہ\linebreak}};%
\node[paragraphtext, anchor=north west, text width=657.0pt, align=justify] at (792.0,1361.5){\setlength{\baselineskip}{47.300000000000004pt} \par
\veryLarge {\hspace{2em}سے ایک بااعتماد اخوت اور کا\linebreak}};%
\node[paragraphtext, anchor=north west, text width=657.0pt, align=justify] at (792.0,1318.0){\setlength{\baselineskip}{47.300000000000004pt} \par
\veryLarge {کی طرف سے حوصلہ افزائی نے تمل\linebreak}};%
\node[paragraphtext, anchor=north west, text width=657.0pt, align=justify] at (792.0,1274.5){\setlength{\baselineskip}{47.300000000000004pt} \par
\veryLarge {میں اسلام ٹھٹھہ سے ہی پھیلا تھا\linebreak}};%
\node[paragraphtext, anchor=north west, text width=657.0pt, align=justify] at (792.0,1231.0){\setlength{\baselineskip}{47.300000000000004pt} \par
\veryLarge {بچپن سے ہی گھر میں دینی علمی\linebreak}};%
\node[paragraphtext, anchor=north west, text width=650.0pt, align=justify] at (117.0,454.5){\setlength{\baselineskip}{47.300000000000004pt} \par
\veryLarge {\hspace{2em}پانچ دریا میں کام کرتے تھے ۔ بادشاہ\linebreak}};%
\node[paragraphtext, anchor=north west, text width=650.0pt, align=justify] at (117.0,411.0){\setlength{\baselineskip}{47.300000000000004pt} \par
\veryLarge {اِس نام سے موسوم کیا گیا ۔ یہ نسخہ\linebreak}};%
\node[paragraphtext, anchor=north west, text width=650.0pt, align=justify] at (117.0,367.5){\setlength{\baselineskip}{47.300000000000004pt} \par
\veryLarge {اور انگلینڈ میں۔ تلنگانہ کے کوٹیلنگلا\linebreak}};%
\node[paragraphtext, anchor=north west, text width=650.0pt, align=justify] at (114.0,900.5){\setlength{\baselineskip}{47.300000000000004pt} \par
\veryLarge {\hspace{2em}اب تو ان کا نام و نشان بھی مٹ ہے،\linebreak}};%
\node[paragraphtext, anchor=north west, text width=650.0pt, align=justify] at (114.0,857.0){\setlength{\baselineskip}{47.300000000000004pt} \par
\veryLarge {راجہ تھا ۔ کہا جاتا ہے اشوک نے اپنے ایک\linebreak}};%
\node[paragraphtext, anchor=north west, text width=650.0pt, align=justify] at (114.0,813.5){\setlength{\baselineskip}{47.300000000000004pt} \par
\veryLarge {کن جنگ میں، بابر کی تیموری افواج نے\linebreak}};%
\node[paragraphtext, anchor=north west, text width=650.0pt, align=justify] at (114.0,770.0){\setlength{\baselineskip}{47.300000000000004pt} \par
\veryLarge {گڑھ کی فوج کا کمانڈر تھا سے سہارنپور\linebreak}};%
\node[paragraphtext, anchor=north west, text width=650.0pt, align=justify] at (114.0,726.5){\setlength{\baselineskip}{47.300000000000004pt} \par
\veryLarge {بات کرنے میں ماہِر ہے وہ محلّے کی رام\linebreak}};%
\node[paragraphtext, anchor=north west, text width=661.0pt, align=justify] at (787.0,418.5){\setlength{\baselineskip}{47.300000000000004pt} \par
\veryLarge {\hspace{2em}کی زندگی پر مبنی ایک سوانحی ہے\linebreak}};%
\node[paragraphtext, anchor=north west, text width=661.0pt, align=justify] at (787.0,375.0){\setlength{\baselineskip}{47.300000000000004pt} \par
\veryLarge {خاندان اجمیر میں پناہ گزین ہوا ۔ اس\linebreak}};%
\node[paragraphtext, anchor=north west, text width=661.0pt, align=justify] at (787.0,331.5){\setlength{\baselineskip}{47.300000000000004pt} \par
\veryLarge {و مدارس میں فیض اکتساب کیا مثلًا\linebreak}};%
\node[paragraphtext, anchor=north west, text width=661.0pt, align=justify] at (787.0,288.0){\setlength{\baselineskip}{47.300000000000004pt} \par
\veryLarge {موضوعات پر ہیں ایک اہم تالیف\linebreak}};%
\node[paragraphtext, anchor=north west, text width=663.0pt, align=justify] at (114.0,1205.5){\setlength{\baselineskip}{47.300000000000004pt} \par
\veryLarge {\hspace{2em}تھا ۔ انسانیت کا گلشن یوں صدیوں\linebreak}};%
\node[paragraphtext, anchor=north west, text width=665.0pt, align=justify] at (793.0,1674.5){\setlength{\baselineskip}{47.300000000000004pt} \par
\veryLarge {\hspace{2em}ایک مقام ہے۔ یہ درخت ہے زمین اس\linebreak}};%
\node[paragraphtext, anchor=north west, text width=665.0pt, align=justify] at (793.0,1631.0){\setlength{\baselineskip}{47.300000000000004pt} \par
\veryLarge {ایڈیٹر بنے۔ اس کے بعد تیرہ سو چوبیس\linebreak}};%
\node[paragraphtext, anchor=north west, text width=665.0pt, align=justify] at (793.0,1587.5){\setlength{\baselineskip}{47.300000000000004pt} \par
\veryLarge {السلام کے ساتھیوں میں شمار ہوتے\linebreak}};%
\node[paragraphtext, anchor=north west, text width=416.0pt, align=left] at (1035.0,860.5){\setlength{\baselineskip}{47.300000000000004pt} \par
\veryLarge {نے لکھا تھا یہ ایسی ہر\linebreak}};%
\node[paragraphtext, anchor=north west, text width=416.0pt, align=left] at (1035.0,817.0){\setlength{\baselineskip}{47.300000000000004pt} \par
\veryLarge {ایک شہر ہے اور راجپوتانہ\linebreak}};%
\node[paragraphtext, anchor=north west, text width=416.0pt, align=left] at (1035.0,773.5){\setlength{\baselineskip}{47.300000000000004pt} \par
\veryLarge {شرکت کی جن میں آپ\linebreak}};%
\node[paragraphtext, anchor=north west, text width=416.0pt, align=left] at (1035.0,730.0){\setlength{\baselineskip}{47.300000000000004pt} \par
\veryLarge {لئے جدوجہد شروع کی ۔\linebreak}};%
\node[paragraphtext, anchor=north west, text width=416.0pt, align=left] at (1035.0,686.5){\setlength{\baselineskip}{47.300000000000004pt} \par
\veryLarge {یہ ام المؤمنین حضرت\linebreak}};%
\node[paragraphtext, anchor=north west, text width=416.0pt, align=left] at (1035.0,643.0){\setlength{\baselineskip}{47.300000000000004pt} \par
\veryLarge {ميرانی تھا، یہ سات\linebreak}};%
\node[paragraphtext, anchor=north west, text width=416.0pt, align=left] at (1035.0,599.5){\setlength{\baselineskip}{47.300000000000004pt} \par
\veryLarge {متھرں مارون تے الا ککر\linebreak}};%
\node[paragraphtext, anchor=north west, text width=416.0pt, align=left] at (1035.0,556.0){\setlength{\baselineskip}{47.300000000000004pt} \par
\veryLarge {کے گندھمردھن پہاڑیوں\linebreak}};%
\node[paragraphtext, anchor=north west, text width=416.0pt, align=left] at (1035.0,512.5){\setlength{\baselineskip}{47.300000000000004pt} \par
\veryLarge {کی ۔ جب دونوں میں\linebreak}};%
\node[paragraphtext, anchor=north west, text width=690.0pt, align=justify] at (776.0,243.5){\setlength{\baselineskip}{47.300000000000004pt} \par
\veryLarge {\hspace{2em}دیتا جسے عارض ممالک مذہبی\linebreak}};%
\node[paragraphtext, anchor=north west, text width=690.0pt, align=justify] at (776.0,200.0){\setlength{\baselineskip}{47.300000000000004pt} \par
\veryLarge {ردِ عمل دیتا ہے۔ آگے چل کر جگت سنگھ،\linebreak}};%
\node[paragraphtext, anchor=north west, text width=690.0pt, align=justify] at (776.0,156.5){\setlength{\baselineskip}{47.300000000000004pt} \par
\veryLarge {ایک کورٹ میں کھیل سے لطف اندوز ہوتے\linebreak}};%
\node[headertext, anchor=north west, text width=391.0pt, align=left] at (911.0,950.5){\setlength{\baselineskip}{66.0pt} \par
\alphaa {شاعر اور\linebreak}};%
\node[paragraphtext, anchor=north west, text width=652.0pt, align=left] at (791.0,1085.5){\setlength{\baselineskip}{47.300000000000004pt} \par
\veryLarge {کو شکست دینے کے مقصد سے دفاعی\linebreak}};%
\node[paragraphtext, anchor=north west, text width=652.0pt, align=left] at (791.0,1042.0){\setlength{\baselineskip}{47.300000000000004pt} \par
\veryLarge {دانہ، آلوچہ، زعفران اور اخروٹ شامل\linebreak}};%
\node[paragraphtext, anchor=north west, text width=228.0pt, align=left] at (1227.0,1152.5){\setlength{\baselineskip}{40.7pt} \par
\verylarge {یہ تانبے سے\linebreak}};%
\node[paragraphtext, anchor=north west, text width=684.0pt, align=justify] at (776.0,1993.5){\setlength{\baselineskip}{47.300000000000004pt} \par
\veryLarge {\hspace{2em}کالج یا جے ایم ڈی سی ،کراچی سندھ\linebreak}};%
\node[paragraphtext, anchor=north west, text width=684.0pt, align=justify] at (776.0,1950.0){\setlength{\baselineskip}{47.300000000000004pt} \par
\veryLarge {جامنی رنگ کا روایتی ماخذ ہے۔ بعض علی\linebreak}};%
\node[paragraphtext, anchor=north west, text width=684.0pt, align=justify] at (776.0,1906.5){\setlength{\baselineskip}{47.300000000000004pt} \par
\veryLarge {نمازی کے ساتھ ساتھ بہت سی دینی\linebreak}};%
\node[paragraphtext, anchor=north west, text width=692.0pt, align=justify] at (780.0,1852.5){\setlength{\baselineskip}{47.300000000000004pt} \par
\veryLarge {\hspace{2em}کی کوٹھی سکھ نواس میں تھے\linebreak}};%
\node[paragraphtext, anchor=north west, text width=692.0pt, align=justify] at (780.0,1809.0){\setlength{\baselineskip}{47.300000000000004pt} \par
\veryLarge {ایک سو تین، نسخے کا حجم نو انتیس ایک\linebreak}};%
\node[paragraphtext, anchor=north west, text width=692.0pt, align=justify] at (780.0,1765.5){\setlength{\baselineskip}{47.300000000000004pt} \par
\veryLarge {سکتے ہیں، بشرطیکہ آپ کا اس ملک میں\linebreak}};%
\node[paragraphtext, anchor=north west, text width=644.0pt, align=left] at (121.0,1906.5){\setlength{\baselineskip}{47.300000000000004pt} \par
\veryLarge {رہائشی علاقہ ہے۔ امام وکیع کے حلقۂ\linebreak}};%
\node[paragraphtext, anchor=north west, text width=644.0pt, align=left] at (121.0,1863.0){\setlength{\baselineskip}{47.300000000000004pt} \par
\veryLarge {کو نکلنے کے لیے مقامی ہلکے خدر کے\linebreak}};%
\node[headertext, anchor=north west, text width=563.0pt, align=left] at (173.0,1775.5){\setlength{\baselineskip}{68.2pt} \par
\superHuge {کی تھی وہ ریاست\linebreak}};%
\node[headertext, anchor=north west, text width=322.0pt, align=left] at (109.0,1983.5){\setlength{\baselineskip}{66.0pt} \par
\alphaa {کرکٙے کرکٹ\linebreak}};%
\end{tikzpicture}%
\end{center}%
\end{document}