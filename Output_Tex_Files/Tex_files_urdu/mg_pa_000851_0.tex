\documentclass[11pt]{article}%
\usepackage[T1]{fontenc}%
\usepackage[utf8]{inputenc}%
\usepackage{lmodern}%
\usepackage{textcomp}%
\usepackage{lastpage}%
\usepackage{tikz}%
\usepackage{fontspec}%
\usepackage{polyglossia}%
\usepackage[paperwidth=1042pt,paperheight=1667pt,margin=0pt]{geometry}%
%
\setmainlanguage{urdu}%
\setotherlanguage{english}%
\newfontfamily\arabicfont[Script=Arabic,Path=/home/vivek/Pager/fonts/urdu/Paragraph/]{Lemonada-Light}%
\newfontfamily\headerfont[Script=Arabic,Path=/home/vivek/Pager/fonts/urdu/Header/]{DGAgnadeen-Light}%
\newfontfamily\paragraphfont[Script=Arabic,Path=/home/vivek/Pager/fonts/urdu/Paragraph/]{Lemonada-Light}%

    \makeatletter
    \newcommand{\zettaHuge}{\@setfontsize\zettaHuge{200}{220}}
    \newcommand{\exaHuge}{\@setfontsize\exaHuge{165}{180}}
    \newcommand{\petaHuge}{\@setfontsize\petaHuge{135}{150}}
    \newcommand{\teraHuge}{\@setfontsize\teraHuge{110}{120}}
    \newcommand{\gigaHuge}{\@setfontsize\gigaHuge{90}{100}}
    \newcommand{\megaHuge}{\@setfontsize\megaHuge{75}{85}}
    \newcommand{\superHuge}{\@setfontsize\superHuge{62}{70}}
    \newcommand{\verylarge}{\@setfontsize\verylarge{37}{42}}
    \newcommand{\veryLarge}{\@setfontsize\veryLarge{43}{49}}
    \newcommand{\veryHuge}{\@setfontsize\veryHuge{62}{70}}
    \newcommand{\alphaa}{\@setfontsize\alphaa{60}{66}}
    \newcommand{\betaa}{\@setfontsize\betaa{57}{63}}
    \newcommand{\gammaa}{\@setfontsize\gammaa{55}{61}}
    \newcommand{\deltaa}{\@setfontsize\deltaa{53}{59}}
    \newcommand{\epsilona}{\@setfontsize\epsilona{51}{57}}
    \newcommand{\zetaa}{\@setfontsize\zetaa{47}{53}}
    \newcommand{\etaa}{\@setfontsize\etaa{45}{51}}
    \newcommand{\iotaa}{\@setfontsize\iotaa{41}{47}}
    \newcommand{\kappaa}{\@setfontsize\kappaa{39}{45}}
    \newcommand{\lambdaa}{\@setfontsize\lambdaa{35}{41}}
    \newcommand{\mua}{\@setfontsize\mua{33}{39}}
    \newcommand{\nua}{\@setfontsize\nua{31}{37}}
    \newcommand{\xia}{\@setfontsize\xia{29}{35}}
    \newcommand{\pia}{\@setfontsize\pia{27}{33}}
    \newcommand{\rhoa}{\@setfontsize\rhoa{24}{30}}
    \newcommand{\sigmaa}{\@setfontsize\sigmaa{22}{28}}
    \newcommand{\taua}{\@setfontsize\taua{18}{24}}
    \newcommand{\upsilona}{\@setfontsize\upsilona{16}{22}}
    \newcommand{\phia}{\@setfontsize\phia{15}{20}}
    \newcommand{\chia}{\@setfontsize\chia{13}{18}}
    \newcommand{\psia}{\@setfontsize\psia{11}{16}}
    \newcommand{\omegaa}{\@setfontsize\omegaa{6}{7}}
    \newcommand{\oomegaa}{\@setfontsize\oomegaa{4}{5}}
    \newcommand{\ooomegaa}{\@setfontsize\ooomegaa{3}{4}}
    \newcommand{\oooomegaaa}{\@setfontsize\oooomegaaa{2}{3}}
    \makeatother
    %
%
\begin{document}%
\normalsize%
\begin{center}%
\begin{tikzpicture}[x=1pt, y=1pt]%
\node[anchor=south west, inner sep=0pt] at (0,0) {\includegraphics[width=1042pt,height=1667pt]{/home/vivek/Pager/images_val/mg_pa_000851_0.png}};%
\tikzset{headertext/.style={font=\headerfont, text=black}}%
\tikzset{paragraphtext/.style={font=\paragraphfont, text=black}}%
\node[headertext, anchor=north west, text width=179.0pt, align=left] at (138.0,1577.5){\setlength{\baselineskip}{40.7pt} \par
\verylarge {گئی ہے۔\linebreak}};%
\node[paragraphtext, anchor=north west, text width=845.0pt, align=justify] at (114.0,1540.5){\setlength{\baselineskip}{47.300000000000004pt} \par
\veryLarge {\hspace{2em}کرنا تھا دو ہزار آدمیيوں پر ایک\linebreak}};%
\node[paragraphtext, anchor=north west, text width=845.0pt, align=justify] at (114.0,1497.0){\setlength{\baselineskip}{47.300000000000004pt} \par
\veryLarge {یہی تاریخ سیستان و بلوچستان میں بھی\linebreak}};%
\node[paragraphtext, anchor=north west, text width=844.0pt, align=justify] at (114.0,1440.5){\setlength{\baselineskip}{47.300000000000004pt} \par
\veryLarge {\hspace{2em}لقب ہیں،اپنے زمانہ کے امام فقیہ\linebreak}};%
\node[paragraphtext, anchor=north west, text width=844.0pt, align=justify] at (114.0,1397.0){\setlength{\baselineskip}{47.300000000000004pt} \par
\veryLarge {تھے۔ محمد علی جناح انیس سو چالیس\linebreak}};%
\node[paragraphtext, anchor=north west, text width=844.0pt, align=justify] at (114.0,1353.5){\setlength{\baselineskip}{47.300000000000004pt} \par
\veryLarge {چوبیس کے دوران ، انھیں انسائیکلوپیڈیا\linebreak}};%
\node[paragraphtext, anchor=north west, text width=844.0pt, align=justify] at (114.0,1310.0){\setlength{\baselineskip}{47.300000000000004pt} \par
\veryLarge {مخالف مذاھب ھیں ۔ خیال ھے کہ یہ\linebreak}};%
\node[paragraphtext, anchor=north west, text width=844.0pt, align=justify] at (114.0,1266.5){\setlength{\baselineskip}{47.300000000000004pt} \par
\veryLarge {اب آپ فیض الحسن ادیب کے نام سے\linebreak}};%
\node[paragraphtext, anchor=north west, text width=844.0pt, align=justify] at (114.0,1223.0){\setlength{\baselineskip}{47.300000000000004pt} \par
\veryLarge {بڑا اور طاقتور راجہ تھا ۔ کہا جاتا ہے\linebreak}};%
\node[paragraphtext, anchor=north west, text width=844.0pt, align=justify] at (114.0,1179.5){\setlength{\baselineskip}{47.300000000000004pt} \par
\veryLarge {ملتی ہے بان کے جد امجد طبقہ علما سے\linebreak}};%
\node[paragraphtext, anchor=north west, text width=844.0pt, align=justify] at (114.0,1136.0){\setlength{\baselineskip}{47.300000000000004pt} \par
\veryLarge {ہیں مارٹل کامبیٹ ڈیڈلی الاینس میں\linebreak}};%
\node[paragraphtext, anchor=north west, text width=844.0pt, align=justify] at (114.0,1092.5){\setlength{\baselineskip}{47.300000000000004pt} \par
\veryLarge {ہوتا ہے اور عوامی حمایت میں اضافہ ہوتا\linebreak}};%
\node[paragraphtext, anchor=north west, text width=844.0pt, align=justify] at (114.0,1049.0){\setlength{\baselineskip}{47.300000000000004pt} \par
\veryLarge {کی عنایت حاصل تھی ۔ بغداد میں عبد\linebreak}};%
\node[paragraphtext, anchor=north west, text width=846.0pt, align=justify] at (113.0,983.5){\setlength{\baselineskip}{47.300000000000004pt} \par
\veryLarge {\hspace{2em}فوج سہارنپور سے جلال آباد طرف\linebreak}};%
\node[paragraphtext, anchor=north west, text width=846.0pt, align=justify] at (113.0,940.0){\setlength{\baselineskip}{47.300000000000004pt} \par
\veryLarge {مملکت بنام پاکستان بنانے کے لیے\linebreak}};%
\node[paragraphtext, anchor=north west, text width=846.0pt, align=justify] at (113.0,896.5){\setlength{\baselineskip}{47.300000000000004pt} \par
\veryLarge {کتاب کلیلہ ودمنہ سے آشنا ہوئے اس\linebreak}};%
\node[paragraphtext, anchor=north west, text width=846.0pt, align=justify] at (113.0,853.0){\setlength{\baselineskip}{47.300000000000004pt} \par
\veryLarge {کشاپ ہر کہ در افتاد برافتاد اِن\linebreak}};%
\node[paragraphtext, anchor=north west, text width=846.0pt, align=justify] at (113.0,809.5){\setlength{\baselineskip}{47.300000000000004pt} \par
\veryLarge {سینتالیس سے پہلے تحصیل اور دیگر\linebreak}};%
\node[paragraphtext, anchor=north west, text width=848.0pt, align=justify] at (113.0,735.5){\setlength{\baselineskip}{47.300000000000004pt} \par
\veryLarge {\hspace{2em}مکمل افلاس کی واضع کشی\linebreak}};%
\node[paragraphtext, anchor=north west, text width=848.0pt, align=justify] at (113.0,692.0){\setlength{\baselineskip}{47.300000000000004pt} \par
\veryLarge {فرمائیں صحیح معنوں میں ہندوستان\linebreak}};%
\node[paragraphtext, anchor=north west, text width=848.0pt, align=justify] at (113.0,648.5){\setlength{\baselineskip}{47.300000000000004pt} \par
\veryLarge {اور دانشور نے بنائی اس پارٹی کے پہلے\linebreak}};%
\node[paragraphtext, anchor=north west, text width=848.0pt, align=justify] at (113.0,605.0){\setlength{\baselineskip}{47.300000000000004pt} \par
\veryLarge {بن گیا ایسے لوگوں کے لیے کہ جو پرانی\linebreak}};%
\node[paragraphtext, anchor=north west, text width=848.0pt, align=justify] at (113.0,561.5){\setlength{\baselineskip}{47.300000000000004pt} \par
\veryLarge {اور چچیرے بھائی تھے۔ محمد حسین آزاد\linebreak}};%
\node[paragraphtext, anchor=north west, text width=848.0pt, align=justify] at (113.0,518.0){\setlength{\baselineskip}{47.300000000000004pt} \par
\veryLarge {کو دن٘یَل کہتے ہیں۔ اس علاقے میں شمن\linebreak}};%
\node[paragraphtext, anchor=north west, text width=848.0pt, align=justify] at (113.0,474.5){\setlength{\baselineskip}{47.300000000000004pt} \par
\veryLarge {میں شامل تھیں ۔ انا میری ڈوڈ پیدائش\linebreak}};%
\node[paragraphtext, anchor=north west, text width=848.0pt, align=justify] at (113.0,431.0){\setlength{\baselineskip}{47.300000000000004pt} \par
\veryLarge {کو جنگ مزاحمت جاپان کے طور پر لے\linebreak}};%
\node[paragraphtext, anchor=north west, text width=848.0pt, align=justify] at (113.0,387.5){\setlength{\baselineskip}{47.300000000000004pt} \par
\veryLarge {نطام الملک آصف جاہ اول نے پورے\linebreak}};%
\node[paragraphtext, anchor=north west, text width=849.0pt, align=justify] at (114.0,334.5){\setlength{\baselineskip}{47.300000000000004pt} \par
\veryLarge {\hspace{2em}شکار پر نکلا اس نے زون کو کے\linebreak}};%
\node[paragraphtext, anchor=north west, text width=849.0pt, align=justify] at (114.0,291.0){\setlength{\baselineskip}{47.300000000000004pt} \par
\veryLarge {نام فیڈرل مالے اسٹیٹس ریلوے اور مالایان\linebreak}};%
\node[paragraphtext, anchor=north west, text width=849.0pt, align=justify] at (114.0,247.5){\setlength{\baselineskip}{47.300000000000004pt} \par
\veryLarge {صاحب امامباڑہ غفران مآب، لکھنؤ میں\linebreak}};%
\node[paragraphtext, anchor=north west, text width=849.0pt, align=justify] at (114.0,204.0){\setlength{\baselineskip}{47.300000000000004pt} \par
\veryLarge {علم و ہنر کا مرکز بن گیا چودہویں\linebreak}};%
\end{tikzpicture}%
\end{center}%
\end{document}