\documentclass[11pt]{article}%
\usepackage[T1]{fontenc}%
\usepackage[utf8]{inputenc}%
\usepackage{lmodern}%
\usepackage{textcomp}%
\usepackage{lastpage}%
\usepackage{tikz}%
\usepackage{fontspec}%
\usepackage{polyglossia}%
\usepackage[paperwidth=1152pt,paperheight=1565pt,margin=0pt]{geometry}%
%
\setmainlanguage{urdu}%
\setotherlanguage{english}%
\newfontfamily\arabicfont[Script=Arabic,Path=/home/vivek/Pager/fonts/urdu/Paragraph/]{Handjet-ExtraLight}%
\newfontfamily\headerfont[Script=Arabic,Path=/home/vivek/Pager/fonts/urdu/Header/]{Estedad-Bold}%
\newfontfamily\paragraphfont[Script=Arabic,Path=/home/vivek/Pager/fonts/urdu/Paragraph/]{Handjet-ExtraLight}%

    \makeatletter
    \newcommand{\zettaHuge}{\@setfontsize\zettaHuge{200}{220}}
    \newcommand{\exaHuge}{\@setfontsize\exaHuge{165}{180}}
    \newcommand{\petaHuge}{\@setfontsize\petaHuge{135}{150}}
    \newcommand{\teraHuge}{\@setfontsize\teraHuge{110}{120}}
    \newcommand{\gigaHuge}{\@setfontsize\gigaHuge{90}{100}}
    \newcommand{\megaHuge}{\@setfontsize\megaHuge{75}{85}}
    \newcommand{\superHuge}{\@setfontsize\superHuge{62}{70}}
    \newcommand{\verylarge}{\@setfontsize\verylarge{37}{42}}
    \newcommand{\veryLarge}{\@setfontsize\veryLarge{43}{49}}
    \newcommand{\veryHuge}{\@setfontsize\veryHuge{62}{70}}
    \newcommand{\alphaa}{\@setfontsize\alphaa{60}{66}}
    \newcommand{\betaa}{\@setfontsize\betaa{57}{63}}
    \newcommand{\gammaa}{\@setfontsize\gammaa{55}{61}}
    \newcommand{\deltaa}{\@setfontsize\deltaa{53}{59}}
    \newcommand{\epsilona}{\@setfontsize\epsilona{51}{57}}
    \newcommand{\zetaa}{\@setfontsize\zetaa{47}{53}}
    \newcommand{\etaa}{\@setfontsize\etaa{45}{51}}
    \newcommand{\iotaa}{\@setfontsize\iotaa{41}{47}}
    \newcommand{\kappaa}{\@setfontsize\kappaa{39}{45}}
    \newcommand{\lambdaa}{\@setfontsize\lambdaa{35}{41}}
    \newcommand{\mua}{\@setfontsize\mua{33}{39}}
    \newcommand{\nua}{\@setfontsize\nua{31}{37}}
    \newcommand{\xia}{\@setfontsize\xia{29}{35}}
    \newcommand{\pia}{\@setfontsize\pia{27}{33}}
    \newcommand{\rhoa}{\@setfontsize\rhoa{24}{30}}
    \newcommand{\sigmaa}{\@setfontsize\sigmaa{22}{28}}
    \newcommand{\taua}{\@setfontsize\taua{18}{24}}
    \newcommand{\upsilona}{\@setfontsize\upsilona{16}{22}}
    \newcommand{\phia}{\@setfontsize\phia{15}{20}}
    \newcommand{\chia}{\@setfontsize\chia{13}{18}}
    \newcommand{\psia}{\@setfontsize\psia{11}{16}}
    \newcommand{\omegaa}{\@setfontsize\omegaa{6}{7}}
    \newcommand{\oomegaa}{\@setfontsize\oomegaa{4}{5}}
    \newcommand{\ooomegaa}{\@setfontsize\ooomegaa{3}{4}}
    \newcommand{\oooomegaaa}{\@setfontsize\oooomegaaa{2}{3}}
    \makeatother
    %
%
\begin{document}%
\normalsize%
\begin{center}%
\begin{tikzpicture}[x=1pt, y=1pt]%
\node[anchor=south west, inner sep=0pt] at (0,0) {\includegraphics[width=1152pt,height=1565pt]{/home/vivek/Pager/images_val/mg_ml_000803_0.png}};%
\tikzset{headertext/.style={font=\headerfont, text=black}}%
\tikzset{paragraphtext/.style={font=\paragraphfont, text=black}}%
\node[paragraphtext, anchor=north west, text width=476.0pt, align=justify] at (576.0,1400.5){\setlength{\baselineskip}{47.300000000000004pt} \par
\veryLarge {\hspace{2em}۔ آنحضرت صلی اللہ وآلہ\linebreak}};%
\node[paragraphtext, anchor=north west, text width=476.0pt, align=justify] at (576.0,1357.0){\setlength{\baselineskip}{47.300000000000004pt} \par
\veryLarge {سے لے کر مغرب میں اسپین اور\linebreak}};%
\node[paragraphtext, anchor=north west, text width=476.0pt, align=justify] at (576.0,1313.5){\setlength{\baselineskip}{47.300000000000004pt} \par
\veryLarge {یہ وہ دور تھا جس میں تیلگو کے\linebreak}};%
\node[paragraphtext, anchor=north west, text width=476.0pt, align=justify] at (576.0,1270.0){\setlength{\baselineskip}{47.300000000000004pt} \par
\veryLarge {تحفۂ اثنا عشریہ کے جواب میں\linebreak}};%
\node[paragraphtext, anchor=north west, text width=476.0pt, align=justify] at (576.0,1226.5){\setlength{\baselineskip}{47.300000000000004pt} \par
\veryLarge {اورمیدان پرمشتمل ہے اورتقریبا\linebreak}};%
\node[paragraphtext, anchor=north west, text width=472.0pt, align=justify] at (84.0,700.5){\setlength{\baselineskip}{47.300000000000004pt} \par
\veryLarge {\hspace{2em}منصور الماتريدي کوبهی امت\linebreak}};%
\node[paragraphtext, anchor=north west, text width=472.0pt, align=justify] at (84.0,657.0){\setlength{\baselineskip}{47.300000000000004pt} \par
\veryLarge {لاہور کی حکومت کبھی حکمران\linebreak}};%
\node[paragraphtext, anchor=north west, text width=472.0pt, align=justify] at (84.0,613.5){\setlength{\baselineskip}{47.300000000000004pt} \par
\veryLarge {ور خواتین کی پہلی کبڈی لیگ،\linebreak}};%
\node[paragraphtext, anchor=north west, text width=472.0pt, align=justify] at (84.0,570.0){\setlength{\baselineskip}{47.300000000000004pt} \par
\veryLarge {چھلکا بھی آسانی سے توڑ سکتا ہے\linebreak}};%
\node[paragraphtext, anchor=north west, text width=472.0pt, align=justify] at (84.0,526.5){\setlength{\baselineskip}{47.300000000000004pt} \par
\veryLarge {عطیے کی درخواست کی۔ اگر ایسا\linebreak}};%
\node[paragraphtext, anchor=north west, text width=472.0pt, align=justify] at (84.0,483.0){\setlength{\baselineskip}{47.300000000000004pt} \par
\veryLarge {برلن میں ایک نمائشی کھیل کے\linebreak}};%
\node[paragraphtext, anchor=north west, text width=472.0pt, align=justify] at (84.0,439.5){\setlength{\baselineskip}{47.300000000000004pt} \par
\veryLarge {میں محلہ گراں، ڈنہ،گویرا،\linebreak}};%
\node[paragraphtext, anchor=north west, text width=472.0pt, align=justify] at (84.0,396.0){\setlength{\baselineskip}{47.300000000000004pt} \par
\veryLarge {ہو چکی ہے، نتیجتاً یہ لفظ نون\linebreak}};%
\node[paragraphtext, anchor=north west, text width=472.0pt, align=justify] at (84.0,352.5){\setlength{\baselineskip}{47.300000000000004pt} \par
\veryLarge {میں کانسی کا تمغہ جیتا، جس کی\linebreak}};%
\node[paragraphtext, anchor=north west, text width=472.0pt, align=justify] at (84.0,309.0){\setlength{\baselineskip}{47.300000000000004pt} \par
\veryLarge {تنہا فنکار ہے جس کے سوانحی\linebreak}};%
\node[paragraphtext, anchor=north west, text width=474.0pt, align=justify] at (578.0,1161.5){\setlength{\baselineskip}{47.300000000000004pt} \par
\veryLarge {\hspace{2em}غزل گلوکارہ، بنگالی اور\linebreak}};%
\node[paragraphtext, anchor=north west, text width=474.0pt, align=justify] at (578.0,1118.0){\setlength{\baselineskip}{47.300000000000004pt} \par
\veryLarge {کے اسلامی جمہوریہ کا آئین، اسلام\linebreak}};%
\node[paragraphtext, anchor=north west, text width=474.0pt, align=justify] at (578.0,1074.5){\setlength{\baselineskip}{47.300000000000004pt} \par
\veryLarge {ہیں اُن کی پہلی کتاب جوشؔ\linebreak}};%
\node[paragraphtext, anchor=north west, text width=474.0pt, align=justify] at (578.0,1031.0){\setlength{\baselineskip}{47.300000000000004pt} \par
\veryLarge {غیر ملکی اسباب،محرکات،ان کا\linebreak}};%
\node[paragraphtext, anchor=north west, text width=474.0pt, align=justify] at (578.0,987.5){\setlength{\baselineskip}{47.300000000000004pt} \par
\veryLarge {اپنی کتاب ٖتذکرۂ شعرائے ضلع\linebreak}};%
\node[paragraphtext, anchor=north west, text width=474.0pt, align=justify] at (578.0,944.0){\setlength{\baselineskip}{47.300000000000004pt} \par
\veryLarge {عیسوی میں مولانا دیوبند میں تھے\linebreak}};%
\node[paragraphtext, anchor=north west, text width=474.0pt, align=justify] at (578.0,900.5){\setlength{\baselineskip}{47.300000000000004pt} \par
\veryLarge {قبائلی ترداد کی وجہ سے پھر اس\linebreak}};%
\node[paragraphtext, anchor=north west, text width=474.0pt, align=justify] at (578.0,857.0){\setlength{\baselineskip}{47.300000000000004pt} \par
\veryLarge {ناقص پذیرائی کے ساتھ کم اوپننگ\linebreak}};%
\node[paragraphtext, anchor=north west, text width=472.0pt, align=justify] at (86.0,262.5){\setlength{\baselineskip}{47.300000000000004pt} \par
\veryLarge {\hspace{2em}۔ وہ دار العلوم دیوبند تقریباً\linebreak}};%
\node[paragraphtext, anchor=north west, text width=472.0pt, align=justify] at (86.0,219.0){\setlength{\baselineskip}{47.300000000000004pt} \par
\veryLarge {پانی سے پیدا کیا فاطرآیت پینتالیس\linebreak}};%
\node[paragraphtext, anchor=north west, text width=472.0pt, align=justify] at (86.0,175.5){\setlength{\baselineskip}{47.300000000000004pt} \par
\veryLarge {اسلم خٹک سابق وزیر موصلات\linebreak}};%
\node[paragraphtext, anchor=north west, text width=473.0pt, align=justify] at (579.0,1466.5){\setlength{\baselineskip}{47.300000000000004pt} \par
\veryLarge {\hspace{2em}نیا بنائے جانے والے تھانہ نام\linebreak}};%
\node[paragraphtext, anchor=north west, text width=471.0pt, align=justify] at (85.0,1469.5){\setlength{\baselineskip}{47.300000000000004pt} \par
\veryLarge {\hspace{2em}رمز ڈاک ہے جسے رابطہ کی\linebreak}};%
\node[paragraphtext, anchor=north west, text width=471.0pt, align=justify] at (85.0,1426.0){\setlength{\baselineskip}{47.300000000000004pt} \par
\veryLarge {کی ہے اور مسیح علیہ السلام کے\linebreak}};%
\node[paragraphtext, anchor=north west, text width=471.0pt, align=justify] at (85.0,1382.5){\setlength{\baselineskip}{47.300000000000004pt} \par
\veryLarge {تھی۔ پیر فضل شاہ نے سرگودھا کے\linebreak}};%
\node[paragraphtext, anchor=north west, text width=471.0pt, align=justify] at (85.0,1339.0){\setlength{\baselineskip}{47.300000000000004pt} \par
\veryLarge {سے تقریبا بارہ ؍ کلو میٹر کی دوری\linebreak}};%
\node[paragraphtext, anchor=north west, text width=471.0pt, align=justify] at (85.0,1295.5){\setlength{\baselineskip}{47.300000000000004pt} \par
\veryLarge {جن مندروں میں اس نوع کی\linebreak}};%
\node[paragraphtext, anchor=north west, text width=471.0pt, align=justify] at (85.0,1252.0){\setlength{\baselineskip}{47.300000000000004pt} \par
\veryLarge {تال اور جذبہ ہے جس کی رہنمائی\linebreak}};%
\node[paragraphtext, anchor=north west, text width=471.0pt, align=justify] at (85.0,1208.5){\setlength{\baselineskip}{47.300000000000004pt} \par
\veryLarge {ہے جس کا استعمال بصورتِ تذکیر\linebreak}};%
\node[paragraphtext, anchor=north west, text width=471.0pt, align=justify] at (85.0,1165.0){\setlength{\baselineskip}{47.300000000000004pt} \par
\veryLarge {کا نام آپ کے جدامجد حضرت شاہ\linebreak}};%
\node[paragraphtext, anchor=north west, text width=473.0pt] at (84.0,942.5) {\veryLarge{2. کے متعلق کہا جاتا ہے کہ فاہیانگ}};%
\node[paragraphtext, anchor=north west, text width=473.0pt] at (84.0,899.0) {\veryLarge{3. لوگ شینا زبان کو طنزاً شیاطین}};%
\node[paragraphtext, anchor=north west, text width=473.0pt] at (84.0,855.5) {\veryLarge{4. والے اس جمود کو توڑنے کی کوشش}};%
\node[paragraphtext, anchor=north west, text width=473.0pt] at (84.0,812.0) {\veryLarge{5. جدید ریسنگ کشتیوں میں چپو}};%
\node[paragraphtext, anchor=north west, text width=473.0pt] at (84.0,768.5) {\veryLarge{6. انگیز کردیتی ھے انکی تخلیقات میں}};%
\node[paragraphtext, anchor=north west, text width=473.0pt, align=justify] at (84.0,1073.5){\setlength{\baselineskip}{47.300000000000004pt} \par
\veryLarge {\hspace{2em}حاصل کیا اور اُن کی میں\linebreak}};%
\node[paragraphtext, anchor=north west, text width=473.0pt, align=justify] at (84.0,1030.0){\setlength{\baselineskip}{47.300000000000004pt} \par
\veryLarge {اور بلوچی اور سنڌي زبان کو برقرار\linebreak}};%
\node[headertext, anchor=north west, text width=103.0pt, align=left] at (610.0,1178.5){\setlength{\baselineskip}{19.8pt} \par
\taua {قسطنطنیہ بھیج\linebreak}};%
\node[headertext, anchor=north west, text width=471.0pt, align=left] at (86.0,1114.5){\setlength{\baselineskip}{19.8pt} \par
\taua {واقع، حضور صاحب کا سکھ گرودوارہ، جسے تخت سچکھنڈ سری حضور ابچل نگر\linebreak}};%
\node[paragraphtext, anchor=north west, text width=368.0pt, align=left] at (640.0,419.5){\setlength{\baselineskip}{47.300000000000004pt} \par
\veryLarge {کا وہ بازار گرم کیا ۔\linebreak}};%
\node[paragraphtext, anchor=north west, text width=368.0pt, align=left] at (640.0,376.0){\setlength{\baselineskip}{47.300000000000004pt} \par
\veryLarge {اور خوشبو والی مصنوعات\linebreak}};%
\node[paragraphtext, anchor=north west, text width=368.0pt, align=left] at (640.0,332.5){\setlength{\baselineskip}{47.300000000000004pt} \par
\veryLarge {اٹھارہ سو ستتر کو ہوا ۔\linebreak}};%
\node[paragraphtext, anchor=north west, text width=368.0pt, align=left] at (640.0,289.0){\setlength{\baselineskip}{47.300000000000004pt} \par
\veryLarge {غیر معمولی بات ہے۔ محمد\linebreak}};%
\node[paragraphtext, anchor=north west, text width=410.0pt, align=left] at (602.0,185.5){\setlength{\baselineskip}{47.300000000000004pt} \par
\veryLarge {لاہور کے قریب ایک قصبہ\linebreak}};%
\node[headertext, anchor=north west, text width=187.0pt, align=left] at (832.0,106.5){\setlength{\baselineskip}{31.900000000000002pt} \par
\xia {چالیسواں سال تھا ۔\linebreak}};%
\end{tikzpicture}%
\end{center}%
\end{document}