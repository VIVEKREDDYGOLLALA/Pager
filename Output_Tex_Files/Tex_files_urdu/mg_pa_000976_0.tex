\documentclass[11pt]{article}%
\usepackage[T1]{fontenc}%
\usepackage[utf8]{inputenc}%
\usepackage{lmodern}%
\usepackage{textcomp}%
\usepackage{lastpage}%
\usepackage{tikz}%
\usepackage{fontspec}%
\usepackage{polyglossia}%
\usepackage[paperwidth=1656pt,paperheight=2339pt,margin=0pt]{geometry}%
%
\setmainlanguage{urdu}%
\setotherlanguage{english}%
\newfontfamily\arabicfont[Script=Arabic,Path=/home/vivek/Pager/fonts/urdu/Paragraph/]{DGAgnadeen-Thin}%
\newfontfamily\headerfont[Script=Arabic,Path=/home/vivek/Pager/fonts/urdu/Header/]{Zain-Black}%
\newfontfamily\paragraphfont[Script=Arabic,Path=/home/vivek/Pager/fonts/urdu/Paragraph/]{DGAgnadeen-Thin}%

    \makeatletter
    \newcommand{\zettaHuge}{\@setfontsize\zettaHuge{200}{220}}
    \newcommand{\exaHuge}{\@setfontsize\exaHuge{165}{180}}
    \newcommand{\petaHuge}{\@setfontsize\petaHuge{135}{150}}
    \newcommand{\teraHuge}{\@setfontsize\teraHuge{110}{120}}
    \newcommand{\gigaHuge}{\@setfontsize\gigaHuge{90}{100}}
    \newcommand{\megaHuge}{\@setfontsize\megaHuge{75}{85}}
    \newcommand{\superHuge}{\@setfontsize\superHuge{62}{70}}
    \newcommand{\verylarge}{\@setfontsize\verylarge{37}{42}}
    \newcommand{\veryLarge}{\@setfontsize\veryLarge{43}{49}}
    \newcommand{\veryHuge}{\@setfontsize\veryHuge{62}{70}}
    \newcommand{\alphaa}{\@setfontsize\alphaa{60}{66}}
    \newcommand{\betaa}{\@setfontsize\betaa{57}{63}}
    \newcommand{\gammaa}{\@setfontsize\gammaa{55}{61}}
    \newcommand{\deltaa}{\@setfontsize\deltaa{53}{59}}
    \newcommand{\epsilona}{\@setfontsize\epsilona{51}{57}}
    \newcommand{\zetaa}{\@setfontsize\zetaa{47}{53}}
    \newcommand{\etaa}{\@setfontsize\etaa{45}{51}}
    \newcommand{\iotaa}{\@setfontsize\iotaa{41}{47}}
    \newcommand{\kappaa}{\@setfontsize\kappaa{39}{45}}
    \newcommand{\lambdaa}{\@setfontsize\lambdaa{35}{41}}
    \newcommand{\mua}{\@setfontsize\mua{33}{39}}
    \newcommand{\nua}{\@setfontsize\nua{31}{37}}
    \newcommand{\xia}{\@setfontsize\xia{29}{35}}
    \newcommand{\pia}{\@setfontsize\pia{27}{33}}
    \newcommand{\rhoa}{\@setfontsize\rhoa{24}{30}}
    \newcommand{\sigmaa}{\@setfontsize\sigmaa{22}{28}}
    \newcommand{\taua}{\@setfontsize\taua{18}{24}}
    \newcommand{\upsilona}{\@setfontsize\upsilona{16}{22}}
    \newcommand{\phia}{\@setfontsize\phia{15}{20}}
    \newcommand{\chia}{\@setfontsize\chia{13}{18}}
    \newcommand{\psia}{\@setfontsize\psia{11}{16}}
    \newcommand{\omegaa}{\@setfontsize\omegaa{6}{7}}
    \newcommand{\oomegaa}{\@setfontsize\oomegaa{4}{5}}
    \newcommand{\ooomegaa}{\@setfontsize\ooomegaa{3}{4}}
    \newcommand{\oooomegaaa}{\@setfontsize\oooomegaaa{2}{3}}
    \makeatother
    %
%
\begin{document}%
\normalsize%
\begin{center}%
\begin{tikzpicture}[x=1pt, y=1pt]%
\node[anchor=south west, inner sep=0pt] at (0,0) {\includegraphics[width=1656pt,height=2339pt]{/home/vivek/Pager/images_val/mg_pa_000976_0.png}};%
\tikzset{headertext/.style={font=\headerfont, text=black}}%
\tikzset{paragraphtext/.style={font=\paragraphfont, text=black}}%
\node[paragraphtext, anchor=north west, text width=641.0pt, align=justify] at (174.0,2139.5){\setlength{\baselineskip}{47.300000000000004pt} \par
\veryLarge {\hspace{2em}ہے۔ نكر ازاد كشمير کے بہت\linebreak}};%
\node[paragraphtext, anchor=north west, text width=641.0pt, align=justify] at (174.0,2096.0){\setlength{\baselineskip}{47.300000000000004pt} \par
\veryLarge {سو ترانوے ہے لہذا اس سے معلوم\linebreak}};%
\node[paragraphtext, anchor=north west, text width=641.0pt, align=justify] at (174.0,2052.5){\setlength{\baselineskip}{47.300000000000004pt} \par
\veryLarge {جاتا ہے ۔ ناتھن برنبام نے اٹھارہ سو\linebreak}};%
\node[paragraphtext, anchor=north west, text width=641.0pt, align=justify] at (174.0,2009.0){\setlength{\baselineskip}{47.300000000000004pt} \par
\veryLarge {کئی ادائیگی کی جالکاریوں کے\linebreak}};%
\node[paragraphtext, anchor=north west, text width=641.0pt, align=justify] at (174.0,1965.5){\setlength{\baselineskip}{47.300000000000004pt} \par
\veryLarge {المطلب تالیف کی ہیں ان کتابوں\linebreak}};%
\node[paragraphtext, anchor=north west, text width=641.0pt, align=justify] at (174.0,1922.0){\setlength{\baselineskip}{47.300000000000004pt} \par
\veryLarge {بھوپتی نے کئی مینس ڈبلز اور مکسڈ\linebreak}};%
\node[paragraphtext, anchor=north west, text width=641.0pt, align=justify] at (174.0,1878.5){\setlength{\baselineskip}{47.300000000000004pt} \par
\veryLarge {یا رب دو ہزار چودہ کی ہندی زبان کی\linebreak}};%
\node[paragraphtext, anchor=north west, text width=641.0pt, align=justify] at (174.0,1835.0){\setlength{\baselineskip}{47.300000000000004pt} \par
\veryLarge {مبتلا ہیں وہ بے وطن بچے ، بچے اور\linebreak}};%
\node[paragraphtext, anchor=north west, text width=641.0pt, align=justify] at (174.0,1791.5){\setlength{\baselineskip}{47.300000000000004pt} \par
\veryLarge {سید محمد گیسو نے دوسرے\linebreak}};%
\node[paragraphtext, anchor=north west, text width=693.0pt, align=justify] at (178.0,1566.5){\setlength{\baselineskip}{47.300000000000004pt} \par
\veryLarge {\hspace{2em}ملائیشیا کے طلباء اکثر کیفے\linebreak}};%
\node[paragraphtext, anchor=north west, text width=693.0pt, align=justify] at (178.0,1523.0){\setlength{\baselineskip}{47.300000000000004pt} \par
\veryLarge {ہے۔ سادھارنی ایک انتخابی طرز کی ترتیب\linebreak}};%
\node[paragraphtext, anchor=north west, text width=693.0pt, align=justify] at (178.0,1479.5){\setlength{\baselineskip}{47.300000000000004pt} \par
\veryLarge {تانگر، ہربن اور پھلاوائی، تاؤ بٹ نیلم وادی\linebreak}};%
\node[paragraphtext, anchor=north west, text width=693.0pt, align=justify] at (178.0,1436.0){\setlength{\baselineskip}{47.300000000000004pt} \par
\veryLarge {ایک اپنی جگہ بنانے کے لیے، امریکاز لیے\linebreak}};%
\node[paragraphtext, anchor=north west, text width=693.0pt, align=justify] at (178.0,1392.5){\setlength{\baselineskip}{47.300000000000004pt} \par
\veryLarge {ہوکر ترتیب وتدوین کے مرحلے میں ہے ۔\linebreak}};%
\node[paragraphtext, anchor=north west, text width=693.0pt, align=justify] at (178.0,1349.0){\setlength{\baselineskip}{47.300000000000004pt} \par
\veryLarge {اگرچہ اس بارے میں علمی بحث ہے کہ\linebreak}};%
\node[paragraphtext, anchor=north west, text width=693.0pt, align=justify] at (178.0,1305.5){\setlength{\baselineskip}{47.300000000000004pt} \par
\veryLarge {تھے۔ چوتھے طبقہ میں صغار تابعین کا\linebreak}};%
\node[paragraphtext, anchor=north west, text width=693.0pt, align=justify] at (178.0,1262.0){\setlength{\baselineskip}{47.300000000000004pt} \par
\veryLarge {شادی نہیں کرسکتا ۔ استاد اول میر نے\linebreak}};%
\node[paragraphtext, anchor=north west, text width=693.0pt, align=justify] at (178.0,1218.5){\setlength{\baselineskip}{47.300000000000004pt} \par
\veryLarge {کو محلہ مغلواڑہ، محلہ مرزائی اور محلہ\linebreak}};%
\node[paragraphtext, anchor=north west, text width=693.0pt, align=justify] at (178.0,1175.0){\setlength{\baselineskip}{47.300000000000004pt} \par
\veryLarge {ترپن عیسوی میں ابدالی کو دہلی چھوڑ\linebreak}};%
\node[paragraphtext, anchor=north west, text width=316.0pt, align=left] at (528.0,901.5){\setlength{\baselineskip}{47.300000000000004pt} \par
\veryLarge {زیر اہتمام\linebreak}};%
\node[paragraphtext, anchor=north west, text width=316.0pt, align=left] at (528.0,858.0){\setlength{\baselineskip}{47.300000000000004pt} \par
\veryLarge {۔ نئے سکے ڈھالے\linebreak}};%
\node[paragraphtext, anchor=north west, text width=316.0pt, align=left] at (528.0,814.5){\setlength{\baselineskip}{47.300000000000004pt} \par
\veryLarge {ذمہ داریوں اور\linebreak}};%
\node[headertext, anchor=north west, text width=421.0pt, align=left] at (279.0,103.5){\setlength{\baselineskip}{47.300000000000004pt} \par
\veryLarge {شادی، حالانکہ مذہب کی\linebreak}};%
\end{tikzpicture}%
\end{center}%
\end{document}