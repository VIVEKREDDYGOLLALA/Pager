\documentclass[11pt]{article}%
\usepackage[T1]{fontenc}%
\usepackage[utf8]{inputenc}%
\usepackage{lmodern}%
\usepackage{textcomp}%
\usepackage{lastpage}%
\usepackage{tikz}%
\usepackage{fontspec}%
\usepackage{polyglossia}%
\usepackage[paperwidth=595pt,paperheight=842pt,margin=0pt]{geometry}%
%
\setmainlanguage{urdu}%
\setotherlanguage{english}%
\newfontfamily\arabicfont[Script=Arabic,Path=/home/vivek/Pager/fonts/urdu/Paragraph/]{Handjet-SemiBold}%
\newfontfamily\headerfont[Script=Arabic,Path=/home/vivek/Pager/fonts/urdu/Header/]{DGAgnadeen-Regular}%
\newfontfamily\paragraphfont[Script=Arabic,Path=/home/vivek/Pager/fonts/urdu/Paragraph/]{Handjet-SemiBold}%

    \makeatletter
    \newcommand{\zettaHuge}{\@setfontsize\zettaHuge{200}{220}}
    \newcommand{\exaHuge}{\@setfontsize\exaHuge{165}{180}}
    \newcommand{\petaHuge}{\@setfontsize\petaHuge{135}{150}}
    \newcommand{\teraHuge}{\@setfontsize\teraHuge{110}{120}}
    \newcommand{\gigaHuge}{\@setfontsize\gigaHuge{90}{100}}
    \newcommand{\megaHuge}{\@setfontsize\megaHuge{75}{85}}
    \newcommand{\superHuge}{\@setfontsize\superHuge{62}{70}}
    \newcommand{\verylarge}{\@setfontsize\verylarge{37}{42}}
    \newcommand{\veryLarge}{\@setfontsize\veryLarge{43}{49}}
    \newcommand{\veryHuge}{\@setfontsize\veryHuge{62}{70}}
    \newcommand{\alphaa}{\@setfontsize\alphaa{60}{66}}
    \newcommand{\betaa}{\@setfontsize\betaa{57}{63}}
    \newcommand{\gammaa}{\@setfontsize\gammaa{55}{61}}
    \newcommand{\deltaa}{\@setfontsize\deltaa{53}{59}}
    \newcommand{\epsilona}{\@setfontsize\epsilona{51}{57}}
    \newcommand{\zetaa}{\@setfontsize\zetaa{47}{53}}
    \newcommand{\etaa}{\@setfontsize\etaa{45}{51}}
    \newcommand{\iotaa}{\@setfontsize\iotaa{41}{47}}
    \newcommand{\kappaa}{\@setfontsize\kappaa{39}{45}}
    \newcommand{\lambdaa}{\@setfontsize\lambdaa{35}{41}}
    \newcommand{\mua}{\@setfontsize\mua{33}{39}}
    \newcommand{\nua}{\@setfontsize\nua{31}{37}}
    \newcommand{\xia}{\@setfontsize\xia{29}{35}}
    \newcommand{\pia}{\@setfontsize\pia{27}{33}}
    \newcommand{\rhoa}{\@setfontsize\rhoa{24}{30}}
    \newcommand{\sigmaa}{\@setfontsize\sigmaa{22}{28}}
    \newcommand{\taua}{\@setfontsize\taua{18}{24}}
    \newcommand{\upsilona}{\@setfontsize\upsilona{16}{22}}
    \newcommand{\phia}{\@setfontsize\phia{15}{20}}
    \newcommand{\chia}{\@setfontsize\chia{13}{18}}
    \newcommand{\psia}{\@setfontsize\psia{11}{16}}
    \newcommand{\omegaa}{\@setfontsize\omegaa{6}{7}}
    \newcommand{\oomegaa}{\@setfontsize\oomegaa{4}{5}}
    \newcommand{\ooomegaa}{\@setfontsize\ooomegaa{3}{4}}
    \newcommand{\oooomegaaa}{\@setfontsize\oooomegaaa{2}{3}}
    \makeatother
    %
%
\begin{document}%
\normalsize%
\begin{center}%
\begin{tikzpicture}[x=1pt, y=1pt]%
\node[anchor=south west, inner sep=0pt] at (0,0) {\includegraphics[width=595pt,height=842pt]{/home/vivek/Pager/images_val/mg_or_000933_1.png}};%
\tikzset{headertext/.style={font=\headerfont, text=black}}%
\tikzset{paragraphtext/.style={font=\paragraphfont, text=black}}%
\node[paragraphtext, anchor=north west, text width=403.0pt, align=left] at (92.0,157.5){\setlength{\baselineskip}{19.8pt} \par
\taua {قرونِ وسطیٰ میں اس کے پانی کی سطح کافی بلند تھی جو شاید آمو\linebreak}};%
\node[paragraphtext, anchor=north west, text width=403.0pt, align=left] at (92.0,139.5){\setlength{\baselineskip}{19.8pt} \par
\taua {وسیلے سے انسانوں تک پہنچایا۔ قرآن سورۃ البروج اکیس بایس معترضین\linebreak}};%
\node[paragraphtext, anchor=north west, text width=403.0pt, align=left] at (92.0,121.5){\setlength{\baselineskip}{19.8pt} \par
\taua {جھگی والا روڈ، جتوئی روڈ، مسافر شاہ روڈ، امام بارگاہ روڈ، تھانہ روڈ،\linebreak}};%
\node[paragraphtext, anchor=north west, text width=403.0pt, align=left] at (92.0,103.5){\setlength{\baselineskip}{19.8pt} \par
\taua {واگھیلا خاندان کے بادشاہ ویرادھاوالا اور وشالدیو کے دور میں وزیر اعظم،\linebreak}};%
\node[paragraphtext, anchor=north west, text width=403.0pt, align=left] at (92.0,85.5){\setlength{\baselineskip}{19.8pt} \par
\taua {کہ یہ کوئی اِتفاقی بات نہیں ہوسکتی ۔ آغا حشر میں ضرور کوئی نہ\linebreak}};%
\node[paragraphtext, anchor=north west, text width=403.0pt, align=left] at (92.0,67.5){\setlength{\baselineskip}{19.8pt} \par
\taua {جسمانی سرگرمی ہے جو آسنوں پر مشتمل ہوتی ہے اکثر سلیس\linebreak}};%
\node[paragraphtext, anchor=north west, text width=423.0pt, align=justify] at (88.0,467.5){\setlength{\baselineskip}{19.8pt} \par
\taua {\hspace{2em}اگلے سال دہلی میں ایک بھی شیعہ باقی نہیں بچے گا ۔ اس کی افواج لوٹ\linebreak}};%
\node[paragraphtext, anchor=north west, text width=423.0pt, align=justify] at (88.0,449.5){\setlength{\baselineskip}{19.8pt} \par
\taua {کیا ۔ ایک دن افتی نسیم نے مجھ سے سوال کیا میری شناخت کیا ہے میں مرد\linebreak}};%
\node[paragraphtext, anchor=north west, text width=423.0pt, align=justify] at (88.0,431.5){\setlength{\baselineskip}{19.8pt} \par
\taua {سرسبزوشاداب اور گھنے جنگلات کاعلاقہ ہے. غار مندر میں پتھر سے کٹا ہوا\linebreak}};%
\node[paragraphtext, anchor=north west, text width=183.0pt, align=left] at (206.0,80.5){\setlength{\baselineskip}{16.5pt} \par
\Large {انسان دابۃ ہی ہے ۔ حرف صحیح قواعد\linebreak}};%
\node[paragraphtext, anchor=north west, text width=246.0pt] at (329.0,398.5) {\taua{• حضرت خضری رحمۃ اللہ علیہ اپنے زمانے کے}};%
\node[paragraphtext, anchor=north west, text width=246.0pt] at (329.0,380.5) {\taua{• کہ اتباعِ تابعین کا اصلی دور دوسری صدی}};%
\node[paragraphtext, anchor=north west, text width=246.0pt] at (329.0,362.5) {\taua{• میں شامل ہیں. جے دیپ چوپڑا کی پہلی}};%
\node[paragraphtext, anchor=north west, text width=246.0pt] at (329.0,344.5) {\taua{• کیا گیا ہے۔ انیس سو پنچاس اور انیس سو}};%
\node[paragraphtext, anchor=north west, text width=246.0pt] at (329.0,326.5) {\taua{• دیا گیا ۔ دو ہزار پندرہ کے عمومی انتخابات}};%
\node[paragraphtext, anchor=north west, text width=246.0pt] at (329.0,308.5) {\taua{• ہے، ایک ساخت جسے آیی کوٹا یا علی کوٹا کے}};%
\node[paragraphtext, anchor=north west, text width=246.0pt] at (329.0,290.5) {\taua{• لالہ رلا رام جین بی اے, پی سی ایس سینئیر}};%
\node[paragraphtext, anchor=north west, text width=246.0pt] at (329.0,272.5) {\taua{• کم تیرہ افراد ہلاک ہوئے۔ پشتین نے ڈوئچے}};%
\node[paragraphtext, anchor=north west, text width=246.0pt] at (329.0,254.5) {\taua{• متوسط اور عالی سطح کے طالب علموں کے}};%
\node[paragraphtext, anchor=north west, text width=246.0pt] at (329.0,236.5) {\taua{• سکتا ہے۔ یہ برصغیر ہند و پاک اور وسطی}};%
\node[paragraphtext, anchor=north west, text width=246.0pt] at (329.0,218.5) {\taua{• گمان ظاھر کرتا ھے وہ وادی سندھ کی}};%
\node[paragraphtext, anchor=north west, text width=105.0pt, align=left] at (245.0,722.5){\setlength{\baselineskip}{19.8pt} \par
\taua {میں وہ افغان صدر\linebreak}};%
\end{tikzpicture}%
\end{center}%
\end{document}