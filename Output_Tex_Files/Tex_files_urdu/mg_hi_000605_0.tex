\documentclass[11pt]{article}%
\usepackage[T1]{fontenc}%
\usepackage[utf8]{inputenc}%
\usepackage{lmodern}%
\usepackage{textcomp}%
\usepackage{lastpage}%
\usepackage{tikz}%
\usepackage{fontspec}%
\usepackage{polyglossia}%
\usepackage[paperwidth=1704pt,paperheight=2230pt,margin=0pt]{geometry}%
%
\setmainlanguage{urdu}%
\setotherlanguage{english}%
\newfontfamily\arabicfont[Script=Arabic,Path=/home/vivek/Pager/fonts/urdu/Paragraph/]{Zain-Light}%
\newfontfamily\headerfont[Script=Arabic,Path=/home/vivek/Pager/fonts/urdu/Header/]{Estedad-Black}%
\newfontfamily\paragraphfont[Script=Arabic,Path=/home/vivek/Pager/fonts/urdu/Paragraph/]{Zain-Light}%

    \makeatletter
    \newcommand{\zettaHuge}{\@setfontsize\zettaHuge{200}{220}}
    \newcommand{\exaHuge}{\@setfontsize\exaHuge{165}{180}}
    \newcommand{\petaHuge}{\@setfontsize\petaHuge{135}{150}}
    \newcommand{\teraHuge}{\@setfontsize\teraHuge{110}{120}}
    \newcommand{\gigaHuge}{\@setfontsize\gigaHuge{90}{100}}
    \newcommand{\megaHuge}{\@setfontsize\megaHuge{75}{85}}
    \newcommand{\superHuge}{\@setfontsize\superHuge{62}{70}}
    \newcommand{\verylarge}{\@setfontsize\verylarge{37}{42}}
    \newcommand{\veryLarge}{\@setfontsize\veryLarge{43}{49}}
    \newcommand{\veryHuge}{\@setfontsize\veryHuge{62}{70}}
    \newcommand{\alphaa}{\@setfontsize\alphaa{60}{66}}
    \newcommand{\betaa}{\@setfontsize\betaa{57}{63}}
    \newcommand{\gammaa}{\@setfontsize\gammaa{55}{61}}
    \newcommand{\deltaa}{\@setfontsize\deltaa{53}{59}}
    \newcommand{\epsilona}{\@setfontsize\epsilona{51}{57}}
    \newcommand{\zetaa}{\@setfontsize\zetaa{47}{53}}
    \newcommand{\etaa}{\@setfontsize\etaa{45}{51}}
    \newcommand{\iotaa}{\@setfontsize\iotaa{41}{47}}
    \newcommand{\kappaa}{\@setfontsize\kappaa{39}{45}}
    \newcommand{\lambdaa}{\@setfontsize\lambdaa{35}{41}}
    \newcommand{\mua}{\@setfontsize\mua{33}{39}}
    \newcommand{\nua}{\@setfontsize\nua{31}{37}}
    \newcommand{\xia}{\@setfontsize\xia{29}{35}}
    \newcommand{\pia}{\@setfontsize\pia{27}{33}}
    \newcommand{\rhoa}{\@setfontsize\rhoa{24}{30}}
    \newcommand{\sigmaa}{\@setfontsize\sigmaa{22}{28}}
    \newcommand{\taua}{\@setfontsize\taua{18}{24}}
    \newcommand{\upsilona}{\@setfontsize\upsilona{16}{22}}
    \newcommand{\phia}{\@setfontsize\phia{15}{20}}
    \newcommand{\chia}{\@setfontsize\chia{13}{18}}
    \newcommand{\psia}{\@setfontsize\psia{11}{16}}
    \newcommand{\omegaa}{\@setfontsize\omegaa{6}{7}}
    \newcommand{\oomegaa}{\@setfontsize\oomegaa{4}{5}}
    \newcommand{\ooomegaa}{\@setfontsize\ooomegaa{3}{4}}
    \newcommand{\oooomegaaa}{\@setfontsize\oooomegaaa{2}{3}}
    \makeatother
    %
%
\begin{document}%
\normalsize%
\begin{center}%
\begin{tikzpicture}[x=1pt, y=1pt]%
\node[anchor=south west, inner sep=0pt] at (0,0) {\includegraphics[width=1704pt,height=2230pt]{/home/vivek/Pager/images_val/mg_hi_000605_0.png}};%
\tikzset{headertext/.style={font=\headerfont, text=black}}%
\tikzset{paragraphtext/.style={font=\paragraphfont, text=black}}%
\node[paragraphtext, anchor=north west, text width=764.0pt, align=justify] at (59.0,1508.5){\setlength{\baselineskip}{47.300000000000004pt} \par
\veryLarge {\hspace{2em}مسموع آواز سے پہلے کی ضعیف غنّیت اردو\linebreak}};%
\node[paragraphtext, anchor=north west, text width=764.0pt, align=justify] at (59.0,1465.0){\setlength{\baselineskip}{47.300000000000004pt} \par
\veryLarge {نام قدیم رکارڈ میں آتا ھے چھ سو بیالیس میں\linebreak}};%
\node[paragraphtext, anchor=north west, text width=764.0pt, align=justify] at (59.0,1421.5){\setlength{\baselineskip}{47.300000000000004pt} \par
\veryLarge {ہے جو سر کے پچھلے حصے میں پٹے سے جڑا ہوتا ہے۔\linebreak}};%
\node[paragraphtext, anchor=north west, text width=764.0pt, align=justify] at (59.0,1378.0){\setlength{\baselineskip}{47.300000000000004pt} \par
\veryLarge {الاقوامی سیپک تکرا فیڈریشن، آئی ایس ٹی اے ایف\linebreak}};%
\node[paragraphtext, anchor=north west, text width=764.0pt, align=justify] at (59.0,1334.5){\setlength{\baselineskip}{47.300000000000004pt} \par
\veryLarge {یعنی تیری اولاد کثیر ہو ۔ سر دَ لہ بدئ مہ اوخیزہ\linebreak}};%
\node[paragraphtext, anchor=north west, text width=764.0pt, align=justify] at (59.0,1291.0){\setlength{\baselineskip}{47.300000000000004pt} \par
\veryLarge {میں آنہ آ جائے وہ سیال قبیلے سے تعلق رکهتے هیں\linebreak}};%
\node[paragraphtext, anchor=north west, text width=764.0pt, align=justify] at (59.0,1247.5){\setlength{\baselineskip}{47.300000000000004pt} \par
\veryLarge {ایک قرارداد پاس ہوٸی جس کے تحت آل انڈیا مسلم\linebreak}};%
\node[paragraphtext, anchor=north west, text width=764.0pt, align=justify] at (59.0,1204.0){\setlength{\baselineskip}{47.300000000000004pt} \par
\veryLarge {صفحات پر مبنی ہے۔ داؤدی کا مقدمہ انیس صفحے\linebreak}};%
\node[paragraphtext, anchor=north west, text width=764.0pt, align=justify] at (59.0,1160.5){\setlength{\baselineskip}{47.300000000000004pt} \par
\veryLarge {دیا ۔ اس نے شروع میں عرب حملہ آورں کو سرکوب\linebreak}};%
\node[paragraphtext, anchor=north west, text width=764.0pt, align=justify] at (59.0,1117.0){\setlength{\baselineskip}{47.300000000000004pt} \par
\veryLarge {کیا گیا ۔ اپریل دو ہزار دس میں، احمد زئی\linebreak}};%
\node[paragraphtext, anchor=north west, text width=764.0pt, align=justify] at (59.0,1073.5){\setlength{\baselineskip}{47.300000000000004pt} \par
\veryLarge {وابستہ ہیں ۔ ساتھ ہی الیکٹرونک میڈیا میں بھی\linebreak}};%
\node[paragraphtext, anchor=north west, text width=764.0pt, align=justify] at (59.0,1030.0){\setlength{\baselineskip}{47.300000000000004pt} \par
\veryLarge {تعین کرتا ہے۔ جب ہائے خفی والے الفاظ پردہ، عرصہ،\linebreak}};%
\node[paragraphtext, anchor=north west, text width=764.0pt, align=justify] at (59.0,986.5){\setlength{\baselineskip}{47.300000000000004pt} \par
\veryLarge {اجنبیت رفع ہو گئی ۔ فرانسیسی رجوع مکرر\linebreak}};%
\node[paragraphtext, anchor=north west, text width=764.0pt, align=justify] at (59.0,943.0){\setlength{\baselineskip}{47.300000000000004pt} \par
\veryLarge {ایک جلال آ گیا آپؒ ساری زندگی سفرمیں رہے\linebreak}};%
\node[paragraphtext, anchor=north west, text width=764.0pt, align=justify] at (69.0,865.5){\setlength{\baselineskip}{47.300000000000004pt} \par
\veryLarge {\hspace{2em}جاتی ہے۔ اس دور میں 'ایران شاہ کے شعلے' بھی\linebreak}};%
\node[paragraphtext, anchor=north west, text width=764.0pt, align=justify] at (69.0,822.0){\setlength{\baselineskip}{47.300000000000004pt} \par
\veryLarge {وہ پہلی زبان جس نے اولین ابجدی حروف وضع کیے\linebreak}};%
\node[paragraphtext, anchor=north west, text width=764.0pt, align=justify] at (69.0,778.5){\setlength{\baselineskip}{47.300000000000004pt} \par
\veryLarge {پندرہ سو سینتیس سے دسمبر اُنیس سو اکستھ کے\linebreak}};%
\node[paragraphtext, anchor=north west, text width=764.0pt, align=justify] at (69.0,735.0){\setlength{\baselineskip}{47.300000000000004pt} \par
\veryLarge {کیا ہے اور وہ فروعونی سرزمین پر اجنبی تھا ۔ کاغذ\linebreak}};%
\node[paragraphtext, anchor=north west, text width=764.0pt, align=justify] at (69.0,691.5){\setlength{\baselineskip}{47.300000000000004pt} \par
\veryLarge {اسلامک یونیورسٹی جکارتا انڈونیشیائی یونیورسٹاس\linebreak}};%
\node[paragraphtext, anchor=north west, text width=764.0pt, align=justify] at (69.0,648.0){\setlength{\baselineskip}{47.300000000000004pt} \par
\veryLarge {ہے جو صرف کھوار اور دیگر چترالی زبانوں میں\linebreak}};%
\node[paragraphtext, anchor=north west, text width=764.0pt, align=justify] at (69.0,604.5){\setlength{\baselineskip}{47.300000000000004pt} \par
\veryLarge {سے ایک جنگ میں روم کا بادشاہ یولیانوس قتل\linebreak}};%
\node[paragraphtext, anchor=north west, text width=764.0pt, align=justify] at (69.0,561.0){\setlength{\baselineskip}{47.300000000000004pt} \par
\veryLarge {کوئی فائدہ نہیں ہوا لیکن ٓآئندہ کے لیے اسے تجربہ\linebreak}};%
\node[paragraphtext, anchor=north west, text width=764.0pt, align=justify] at (69.0,517.5){\setlength{\baselineskip}{47.300000000000004pt} \par
\veryLarge {مشہور ہیں اور بہت سے سیاحوں اور مذہبی عقیدت\linebreak}};%
\node[paragraphtext, anchor=north west, text width=764.0pt, align=justify] at (69.0,474.0){\setlength{\baselineskip}{47.300000000000004pt} \par
\veryLarge {،گاخ ، غوختنہ ،کو ،گوختنہ ، جبکہ تختیدل ،کو ،\linebreak}};%
\node[paragraphtext, anchor=north west, text width=738.0pt, align=justify] at (871.0,953.5){\setlength{\baselineskip}{47.300000000000004pt} \par
\veryLarge {\hspace{2em}کوچ کر گئے ۔ جس سے کلاچی ،\linebreak}};%
\node[paragraphtext, anchor=north west, text width=738.0pt, align=justify] at (871.0,910.0){\setlength{\baselineskip}{47.300000000000004pt} \par
\veryLarge {کے خلاف شروع سے ہی فنکاروں کو بہت سی\linebreak}};%
\node[paragraphtext, anchor=north west, text width=738.0pt, align=justify] at (871.0,866.5){\setlength{\baselineskip}{47.300000000000004pt} \par
\veryLarge {بھی ایک حقیقت ہے کہ وہ عُمر بھر آٹے دال کی\linebreak}};%
\node[paragraphtext, anchor=north west, text width=738.0pt, align=justify] at (871.0,823.0){\setlength{\baselineskip}{47.300000000000004pt} \par
\veryLarge {کروڑ سے زیادہ کی رقم خرچ کی۔ سلمبم میں\linebreak}};%
\node[paragraphtext, anchor=north west, text width=738.0pt, align=justify] at (871.0,779.5){\setlength{\baselineskip}{47.300000000000004pt} \par
\veryLarge {بیٹے، مرتضیٰ مصطفٰی، قادر مصطفٰی، ربّانی\linebreak}};%
\node[paragraphtext, anchor=north west, text width=738.0pt, align=justify] at (871.0,736.0){\setlength{\baselineskip}{47.300000000000004pt} \par
\veryLarge {ایم سی جی میں کھیلا گیا ۔ ٹی ٹوئنٹی\linebreak}};%
\node[paragraphtext, anchor=north west, text width=738.0pt, align=justify] at (871.0,692.5){\setlength{\baselineskip}{47.300000000000004pt} \par
\veryLarge {رومن شہنشاہ تھا جو حضرت عیسیٰ علیہ السلام\linebreak}};%
\node[paragraphtext, anchor=north west, text width=738.0pt, align=justify] at (871.0,649.0){\setlength{\baselineskip}{47.300000000000004pt} \par
\veryLarge {کرتے ہیں۔ جہاں سے فراغت کے بعد انیس سو\linebreak}};%
\node[paragraphtext, anchor=north west, text width=738.0pt, align=justify] at (871.0,605.5){\setlength{\baselineskip}{47.300000000000004pt} \par
\veryLarge {عنوان بھی، کیونکہ اس میں منافقین ہی کے طرز\linebreak}};%
\node[paragraphtext, anchor=north west, text width=771.0pt, align=justify] at (70.0,410.5){\setlength{\baselineskip}{47.300000000000004pt} \par
\veryLarge {\hspace{2em}۔ اُردو ناقدین میں صرف دو یعنی مظفر سیّد\linebreak}};%
\node[paragraphtext, anchor=north west, text width=771.0pt, align=justify] at (70.0,367.0){\setlength{\baselineskip}{47.300000000000004pt} \par
\veryLarge {انحصاری کی علامت بن گئی ۔ سنہ انیس سو بیس\linebreak}};%
\node[paragraphtext, anchor=north west, text width=771.0pt, align=justify] at (70.0,323.5){\setlength{\baselineskip}{47.300000000000004pt} \par
\veryLarge {وجود میں آِئی جو بعد میں جنوبی یمن کی بنیاد بنی\linebreak}};%
\node[paragraphtext, anchor=north west, text width=771.0pt, align=justify] at (70.0,280.0){\setlength{\baselineskip}{47.300000000000004pt} \par
\veryLarge {کے ساتھ گھر چھوڑ آو لیکن تری خان اس کے\linebreak}};%
\node[paragraphtext, anchor=north west, text width=771.0pt, align=justify] at (70.0,236.5){\setlength{\baselineskip}{47.300000000000004pt} \par
\veryLarge {مرہٹوں نے بھیسروں پنت تانیتا کو سہارنپور کے\linebreak}};%
\node[paragraphtext, anchor=north west, text width=771.0pt, align=justify] at (70.0,193.0){\setlength{\baselineskip}{47.300000000000004pt} \par
\veryLarge {نامزد کیا تھا ۔ انہوں نے شہر کو قلعہ بند کیا اور\linebreak}};%
\node[paragraphtext, anchor=north west, text width=730.0pt, align=justify] at (880.0,546.5){\setlength{\baselineskip}{47.300000000000004pt} \par
\veryLarge {\hspace{2em}اوائل مغل عہد میں یہاں مغل فوج پیادے\linebreak}};%
\node[paragraphtext, anchor=north west, text width=730.0pt, align=justify] at (880.0,503.0){\setlength{\baselineskip}{47.300000000000004pt} \par
\veryLarge {اویلبل' ظاہر کیا جائے گا۔ وادیٴ شیوک اور نبرا\linebreak}};%
\node[paragraphtext, anchor=north west, text width=730.0pt, align=justify] at (880.0,459.5){\setlength{\baselineskip}{47.300000000000004pt} \par
\veryLarge {حیثیت سے مغربی مہاراشٹر پر حکومت کی، جو\linebreak}};%
\node[paragraphtext, anchor=north west, text width=730.0pt, align=justify] at (880.0,416.0){\setlength{\baselineskip}{47.300000000000004pt} \par
\veryLarge {کے بعد پنجاب یونیورسٹی لاہور سے بی اے اور\linebreak}};%
\node[paragraphtext, anchor=north west, text width=730.0pt, align=justify] at (880.0,372.5){\setlength{\baselineskip}{47.300000000000004pt} \par
\veryLarge {بن ثعلبہ ازدی قبیلہ ازدشنوءہ سے ہیں، اسلام سے\linebreak}};%
\node[paragraphtext, anchor=north west, text width=730.0pt, align=justify] at (880.0,329.0){\setlength{\baselineskip}{47.300000000000004pt} \par
\veryLarge {دس جنوری انیس سو چھیاسٹھ عیسوی کو\linebreak}};%
\node[paragraphtext, anchor=north west, text width=730.0pt, align=justify] at (880.0,285.5){\setlength{\baselineskip}{47.300000000000004pt} \par
\veryLarge {انہیں حدیث اوراصول حدیث پرکمال دسترس\linebreak}};%
\node[paragraphtext, anchor=north west, text width=730.0pt, align=justify] at (880.0,242.0){\setlength{\baselineskip}{47.300000000000004pt} \par
\veryLarge {راغب مرادآبادی کو حسن کارکردگی پر صدارتی\linebreak}};%
\node[paragraphtext, anchor=north west, text width=730.0pt, align=justify] at (880.0,198.5){\setlength{\baselineskip}{47.300000000000004pt} \par
\veryLarge {اور ادویات دان ہیں جنھوں نے انیس سو اٹھانوے\linebreak}};%
\node[paragraphtext, anchor=north west, text width=746.0pt, align=justify] at (863.0,1139.5){\setlength{\baselineskip}{47.300000000000004pt} \par
\veryLarge {\hspace{2em}ہوگی کہ وندوز آٹھ میں سندھی کو کیا\linebreak}};%
\node[paragraphtext, anchor=north west, text width=746.0pt, align=justify] at (863.0,1096.0){\setlength{\baselineskip}{47.300000000000004pt} \par
\veryLarge {بیس سال تو اس نے سنبھل میں گزارے لیکن آخری\linebreak}};%
\node[paragraphtext, anchor=north west, text width=746.0pt, align=justify] at (863.0,1052.5){\setlength{\baselineskip}{47.300000000000004pt} \par
\veryLarge {کرنے کے باوجود مواصلاتی رابطہ قائم نہ ہونے کی\linebreak}};%
\node[paragraphtext, anchor=north west, text width=746.0pt, align=justify] at (863.0,1009.0){\setlength{\baselineskip}{47.300000000000004pt} \par
\veryLarge {شائع کیا جس میں ہارر فکشن کے دو عناصر\linebreak}};%
\node[paragraphtext, anchor=north west, text width=328.0pt, align=left] at (659.0,2179.5){\setlength{\baselineskip}{47.300000000000004pt} \par
\veryLarge {کہنے لگا کہ اب میں\linebreak}};%
\node[paragraphtext, anchor=north west, text width=328.0pt, align=left] at (659.0,2136.0){\setlength{\baselineskip}{47.300000000000004pt} \par
\veryLarge {ان کے کوچنی بنیاد اور\linebreak}};%
\node[paragraphtext, anchor=north west, text width=328.0pt, align=left] at (659.0,2092.5){\setlength{\baselineskip}{47.300000000000004pt} \par
\veryLarge {پي اے اگرچہ البيروني\linebreak}};%
\node[paragraphtext, anchor=north west, text width=328.0pt, align=left] at (659.0,2049.0){\setlength{\baselineskip}{47.300000000000004pt} \par
\veryLarge {اور سفیان ثوری بھی\linebreak}};%
\node[paragraphtext, anchor=north west, text width=328.0pt, align=left] at (659.0,2005.5){\setlength{\baselineskip}{47.300000000000004pt} \par
\veryLarge {ایلیٹ کو جز وقتی\linebreak}};%
\node[headertext, anchor=north west, text width=1353.0pt, align=left, text=black] at (152.0,1771.5){\setlength{\baselineskip}{82.5pt} \par
\megaHuge {ہے۔ ذوالخلصہ عرب کے قبیلہ خثعم کے بت\linebreak}};%
\node[paragraphtext, anchor=north west, text width=268.0pt, align=left] at (701.0,1637.5){\setlength{\baselineskip}{47.300000000000004pt} \par
\veryLarge {کے ساحل\linebreak}};%
\node[paragraphtext, anchor=north west, text width=522.0pt, align=left] at (1084.0,1497.5){\setlength{\baselineskip}{42.900000000000006pt} \par
\kappaa {نہری نظام دست یاب نہیں ہوتا یہ بار\linebreak}};%
\node[paragraphtext, anchor=north west, text width=522.0pt, align=left] at (1084.0,1454.0){\setlength{\baselineskip}{42.900000000000006pt} \par
\kappaa {قیام ہے، جن میں پراگ میں موجود\linebreak}};%
\node[paragraphtext, anchor=north west, text width=522.0pt, align=left] at (1084.0,1410.5){\setlength{\baselineskip}{42.900000000000006pt} \par
\kappaa {ایورسٹ سر کرنے کا اعزاز ایک سولہ\linebreak}};%
\node[paragraphtext, anchor=north west, text width=522.0pt, align=left] at (1084.0,1367.0){\setlength{\baselineskip}{42.900000000000006pt} \par
\kappaa {زوال شروع ہو گیا۔ اینجلا آئسا ڈورا\linebreak}};%
\node[paragraphtext, anchor=north west, text width=522.0pt, align=left] at (1084.0,1323.5){\setlength{\baselineskip}{42.900000000000006pt} \par
\kappaa {جیتنے والے اسکواڈ کا ایک اہم رکن تھا\linebreak}};%
\node[paragraphtext, anchor=north west, text width=522.0pt, align=left] at (1084.0,1280.0){\setlength{\baselineskip}{42.900000000000006pt} \par
\kappaa {عاجز جمالی کو ھائی اسکول کے زمانے\linebreak}};%
\node[paragraphtext, anchor=north west, text width=522.0pt, align=left] at (1084.0,1236.5){\setlength{\baselineskip}{42.900000000000006pt} \par
\kappaa {ہان سلطنت کے زمانے میں رہتے تھے ۔\linebreak}};%
\node[headertext, anchor=north west, text width=248.0pt, align=left] at (0.0,2079.5){\setlength{\baselineskip}{47.300000000000004pt} \par
\veryLarge {پر اسماعیلی\linebreak}};%
\node[headertext, anchor=north west, text width=248.0pt, align=left] at (0.0,2036.0){\setlength{\baselineskip}{47.300000000000004pt} \par
\veryLarge {کوکس کمشنر\linebreak}};%
\node[headertext, anchor=north west, text width=697.0pt, align=left] at (502.0,128.5){\setlength{\baselineskip}{47.300000000000004pt} \par
\veryLarge {پختگی پیداکرلی اورحالات نے مدرسہ قائم کرنے\linebreak}};%
\node[headertext, anchor=north west, text width=697.0pt, align=left] at (502.0,85.0){\setlength{\baselineskip}{47.300000000000004pt} \par
\veryLarge {کیے ۔ ان کے کئی سارے تاریخی اور کیلینڈری\linebreak}};%
\node[paragraphtext, anchor=north west, text width=350.0pt, align=left] at (863.0,1195.5){\setlength{\baselineskip}{42.900000000000006pt} \par
\kappaa {کا خیال ہے کہ تراشنا پہلی\linebreak}};%
\node[paragraphtext, anchor=north west, text width=773.0pt, align=left] at (838.0,1506.5){\setlength{\baselineskip}{47.300000000000004pt} \par
\veryLarge {کہنا ہے یہ سب سیتھی نہیں تھے مگر ایرانوں کی\linebreak}};%
\node[paragraphtext, anchor=north west, text width=773.0pt, align=left] at (838.0,1463.0){\setlength{\baselineskip}{47.300000000000004pt} \par
\veryLarge {ان بیمار لوگوں سے امتیاز ہٹانے پر زور دیا گیا تھا ۔\linebreak}};%
\node[paragraphtext, anchor=north west, text width=773.0pt, align=left] at (838.0,1419.5){\setlength{\baselineskip}{47.300000000000004pt} \par
\veryLarge {کے مرکزی جامع مسجدمحلہ گجران میں مولوی\linebreak}};%
\node[paragraphtext, anchor=north west, text width=773.0pt, align=left] at (838.0,1376.0){\setlength{\baselineskip}{47.300000000000004pt} \par
\veryLarge {سے ایک لڑکی سلطانہ برآمد ہوتی ہے جو کلیانہ کے\linebreak}};%
\node[paragraphtext, anchor=north west, text width=773.0pt, align=left] at (838.0,1332.5){\setlength{\baselineskip}{47.300000000000004pt} \par
\veryLarge {بروز بدھ سترہ صفر المظفر گیارہ سو اٹھانوے ہجری\linebreak}};%
\node[paragraphtext, anchor=north west, text width=773.0pt, align=left] at (838.0,1289.0){\setlength{\baselineskip}{47.300000000000004pt} \par
\veryLarge {تبت، برہما، دکن اور سیلون سنگلدیپ میں بھکشو\linebreak}};%
\node[paragraphtext, anchor=north west, text width=773.0pt, align=left] at (838.0,1245.5){\setlength{\baselineskip}{47.300000000000004pt} \par
\veryLarge {تھا، حالانکہ اس کا بیٹا، کمارپال، اپنے بیشتر علاقوں\linebreak}};%
\node[paragraphtext, anchor=north west, text width=773.0pt, align=left] at (838.0,1202.0){\setlength{\baselineskip}{47.300000000000004pt} \par
\veryLarge {بعد ازاں، اُنہوں نے اپنا ذاتی تخلیقاتی ادارہ ایم ڈی\linebreak}};%
\end{tikzpicture}%
\end{center}%
\end{document}