\documentclass[11pt]{article}%
\usepackage[T1]{fontenc}%
\usepackage[utf8]{inputenc}%
\usepackage{lmodern}%
\usepackage{textcomp}%
\usepackage{lastpage}%
\usepackage{tikz}%
\usepackage{fontspec}%
\usepackage{polyglossia}%
\usepackage[paperwidth=1056pt,paperheight=1606pt,margin=0pt]{geometry}%
%
\setmainlanguage{urdu}%
\setotherlanguage{english}%
\newfontfamily\arabicfont[Script=Arabic,Path=/home/vivek/Pager/fonts/urdu/Paragraph/]{ScheherazadeNew-Regular}%
\newfontfamily\headerfont[Script=Arabic,Path=/home/vivek/Pager/fonts/urdu/Header/]{Marhey-SemiBold}%
\newfontfamily\paragraphfont[Script=Arabic,Path=/home/vivek/Pager/fonts/urdu/Paragraph/]{ScheherazadeNew-Regular}%

    \makeatletter
    \newcommand{\zettaHuge}{\@setfontsize\zettaHuge{200}{220}}
    \newcommand{\exaHuge}{\@setfontsize\exaHuge{165}{180}}
    \newcommand{\petaHuge}{\@setfontsize\petaHuge{135}{150}}
    \newcommand{\teraHuge}{\@setfontsize\teraHuge{110}{120}}
    \newcommand{\gigaHuge}{\@setfontsize\gigaHuge{90}{100}}
    \newcommand{\megaHuge}{\@setfontsize\megaHuge{75}{85}}
    \newcommand{\superHuge}{\@setfontsize\superHuge{62}{70}}
    \newcommand{\verylarge}{\@setfontsize\verylarge{37}{42}}
    \newcommand{\veryLarge}{\@setfontsize\veryLarge{43}{49}}
    \newcommand{\veryHuge}{\@setfontsize\veryHuge{62}{70}}
    \newcommand{\alphaa}{\@setfontsize\alphaa{60}{66}}
    \newcommand{\betaa}{\@setfontsize\betaa{57}{63}}
    \newcommand{\gammaa}{\@setfontsize\gammaa{55}{61}}
    \newcommand{\deltaa}{\@setfontsize\deltaa{53}{59}}
    \newcommand{\epsilona}{\@setfontsize\epsilona{51}{57}}
    \newcommand{\zetaa}{\@setfontsize\zetaa{47}{53}}
    \newcommand{\etaa}{\@setfontsize\etaa{45}{51}}
    \newcommand{\iotaa}{\@setfontsize\iotaa{41}{47}}
    \newcommand{\kappaa}{\@setfontsize\kappaa{39}{45}}
    \newcommand{\lambdaa}{\@setfontsize\lambdaa{35}{41}}
    \newcommand{\mua}{\@setfontsize\mua{33}{39}}
    \newcommand{\nua}{\@setfontsize\nua{31}{37}}
    \newcommand{\xia}{\@setfontsize\xia{29}{35}}
    \newcommand{\pia}{\@setfontsize\pia{27}{33}}
    \newcommand{\rhoa}{\@setfontsize\rhoa{24}{30}}
    \newcommand{\sigmaa}{\@setfontsize\sigmaa{22}{28}}
    \newcommand{\taua}{\@setfontsize\taua{18}{24}}
    \newcommand{\upsilona}{\@setfontsize\upsilona{16}{22}}
    \newcommand{\phia}{\@setfontsize\phia{15}{20}}
    \newcommand{\chia}{\@setfontsize\chia{13}{18}}
    \newcommand{\psia}{\@setfontsize\psia{11}{16}}
    \newcommand{\omegaa}{\@setfontsize\omegaa{6}{7}}
    \newcommand{\oomegaa}{\@setfontsize\oomegaa{4}{5}}
    \newcommand{\ooomegaa}{\@setfontsize\ooomegaa{3}{4}}
    \newcommand{\oooomegaaa}{\@setfontsize\oooomegaaa{2}{3}}
    \makeatother
    %
%
\begin{document}%
\normalsize%
\begin{center}%
\begin{tikzpicture}[x=1pt, y=1pt]%
\node[anchor=south west, inner sep=0pt] at (0,0) {\includegraphics[width=1056pt,height=1606pt]{/home/vivek/Pager/images_val/mg_as_000576_0.png}};%
\tikzset{headertext/.style={font=\headerfont, text=black}}%
\tikzset{paragraphtext/.style={font=\paragraphfont, text=black}}%
\node[headertext, anchor=north west, text width=156.0pt, align=left] at (447.0,1506.5){\setlength{\baselineskip}{38.5pt} \par
\lambdaa {پر سے\linebreak}};%
\node[paragraphtext, anchor=north west, text width=845.0pt, align=justify] at (104.0,1455.5){\setlength{\baselineskip}{47.300000000000004pt} \par
\veryLarge {\hspace{2em}نئے دور میں داخل ہوا جسے ماہرین محجر کا\linebreak}};%
\node[paragraphtext, anchor=north west, text width=845.0pt, align=justify] at (104.0,1412.0){\setlength{\baselineskip}{47.300000000000004pt} \par
\veryLarge {میں، جے دیو ورما ایک مشہور شاعر ہیں جو ایک\linebreak}};%
\node[paragraphtext, anchor=north west, text width=845.0pt, align=justify] at (104.0,1368.5){\setlength{\baselineskip}{47.300000000000004pt} \par
\veryLarge {ایل ایوارڈ برائے سال دو ہزار پندرہ میں فکشن کی\linebreak}};%
\node[paragraphtext, anchor=north west, text width=845.0pt, align=justify] at (104.0,1325.0){\setlength{\baselineskip}{47.300000000000004pt} \par
\veryLarge {کی فوجوں کی بسال تک راہنمائی کی تھی۔ تاہم\linebreak}};%
\node[paragraphtext, anchor=north west, text width=845.0pt, align=justify] at (104.0,1281.5){\setlength{\baselineskip}{47.300000000000004pt} \par
\veryLarge {عمارت میں ہیں تو زمین پر کسی پلنگ یا مضبوط میز\linebreak}};%
\node[paragraphtext, anchor=north west, text width=845.0pt, align=justify] at (104.0,1238.0){\setlength{\baselineskip}{47.300000000000004pt} \par
\veryLarge {میں جرمن زبان میں شائع کیا۔ اس تصنیف میں اس نے\linebreak}};%
\node[paragraphtext, anchor=north west, text width=845.0pt, align=justify] at (104.0,1194.5){\setlength{\baselineskip}{47.300000000000004pt} \par
\veryLarge {کرتے تھے اور ہر بار غلاموں کو شکست ہوتی تھی ۔\linebreak}};%
\node[paragraphtext, anchor=north west, text width=845.0pt, align=justify] at (104.0,1151.0){\setlength{\baselineskip}{47.300000000000004pt} \par
\veryLarge {ہے۔ زاویہ کنتہ ضلع عربی دائرة زاوية كنتة الجزائر کا\linebreak}};%
\node[paragraphtext, anchor=north west, text width=845.0pt, align=justify] at (104.0,1107.5){\setlength{\baselineskip}{47.300000000000004pt} \par
\veryLarge {بنا کر لگائی جانے والی صناعی ہڈی میں موجود\linebreak}};%
\node[paragraphtext, anchor=north west, text width=845.0pt, align=justify] at (104.0,1064.0){\setlength{\baselineskip}{47.300000000000004pt} \par
\veryLarge {میں بیروت، دمشق کے علاوہ، جامعہ پشاور، پاکستان\linebreak}};%
\node[paragraphtext, anchor=north west, text width=845.0pt, align=justify] at (104.0,1020.5){\setlength{\baselineskip}{47.300000000000004pt} \par
\veryLarge {بدنام زمانہ ایف ایس ایف فیڈرل سیکورٹی فورس کو\linebreak}};%
\node[paragraphtext, anchor=north west, text width=845.0pt, align=justify] at (104.0,977.0){\setlength{\baselineskip}{47.300000000000004pt} \par
\veryLarge {ستارہ سے زائد ناولز لکھےٗ جن کے نام یوں ہیں پہاڑی\linebreak}};%
\node[paragraphtext, anchor=north west, text width=845.0pt, align=justify] at (104.0,933.5){\setlength{\baselineskip}{47.300000000000004pt} \par
\veryLarge {رہا اور انیس سو اٹھارہ میں سوشلسٹ پارٹی بنی اس نے\linebreak}};%
\node[paragraphtext, anchor=north west, text width=845.0pt, align=justify] at (104.0,890.0){\setlength{\baselineskip}{47.300000000000004pt} \par
\veryLarge {اسماعیل سے حاصل کی بعد ازاں تحصیل اوگی گاؤں\linebreak}};%
\node[paragraphtext, anchor=north west, text width=845.0pt, align=justify] at (104.0,846.5){\setlength{\baselineskip}{47.300000000000004pt} \par
\veryLarge {بابر نے راجپوتوں اور افغانوں کو زیر کرتے ہوئے بالائی\linebreak}};%
\node[paragraphtext, anchor=north west, text width=845.0pt, align=justify] at (104.0,803.0){\setlength{\baselineskip}{47.300000000000004pt} \par
\veryLarge {وشالدیو کے دور میں وزیر اعظم، واستوپالا نے ایک\linebreak}};%
\node[paragraphtext, anchor=north west, text width=845.0pt, align=justify] at (104.0,759.5){\setlength{\baselineskip}{47.300000000000004pt} \par
\veryLarge {زیادہ امکان رکھتے ہیں، یعنی اچھے معیار کی مماثل\linebreak}};%
\node[paragraphtext, anchor=north west, text width=845.0pt, align=justify] at (104.0,716.0){\setlength{\baselineskip}{47.300000000000004pt} \par
\veryLarge {کردی ہیں۔ ضلع خوشاب کے بڑے موضع جات میں\linebreak}};%
\node[paragraphtext, anchor=north west, text width=845.0pt, align=justify] at (104.0,672.5){\setlength{\baselineskip}{47.300000000000004pt} \par
\veryLarge {تشکیل دیتا ہے ۔ جمالی بھی خود اپنا جد بھنڈ قبیلہ\linebreak}};%
\node[paragraphtext, anchor=north west, text width=845.0pt, align=justify] at (104.0,629.0){\setlength{\baselineskip}{47.300000000000004pt} \par
\veryLarge {دسمبر دو ہزار بارہ پر مشتمل ہے اور ایک ہزار تیس مربع\linebreak}};%
\node[paragraphtext, anchor=north west, text width=845.0pt, align=justify] at (104.0,585.5){\setlength{\baselineskip}{47.300000000000004pt} \par
\veryLarge {عملی مخالفت بھی نہیں کی کورولینکو سیاسی\linebreak}};%
\node[paragraphtext, anchor=north west, text width=845.0pt, align=justify] at (104.0,542.0){\setlength{\baselineskip}{47.300000000000004pt} \par
\veryLarge {کی۔ سلمبم میں مختلف قسم کے ہتھیار شامل ہوتے\linebreak}};%
\node[paragraphtext, anchor=north west, text width=845.0pt, align=justify] at (104.0,498.5){\setlength{\baselineskip}{47.300000000000004pt} \par
\veryLarge {زیر اہتمام رانجی ٹرافی پلیٹس سے ریاستی ٹیمیں شامل\linebreak}};%
\node[paragraphtext, anchor=north west, text width=845.0pt, align=justify] at (104.0,455.0){\setlength{\baselineskip}{47.300000000000004pt} \par
\veryLarge {حقائق کو بیان کرتا ہے جن پہ بات کرنا پسند نہیں کیا\linebreak}};%
\node[paragraphtext, anchor=north west, text width=845.0pt, align=justify] at (104.0,411.5){\setlength{\baselineskip}{47.300000000000004pt} \par
\veryLarge {نُورانی صُبح تھی ماہ ربیع الاوّل کی بارہویں تاریخ اور\linebreak}};%
\node[paragraphtext, anchor=north west, text width=845.0pt, align=justify] at (104.0,368.0){\setlength{\baselineskip}{47.300000000000004pt} \par
\veryLarge {بیالیس عیسوی کو موضع شوتخار تورکہو کے وزیر خان\linebreak}};%
\node[paragraphtext, anchor=north west, text width=845.0pt, align=justify] at (104.0,324.5){\setlength{\baselineskip}{47.300000000000004pt} \par
\veryLarge {کی ایک مستند تاریخی، ثقافتی، جغرافیائی اور معاشی\linebreak}};%
\node[paragraphtext, anchor=north west, text width=845.0pt, align=justify] at (104.0,281.0){\setlength{\baselineskip}{47.300000000000004pt} \par
\veryLarge {پیغام سنایا وہ سننے سے پہلے مومن تھیں، کیونکہ ان\linebreak}};%
\node[paragraphtext, anchor=north west, text width=845.0pt, align=justify] at (104.0,237.5){\setlength{\baselineskip}{47.300000000000004pt} \par
\veryLarge {گئے ۔ آخرمانکپورمیں بھی ایک خوریز جنگ ہوئی جس\linebreak}};%
\node[paragraphtext, anchor=north west, text width=845.0pt, align=justify] at (104.0,194.0){\setlength{\baselineskip}{47.300000000000004pt} \par
\veryLarge {حد تک باہم مربوط ہیں اب یہ کہ ان علوم کا ترجمہ سے\linebreak}};%
\end{tikzpicture}%
\end{center}%
\end{document}