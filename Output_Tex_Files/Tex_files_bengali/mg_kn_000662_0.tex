\documentclass[11pt]{article}%
\usepackage[T1]{fontenc}%
\usepackage[utf8]{inputenc}%
\usepackage{lmodern}%
\usepackage{textcomp}%
\usepackage{lastpage}%
\usepackage{tikz}%
\usepackage{fontspec}%
\usepackage{polyglossia}%
\usepackage[paperwidth=1275pt,paperheight=1763pt,margin=0pt]{geometry}%
%
\setmainlanguage{bengali}%
\setotherlanguage{english}%
\newfontfamily\bengalifont[Script=Bengali,Path=/home/vivek/Pager/fonts/bengali/Paragraph/]{Mina-Bold}%
\newfontfamily\headerfont[Script=Bengali,Path=/home/vivek/Pager/fonts/bengali/Header/]{NotoSansBengali_Condensed-Black}%
\newfontfamily\paragraphfont[Script=Bengali,Path=/home/vivek/Pager/fonts/bengali/Paragraph/]{Mina-Bold}%

    \makeatletter
    \newcommand{\zettaHuge}{\@setfontsize\zettaHuge{200}{220}}
    \newcommand{\exaHuge}{\@setfontsize\exaHuge{165}{180}}
    \newcommand{\petaHuge}{\@setfontsize\petaHuge{135}{150}}
    \newcommand{\teraHuge}{\@setfontsize\teraHuge{110}{120}}
    \newcommand{\gigaHuge}{\@setfontsize\gigaHuge{90}{100}}
    \newcommand{\megaHuge}{\@setfontsize\megaHuge{75}{85}}
    \newcommand{\superHuge}{\@setfontsize\superHuge{62}{70}}
    \newcommand{\verylarge}{\@setfontsize\verylarge{37}{42}}
    \newcommand{\veryLarge}{\@setfontsize\veryLarge{43}{49}}
    \newcommand{\veryHuge}{\@setfontsize\veryHuge{62}{70}}
    \newcommand{\alphaa}{\@setfontsize\alphaa{60}{66}}
    \newcommand{\betaa}{\@setfontsize\betaa{57}{63}}
    \newcommand{\gammaa}{\@setfontsize\gammaa{55}{61}}
    \newcommand{\deltaa}{\@setfontsize\deltaa{53}{59}}
    \newcommand{\epsilona}{\@setfontsize\epsilona{51}{57}}
    \newcommand{\zetaa}{\@setfontsize\zetaa{47}{53}}
    \newcommand{\etaa}{\@setfontsize\etaa{45}{51}}
    \newcommand{\iotaa}{\@setfontsize\iotaa{41}{47}}
    \newcommand{\kappaa}{\@setfontsize\kappaa{39}{45}}
    \newcommand{\lambdaa}{\@setfontsize\lambdaa{35}{41}}
    \newcommand{\mua}{\@setfontsize\mua{33}{39}}
    \newcommand{\nua}{\@setfontsize\nua{31}{37}}
    \newcommand{\xia}{\@setfontsize\xia{29}{35}}
    \newcommand{\pia}{\@setfontsize\pia{27}{33}}
    \newcommand{\rhoa}{\@setfontsize\rhoa{24}{30}}
    \newcommand{\sigmaa}{\@setfontsize\sigmaa{22}{28}}
    \newcommand{\taua}{\@setfontsize\taua{18}{24}}
    \newcommand{\upsilona}{\@setfontsize\upsilona{16}{22}}
    \newcommand{\phia}{\@setfontsize\phia{15}{20}}
    \newcommand{\chia}{\@setfontsize\chia{13}{18}}
    \newcommand{\psia}{\@setfontsize\psia{11}{16}}
    \newcommand{\omegaa}{\@setfontsize\omegaa{6}{7}}
    \newcommand{\oomegaa}{\@setfontsize\oomegaa{4}{5}}
    \newcommand{\ooomegaa}{\@setfontsize\ooomegaa{3}{4}}
    \newcommand{\oooomegaaa}{\@setfontsize\oooomegaaa{2}{3}}
    \makeatother
    %
%
\begin{document}%
\normalsize%
\begin{center}%
\begin{tikzpicture}[x=1pt, y=1pt]%
\node[anchor=south west, inner sep=0pt] at (0,0) {\includegraphics[width=1275pt,height=1763pt]{/home/vivek/Pager/images_val/mg_kn_000662_0.png}};%
\tikzset{headertext/.style={font=\headerfont, text=black}}%
\tikzset{paragraphtext/.style={font=\paragraphfont, text=black}}%
\node[paragraphtext, anchor=north west, text width=438.0pt] at (80.0,722.5) {\veryLarge{2. গাওয়ার শৈলী বিভিন্ন}};%
\node[paragraphtext, anchor=north west, text width=438.0pt] at (80.0,679.0) {\veryLarge{3. অভিনয়, এবং দুহাজার}};%
\node[paragraphtext, anchor=north west, text width=560.0pt] at (351.0,1099.5) {\veryLarge{4. নিম্নঢালকে এড়িয়ে চলে। দাঁড়া}};%
\node[paragraphtext, anchor=north west, text width=560.0pt] at (351.0,1056.0) {\veryLarge{5. সালে জাহাঙ্গীরনগর}};%
\node[paragraphtext, anchor=north west, text width=560.0pt] at (351.0,1012.5) {\veryLarge{6. করি না আর কি, আমি}};%
\node[paragraphtext, anchor=north west, text width=560.0pt] at (351.0,969.0) {\veryLarge{7. রাখা সাধারণভাবে কঠিন হয়।}};%
\node[paragraphtext, anchor=north west, text width=1157.0pt, align=left] at (63.0,360.5){\setlength{\baselineskip}{47.300000000000004pt} \par
\veryLarge {জানিয়েছেন। ঐদিন ওর মেজাজ ভালো ছিল না। তাঁর পরবর্তী\linebreak}};%
\node[paragraphtext, anchor=north west, text width=1157.0pt, align=left] at (63.0,317.0){\setlength{\baselineskip}{47.300000000000004pt} \par
\veryLarge {পেজে পৌঁছানোর জন্য কোম্পানির নামের ওপর ক্লিক করুন।\linebreak}};%
\node[paragraphtext, anchor=north west, text width=1157.0pt, align=left] at (63.0,273.5){\setlength{\baselineskip}{47.300000000000004pt} \par
\veryLarge {অপিতরুনহাসাধ্যতেসুখং, অর্থাৎ রোগ এবং তার প্রকোপের ব্যাপারে\linebreak}};%
\node[paragraphtext, anchor=north west, text width=1157.0pt, align=left] at (63.0,230.0){\setlength{\baselineskip}{47.300000000000004pt} \par
\veryLarge {হ্যাঁ ঠিকই এর আগে যেটা খেয়েছে তার জন্যই হয়েছে আপনার\linebreak}};%
\node[paragraphtext, anchor=north west, text width=1157.0pt, align=left] at (63.0,186.5){\setlength{\baselineskip}{47.300000000000004pt} \par
\veryLarge {হ্যাঁ তো আপনি যে কাঠি রোলটা যে দিন বিক্রি করেন যে তেল\linebreak}};%
\node[paragraphtext, anchor=north west, text width=1157.0pt, align=left] at (63.0,143.0){\setlength{\baselineskip}{47.300000000000004pt} \par
\veryLarge {প্রকাশ অনুমোদন করে যা অন্যথায় সিন্থেসাইজ করা কঠিন।\linebreak}};%
\node[headertext, anchor=north west, text width=421.0pt, align=left] at (736.0,643.5){\setlength{\baselineskip}{42.900000000000006pt} \par
\kappaa {বলেছেন, উত্তর পূর্বের\linebreak}};%
\node[paragraphtext, anchor=north west, text width=180.0pt, align=left] at (989.0,891.5){\setlength{\baselineskip}{47.300000000000004pt} \par
\veryLarge {স্বাধীনতা\linebreak}};%
\node[paragraphtext, anchor=north west, text width=180.0pt, align=left] at (989.0,848.0){\setlength{\baselineskip}{47.300000000000004pt} \par
\veryLarge {সুলতান\linebreak}};%
\node[headertext, anchor=north west, text width=232.0pt, align=left] at (76.0,883.5){\setlength{\baselineskip}{42.900000000000006pt} \par
\kappaa {এবং\linebreak}};%
\node[paragraphtext, anchor=north west, text width=203.0pt, align=left] at (85.0,1144.5){\setlength{\baselineskip}{47.300000000000004pt} \par
\veryLarge {রায়ের\linebreak}};%
\node[headertext, anchor=north west, text width=114.0pt, align=left] at (36.0,68.5){\setlength{\baselineskip}{47.300000000000004pt} \par
\veryLarge {ঢেকে\linebreak}};%
\node[paragraphtext, anchor=north west, text width=257.0pt, align=justify] at (347.0,1130.5){\setlength{\baselineskip}{31.900000000000002pt} \par
\xia {মিথ্যা বিবৃতি দেওয়ার\linebreak}};%
\node[paragraphtext, anchor=north west, text width=351.0pt, align=left] at (111.0,514.5){\setlength{\baselineskip}{47.300000000000004pt} \par
\veryLarge {ভুজঙ্গায়ন দশাবতার\linebreak}};%
\node[paragraphtext, anchor=north west, text width=351.0pt, align=left] at (111.0,471.0){\setlength{\baselineskip}{47.300000000000004pt} \par
\veryLarge {ঊননব্বই সালের,\linebreak}};%
\node[headertext, anchor=north west, text width=379.0pt, align=left] at (353.0,1227.5){\setlength{\baselineskip}{42.900000000000006pt} \par
\kappaa {খেলোয়াড়\linebreak}};%
\node[paragraphtext, anchor=north west, text width=876.0pt, align=justify] at (349.0,1187.5){\setlength{\baselineskip}{31.900000000000002pt} \par
\xia {বিরূদ্ধে লড়াইয়ের সমর্থনে লাইফলাইন উড়ান পরিষেবার মাধ্যমে, পাঁচশো\linebreak}};%
\node[paragraphtext, anchor=north west, text width=816.0pt] at (349.0,1357.5) {\veryLarge{8. দেওয়ার পরে ঠিক আছে এখন আসুন দেখা}};%
\node[paragraphtext, anchor=north west, text width=816.0pt] at (349.0,1314.0) {\veryLarge{9. দায়িত্ব পালন করবে, কারণ তারা একটি ব্যস্ত}};%
\node[headertext, anchor=north west, text width=979.0pt, align=left, text=black] at (154.0,1686.5){\setlength{\baselineskip}{68.2pt} \par
\superHuge {মধ্যে আতঙ্ক রয়ে থাকতো যে ফ্যানটি কখন\linebreak}};%
\node[headertext, anchor=north west, text width=908.0pt, align=left, text=black] at (195.0,1564.5){\setlength{\baselineskip}{47.300000000000004pt} \par
\veryLarge {যা পরঈদর অক্ষরগুলি দিয়ে শুরু হয় এবং বেশিরভাগ\linebreak}};%
\node[paragraphtext, anchor=north west, text width=851.0pt, align=justify] at (68.0,838.5){\setlength{\baselineskip}{31.900000000000002pt} \par
\xia {কোম্পানির একটি ফাঁড়ি আক্রমণ করা হয়। তারা দুই দেশের\linebreak}};%
\node[paragraphtext, anchor=north west, text width=851.0pt, align=justify] at (68.0,795.0){\setlength{\baselineskip}{31.900000000000002pt} \par
\xia {দেখা যায় ভারতের বৈগা এবং মাল পাহাড়িয়া ইকুয়েডরের জিভারো\linebreak}};%
\node[paragraphtext, anchor=north west, text width=609.0pt] at (350.0,891.5) {\veryLarge{10. কোষের অনিয়ন্ত্রিত বিভাজন হল}};%
\node[paragraphtext, anchor=north west, text width=470.0pt, align=justify] at (743.0,593.5){\setlength{\baselineskip}{31.900000000000002pt} \par
\xia {ভাবতেন। কেন্দ্রীয় শিল্প, বাণিজ্য এবং\linebreak}};%
\node[paragraphtext, anchor=north west, text width=470.0pt, align=justify] at (743.0,550.0){\setlength{\baselineskip}{31.900000000000002pt} \par
\xia {ফর ফ্যাশন -এর পরে এখন, খাদি ফর\linebreak}};%
\node[paragraphtext, anchor=north west, text width=470.0pt, align=justify] at (743.0,506.5){\setlength{\baselineskip}{31.900000000000002pt} \par
\xia {ইন্ট্রা ব্রিকস ঐক্যবদ্ধতা বৃদ্ধি এবং এর\linebreak}};%
\node[paragraphtext, anchor=north west, text width=470.0pt, align=justify] at (743.0,463.0){\setlength{\baselineskip}{31.900000000000002pt} \par
\xia {ভৌগলিক এলাকায় জন্মগ্রহণকারী,\linebreak}};%
\node[paragraphtext, anchor=north west, text width=770.0pt, align=justify] at (348.0,1466.5){\setlength{\baselineskip}{31.900000000000002pt} \par
\xia {বর্তমানে সাম্প্রতিক বছরগুলিতে বেশ কিছু শিল্প যেমন মৃৎশিল্প\linebreak}};%
\node[paragraphtext, anchor=north west, text width=770.0pt, align=justify] at (348.0,1423.0){\setlength{\baselineskip}{31.900000000000002pt} \par
\xia {স্মৃতিসৌধ নির্মাণের উদ্যোগ নেওয়া হয়েছে, যেখানে তাঁর শেষকৃত্য\linebreak}};%
\end{tikzpicture}%
\end{center}%
\end{document}