\documentclass[11pt]{article}%
\usepackage[T1]{fontenc}%
\usepackage[utf8]{inputenc}%
\usepackage{lmodern}%
\usepackage{textcomp}%
\usepackage{lastpage}%
\usepackage{tikz}%
\usepackage{fontspec}%
\usepackage{polyglossia}%
\usepackage[paperwidth=1042pt,paperheight=1667pt,margin=0pt]{geometry}%
%
\setmainlanguage{bengali}%
\setotherlanguage{english}%
\newfontfamily\bengalifont[Script=Bengali,Path=/home/vivek/Pager/fonts/bengali/Paragraph/]{NotoSerifBengali_SemiCondensed-Light}%
\newfontfamily\headerfont[Script=Bengali,Path=/home/vivek/Pager/fonts/bengali/Header/]{NotoSansBengali_ExtraCondensed-ExtraBold}%
\newfontfamily\paragraphfont[Script=Bengali,Path=/home/vivek/Pager/fonts/bengali/Paragraph/]{NotoSerifBengali_SemiCondensed-Light}%

    \makeatletter
    \newcommand{\zettaHuge}{\@setfontsize\zettaHuge{200}{220}}
    \newcommand{\exaHuge}{\@setfontsize\exaHuge{165}{180}}
    \newcommand{\petaHuge}{\@setfontsize\petaHuge{135}{150}}
    \newcommand{\teraHuge}{\@setfontsize\teraHuge{110}{120}}
    \newcommand{\gigaHuge}{\@setfontsize\gigaHuge{90}{100}}
    \newcommand{\megaHuge}{\@setfontsize\megaHuge{75}{85}}
    \newcommand{\superHuge}{\@setfontsize\superHuge{62}{70}}
    \newcommand{\verylarge}{\@setfontsize\verylarge{37}{42}}
    \newcommand{\veryLarge}{\@setfontsize\veryLarge{43}{49}}
    \newcommand{\veryHuge}{\@setfontsize\veryHuge{62}{70}}
    \newcommand{\alphaa}{\@setfontsize\alphaa{60}{66}}
    \newcommand{\betaa}{\@setfontsize\betaa{57}{63}}
    \newcommand{\gammaa}{\@setfontsize\gammaa{55}{61}}
    \newcommand{\deltaa}{\@setfontsize\deltaa{53}{59}}
    \newcommand{\epsilona}{\@setfontsize\epsilona{51}{57}}
    \newcommand{\zetaa}{\@setfontsize\zetaa{47}{53}}
    \newcommand{\etaa}{\@setfontsize\etaa{45}{51}}
    \newcommand{\iotaa}{\@setfontsize\iotaa{41}{47}}
    \newcommand{\kappaa}{\@setfontsize\kappaa{39}{45}}
    \newcommand{\lambdaa}{\@setfontsize\lambdaa{35}{41}}
    \newcommand{\mua}{\@setfontsize\mua{33}{39}}
    \newcommand{\nua}{\@setfontsize\nua{31}{37}}
    \newcommand{\xia}{\@setfontsize\xia{29}{35}}
    \newcommand{\pia}{\@setfontsize\pia{27}{33}}
    \newcommand{\rhoa}{\@setfontsize\rhoa{24}{30}}
    \newcommand{\sigmaa}{\@setfontsize\sigmaa{22}{28}}
    \newcommand{\taua}{\@setfontsize\taua{18}{24}}
    \newcommand{\upsilona}{\@setfontsize\upsilona{16}{22}}
    \newcommand{\phia}{\@setfontsize\phia{15}{20}}
    \newcommand{\chia}{\@setfontsize\chia{13}{18}}
    \newcommand{\psia}{\@setfontsize\psia{11}{16}}
    \newcommand{\omegaa}{\@setfontsize\omegaa{6}{7}}
    \newcommand{\oomegaa}{\@setfontsize\oomegaa{4}{5}}
    \newcommand{\ooomegaa}{\@setfontsize\ooomegaa{3}{4}}
    \newcommand{\oooomegaaa}{\@setfontsize\oooomegaaa{2}{3}}
    \makeatother
    %
%
\begin{document}%
\normalsize%
\begin{center}%
\begin{tikzpicture}[x=1pt, y=1pt]%
\node[anchor=south west, inner sep=0pt] at (0,0) {\includegraphics[width=1042pt,height=1667pt]{/home/vivek/Pager/images_val/mg_pa_000851_0.png}};%
\tikzset{headertext/.style={font=\headerfont, text=black}}%
\tikzset{paragraphtext/.style={font=\paragraphfont, text=black}}%
\node[headertext, anchor=north west, text width=179.0pt, align=left] at (138.0,1577.5){\setlength{\baselineskip}{40.7pt} \par
\verylarge {বিশ্বেশ্বর,\linebreak}};%
\node[paragraphtext, anchor=north west, text width=845.0pt, align=justify] at (114.0,1540.5){\setlength{\baselineskip}{47.300000000000004pt} \par
\veryLarge {হাজার ষোলো সালে মুম্বাইয়ের কাছে অবস্থিত আরব সাগরের\linebreak}};%
\node[paragraphtext, anchor=north west, text width=845.0pt, align=justify] at (114.0,1497.0){\setlength{\baselineskip}{47.300000000000004pt} \par
\veryLarge {অবস্থিত। মণিপুরের নদী অববাহিকায় আটটি প্রধান নদী\linebreak}};%
\node[paragraphtext, anchor=north west, text width=844.0pt, align=justify] at (114.0,1440.5){\setlength{\baselineskip}{47.300000000000004pt} \par
\veryLarge {রাওয়া কলা-গুড়-মিঠাই। এমনকি একই অ্যামনিওটিক কোষ\linebreak}};%
\node[paragraphtext, anchor=north west, text width=844.0pt, align=justify] at (114.0,1397.0){\setlength{\baselineskip}{47.300000000000004pt} \par
\veryLarge {টাইটেনিয়াম ও পরমাণুকেন্দ্রের গলন বা ফিউশন প্রক্রিয়ার\linebreak}};%
\node[paragraphtext, anchor=north west, text width=844.0pt, align=justify] at (114.0,1353.5){\setlength{\baselineskip}{47.300000000000004pt} \par
\veryLarge {খানের মতো বেশ কয়েকটি গুরুকুল উন্নীত হয়।\linebreak}};%
\node[paragraphtext, anchor=north west, text width=844.0pt, align=justify] at (114.0,1310.0){\setlength{\baselineskip}{47.300000000000004pt} \par
\veryLarge {সর্বনিম্ন জ্ঞান গ্রহণ করা হয়। দত্তক পুত্র রামকৃঞ্চ রায়ের\linebreak}};%
\node[paragraphtext, anchor=north west, text width=844.0pt, align=justify] at (114.0,1266.5){\setlength{\baselineskip}{47.300000000000004pt} \par
\veryLarge {সদস্যদের ওই দিন গ্রেফতার করা হয়। অবৈধ ওষুধগুলি\linebreak}};%
\node[paragraphtext, anchor=north west, text width=844.0pt, align=justify] at (114.0,1223.0){\setlength{\baselineskip}{47.300000000000004pt} \par
\veryLarge {জন্য, দেশভক্তির প্রেরণা জাগানো এই আগষ্ট মাসের জন্য\linebreak}};%
\node[paragraphtext, anchor=north west, text width=844.0pt, align=justify] at (114.0,1179.5){\setlength{\baselineskip}{47.300000000000004pt} \par
\veryLarge {উদ্দেশ্যে তাদের অনুগামী হন। এখন আমরা ধরে নিই যে\linebreak}};%
\node[paragraphtext, anchor=north west, text width=844.0pt, align=justify] at (114.0,1136.0){\setlength{\baselineskip}{47.300000000000004pt} \par
\veryLarge {যুদ্ধজাহাজ অংশ নিচ্ছে। সেইলের অধ্যক্ষা সোমা মণ্ডল\linebreak}};%
\node[paragraphtext, anchor=north west, text width=844.0pt, align=justify] at (114.0,1092.5){\setlength{\baselineskip}{47.300000000000004pt} \par
\veryLarge {অনুভূতি বোধগম্যভাবে আশাব্যঞ্জক ছিল৷ ডাক্তার ঘোষও\linebreak}};%
\node[paragraphtext, anchor=north west, text width=844.0pt, align=justify] at (114.0,1049.0){\setlength{\baselineskip}{47.300000000000004pt} \par
\veryLarge {ইউসেবিয়া শব্দটি ব্যবহার করেছেন। তাঁর পালিত ভাই সমস্ত\linebreak}};%
\node[paragraphtext, anchor=north west, text width=846.0pt, align=justify] at (113.0,983.5){\setlength{\baselineskip}{47.300000000000004pt} \par
\veryLarge {সেখানে আপনার সমস্ত পুঁজি ব্যয় করতে এবং সেই প্রচেষ্টায়\linebreak}};%
\node[paragraphtext, anchor=north west, text width=846.0pt, align=justify] at (113.0,940.0){\setlength{\baselineskip}{47.300000000000004pt} \par
\veryLarge {আবদালি-রোহিলা সেনাবাহিনীর সঙ্গে যুদ্ধে ব্যস্ত ছিলেন।\linebreak}};%
\node[paragraphtext, anchor=north west, text width=846.0pt, align=justify] at (113.0,896.5){\setlength{\baselineskip}{47.300000000000004pt} \par
\veryLarge {অবস্থিত একটি জলাধার। মীর কাসিম নিজে, আওয়াধের\linebreak}};%
\node[paragraphtext, anchor=north west, text width=846.0pt, align=justify] at (113.0,853.0){\setlength{\baselineskip}{47.300000000000004pt} \par
\veryLarge {ডেইলি এবং সুপ্রভাথম। সুতরাং এর ক্ষেত্রে বিনিয়োগ\linebreak}};%
\node[paragraphtext, anchor=north west, text width=846.0pt, align=justify] at (113.0,809.5){\setlength{\baselineskip}{47.300000000000004pt} \par
\veryLarge {তিন চারবার পায়খানা হয়েছে মানে দিনে তিন চারবার\linebreak}};%
\node[paragraphtext, anchor=north west, text width=848.0pt, align=justify] at (113.0,735.5){\setlength{\baselineskip}{47.300000000000004pt} \par
\veryLarge {কলা গাছে থেকে পাড়া হয়েছে। প্রাপ্ত সবচেয়ে প্রাচীন বাঘনখে\linebreak}};%
\node[paragraphtext, anchor=north west, text width=848.0pt, align=justify] at (113.0,692.0){\setlength{\baselineskip}{47.300000000000004pt} \par
\veryLarge {মাখামাখি হয়ে গেছে। ওর বাচ্চারা জোরে কথা বললে ধমক\linebreak}};%
\node[paragraphtext, anchor=north west, text width=848.0pt, align=justify] at (113.0,648.5){\setlength{\baselineskip}{47.300000000000004pt} \par
\veryLarge {থ্রোয়ের মতো অনেক সম্ভাব্য মৃত্যু বা পঙ্গু করে দেওয়ার মত\linebreak}};%
\node[paragraphtext, anchor=north west, text width=848.0pt, align=justify] at (113.0,605.0){\setlength{\baselineskip}{47.300000000000004pt} \par
\veryLarge {মূলত রপ্তানির উদ্দেশ্যে লৌহ আকরিক উত্তোলিত হয়। এটি\linebreak}};%
\node[paragraphtext, anchor=north west, text width=848.0pt, align=justify] at (113.0,561.5){\setlength{\baselineskip}{47.300000000000004pt} \par
\veryLarge {যে কোষের নিউক্লিয়াগুলি সাধারণ স্তন নালী এপিথেলিয়াল\linebreak}};%
\node[paragraphtext, anchor=north west, text width=848.0pt, align=justify] at (113.0,518.0){\setlength{\baselineskip}{47.300000000000004pt} \par
\veryLarge {ভারতের স্যাটেলাইট চিত্রগুলিতে দৃশ্যমান মানবসৃষ্ট বৃহত্তম\linebreak}};%
\node[paragraphtext, anchor=north west, text width=848.0pt, align=justify] at (113.0,474.5){\setlength{\baselineskip}{47.300000000000004pt} \par
\veryLarge {লোকেরা ভারতের জনসংখ্যার ঘনত্ব বণ্টন ও বৃদ্ধি একইসঙ্গে\linebreak}};%
\node[paragraphtext, anchor=north west, text width=848.0pt, align=justify] at (113.0,431.0){\setlength{\baselineskip}{47.300000000000004pt} \par
\veryLarge {হয়েছিল। এতে অভিনয় করেছেন সঞ্জীব কুমার, সাঈদ\linebreak}};%
\node[paragraphtext, anchor=north west, text width=848.0pt, align=justify] at (113.0,387.5){\setlength{\baselineskip}{47.300000000000004pt} \par
\veryLarge {কুড়মালি ভাষায় রচিত আরো একটি গ্রন্থ। যক্ষগানের রাগগুলি\linebreak}};%
\node[paragraphtext, anchor=north west, text width=849.0pt, align=justify] at (114.0,334.5){\setlength{\baselineskip}{47.300000000000004pt} \par
\veryLarge {জনসংখ্যার মধ্যে, তেইশ দশমিক চব্বিশ শতাংশ মানুষ\linebreak}};%
\node[paragraphtext, anchor=north west, text width=849.0pt, align=justify] at (114.0,291.0){\setlength{\baselineskip}{47.300000000000004pt} \par
\veryLarge {এবং ধর্মীয় অনুভূতিতে আঘাত করার জন্য তাঁর বিরুদ্ধে\linebreak}};%
\node[paragraphtext, anchor=north west, text width=849.0pt, align=justify] at (114.0,247.5){\setlength{\baselineskip}{47.300000000000004pt} \par
\veryLarge {দিন হিসাবে বিবেচনা করা হয়, যা মানুষের মধ্যে ভ্রাতৃত্ববোধের\linebreak}};%
\node[paragraphtext, anchor=north west, text width=849.0pt, align=justify] at (114.0,204.0){\setlength{\baselineskip}{47.300000000000004pt} \par
\veryLarge {চোখ করকর করা, অত্যধিক চোখ ঘোরানো এবং চোখ থেকে\linebreak}};%
\end{tikzpicture}%
\end{center}%
\end{document}