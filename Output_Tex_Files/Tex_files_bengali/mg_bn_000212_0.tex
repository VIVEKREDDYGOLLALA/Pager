\documentclass[11pt]{article}%
\usepackage[T1]{fontenc}%
\usepackage[utf8]{inputenc}%
\usepackage{lmodern}%
\usepackage{textcomp}%
\usepackage{lastpage}%
\usepackage{tikz}%
\usepackage{fontspec}%
\usepackage{polyglossia}%
\usepackage[paperwidth=1176pt,paperheight=1320pt,margin=0pt]{geometry}%
%
\setmainlanguage{bengali}%
\setotherlanguage{english}%
\newfontfamily\bengalifont[Script=Bengali,Path=/home/vivek/Pager/fonts/bengali/Paragraph/]{Atma-SemiBold}%
\newfontfamily\headerfont[Script=Bengali,Path=/home/vivek/Pager/fonts/bengali/Header/]{NotoSansBengali-Bold}%
\newfontfamily\paragraphfont[Script=Bengali,Path=/home/vivek/Pager/fonts/bengali/Paragraph/]{Atma-SemiBold}%

    \makeatletter
    \newcommand{\zettaHuge}{\@setfontsize\zettaHuge{200}{220}}
    \newcommand{\exaHuge}{\@setfontsize\exaHuge{165}{180}}
    \newcommand{\petaHuge}{\@setfontsize\petaHuge{135}{150}}
    \newcommand{\teraHuge}{\@setfontsize\teraHuge{110}{120}}
    \newcommand{\gigaHuge}{\@setfontsize\gigaHuge{90}{100}}
    \newcommand{\megaHuge}{\@setfontsize\megaHuge{75}{85}}
    \newcommand{\superHuge}{\@setfontsize\superHuge{62}{70}}
    \newcommand{\verylarge}{\@setfontsize\verylarge{37}{42}}
    \newcommand{\veryLarge}{\@setfontsize\veryLarge{43}{49}}
    \newcommand{\veryHuge}{\@setfontsize\veryHuge{62}{70}}
    \newcommand{\alphaa}{\@setfontsize\alphaa{60}{66}}
    \newcommand{\betaa}{\@setfontsize\betaa{57}{63}}
    \newcommand{\gammaa}{\@setfontsize\gammaa{55}{61}}
    \newcommand{\deltaa}{\@setfontsize\deltaa{53}{59}}
    \newcommand{\epsilona}{\@setfontsize\epsilona{51}{57}}
    \newcommand{\zetaa}{\@setfontsize\zetaa{47}{53}}
    \newcommand{\etaa}{\@setfontsize\etaa{45}{51}}
    \newcommand{\iotaa}{\@setfontsize\iotaa{41}{47}}
    \newcommand{\kappaa}{\@setfontsize\kappaa{39}{45}}
    \newcommand{\lambdaa}{\@setfontsize\lambdaa{35}{41}}
    \newcommand{\mua}{\@setfontsize\mua{33}{39}}
    \newcommand{\nua}{\@setfontsize\nua{31}{37}}
    \newcommand{\xia}{\@setfontsize\xia{29}{35}}
    \newcommand{\pia}{\@setfontsize\pia{27}{33}}
    \newcommand{\rhoa}{\@setfontsize\rhoa{24}{30}}
    \newcommand{\sigmaa}{\@setfontsize\sigmaa{22}{28}}
    \newcommand{\taua}{\@setfontsize\taua{18}{24}}
    \newcommand{\upsilona}{\@setfontsize\upsilona{16}{22}}
    \newcommand{\phia}{\@setfontsize\phia{15}{20}}
    \newcommand{\chia}{\@setfontsize\chia{13}{18}}
    \newcommand{\psia}{\@setfontsize\psia{11}{16}}
    \newcommand{\omegaa}{\@setfontsize\omegaa{6}{7}}
    \newcommand{\oomegaa}{\@setfontsize\oomegaa{4}{5}}
    \newcommand{\ooomegaa}{\@setfontsize\ooomegaa{3}{4}}
    \newcommand{\oooomegaaa}{\@setfontsize\oooomegaaa{2}{3}}
    \makeatother
    %
%
\begin{document}%
\normalsize%
\begin{center}%
\begin{tikzpicture}[x=1pt, y=1pt]%
\node[anchor=south west, inner sep=0pt] at (0,0) {\includegraphics[width=1176pt,height=1320pt]{/home/vivek/Pager/images_val/mg_bn_000212_0.png}};%
\tikzset{headertext/.style={font=\headerfont, text=black}}%
\tikzset{paragraphtext/.style={font=\paragraphfont, text=black}}%
\node[paragraphtext, anchor=north west, text width=309.0pt, align=justify] at (67.0,615.5){\setlength{\baselineskip}{19.8pt} \par
\taua {তাঁদের হিসাব মতো বার্ষিক আয় দুলক্ষ মার্কিন\linebreak}};%
\node[paragraphtext, anchor=north west, text width=309.0pt, align=justify] at (67.0,597.5){\setlength{\baselineskip}{19.8pt} \par
\taua {পেলে এখন আমার ভবিষ্যৎটা সব থেকে খারাপ\linebreak}};%
\node[paragraphtext, anchor=north west, text width=309.0pt, align=justify] at (67.0,579.5){\setlength{\baselineskip}{19.8pt} \par
\taua {তাঁকে পূর্ণ সামরিক সম্মান দেয়। এই অনুষ্ঠানে\linebreak}};%
\node[paragraphtext, anchor=north west, text width=309.0pt, align=justify] at (67.0,561.5){\setlength{\baselineskip}{19.8pt} \par
\taua {থেকে প্রায় ত্রিশ কিমি দূরে অবস্থিত। মাঙ্কি\linebreak}};%
\node[paragraphtext, anchor=north west, text width=309.0pt, align=justify] at (67.0,543.5){\setlength{\baselineskip}{19.8pt} \par
\taua {ফুলে মারাঠি লোক পরিবেশনার মাধ্যমে তার\linebreak}};%
\node[paragraphtext, anchor=north west, text width=309.0pt, align=justify] at (67.0,525.5){\setlength{\baselineskip}{19.8pt} \par
\taua {মনে করি মাতৃযান আপনি বোধহয় শৌচালয়ের\linebreak}};%
\node[paragraphtext, anchor=north west, text width=309.0pt, align=justify] at (67.0,507.5){\setlength{\baselineskip}{19.8pt} \par
\taua {অধাতব ধর্ম হ্রাস পায়। জানো তো ভায়া, আজ\linebreak}};%
\node[paragraphtext, anchor=north west, text width=309.0pt, align=justify] at (67.0,489.5){\setlength{\baselineskip}{19.8pt} \par
\taua {তিড়াশি লক্ষ, এইচসিকিউ ট্যাবলেট সরবরাহ\linebreak}};%
\node[paragraphtext, anchor=north west, text width=312.0pt, align=justify] at (67.0,1048.5){\setlength{\baselineskip}{19.8pt} \par
\taua {করতে সাহায্য করবে। সুখিয়া কাকা বিটৌনা নামে\linebreak}};%
\node[paragraphtext, anchor=north west, text width=312.0pt, align=justify] at (67.0,1030.5){\setlength{\baselineskip}{19.8pt} \par
\taua {হাজার হাজার ফালুন গং সদস্যদের আটক করেছে\linebreak}};%
\node[paragraphtext, anchor=north west, text width=312.0pt, align=justify] at (67.0,1012.5){\setlength{\baselineskip}{19.8pt} \par
\taua {মিথিলা পালকার একজন ভারতীয় অভিনেত্রী, যিনি\linebreak}};%
\node[paragraphtext, anchor=north west, text width=312.0pt, align=justify] at (67.0,994.5){\setlength{\baselineskip}{19.8pt} \par
\taua {হয় তবে সম্ভাব্য মোট বুলিয়ান গঠনের সংখ্যা\linebreak}};%
\node[paragraphtext, anchor=north west, text width=312.0pt, align=justify] at (67.0,976.5){\setlength{\baselineskip}{19.8pt} \par
\taua {এবং দুটি আইফা পুরস্কার সহ বিভিন্ন বিভাগে\linebreak}};%
\node[paragraphtext, anchor=north west, text width=312.0pt, align=justify] at (67.0,958.5){\setlength{\baselineskip}{19.8pt} \par
\taua {দলের সঙ্গে অনেক সংস্থা ও স্বেচ্ছাসেবী সংগঠণও\linebreak}};%
\node[paragraphtext, anchor=north west, text width=312.0pt, align=justify] at (67.0,940.5){\setlength{\baselineskip}{19.8pt} \par
\taua {উড়ে যাওয়ার সময় লালঠোঁট বিষুবীয়া গতিতে\linebreak}};%
\node[paragraphtext, anchor=north west, text width=312.0pt, align=justify] at (67.0,922.5){\setlength{\baselineskip}{19.8pt} \par
\taua {কর্তন শুরু করে। তামিলনাড়ু এক লক্ষ ত্রিশ হাজার\linebreak}};%
\node[paragraphtext, anchor=north west, text width=312.0pt, align=justify] at (67.0,904.5){\setlength{\baselineskip}{19.8pt} \par
\taua {বিদ্রোহের পর ব্রিটিশরা দুর্গটির বেশিরভাগ\linebreak}};%
\node[paragraphtext, anchor=north west, text width=312.0pt, align=justify] at (67.0,886.5){\setlength{\baselineskip}{19.8pt} \par
\taua {ও অন্যান্য মাদক দ্রব্য বিপুল পরিমাণে বাজেয়াপ্ত\linebreak}};%
\node[paragraphtext, anchor=north west, text width=312.0pt, align=justify] at (67.0,868.5){\setlength{\baselineskip}{19.8pt} \par
\taua {ছয়-এ, খানকে পাঁচ বছরের জেল দেওয়া হয় এবং\linebreak}};%
\node[paragraphtext, anchor=north west, text width=312.0pt, align=justify] at (67.0,850.5){\setlength{\baselineskip}{19.8pt} \par
\taua {দেয় কারণ কেউ কেউ মনে করেন মসজিদটি নঁতে\linebreak}};%
\node[paragraphtext, anchor=north west, text width=312.0pt, align=justify] at (67.0,832.5){\setlength{\baselineskip}{19.8pt} \par
\taua {ধাপকে বাইআতে আকাবায়ে ঊলা বলে। সাজেশন\linebreak}};%
\node[paragraphtext, anchor=north west, text width=312.0pt, align=justify] at (67.0,814.5){\setlength{\baselineskip}{19.8pt} \par
\taua {ভট্ট পরিচালিত তাঁর প্রথম হিন্দি-তেলুগু দ্বিভাষিক\linebreak}};%
\node[paragraphtext, anchor=north west, text width=312.0pt, align=justify] at (67.0,796.5){\setlength{\baselineskip}{19.8pt} \par
\taua {এবং ফসল সংরক্ষণসহ দুর্যোগ ব্যবস্থাপনার\linebreak}};%
\node[paragraphtext, anchor=north west, text width=312.0pt, align=justify] at (67.0,778.5){\setlength{\baselineskip}{19.8pt} \par
\taua {ভবিষ্যৎের অনেক সমস্যা ও জীবন জিজ্ঞাসার কথা\linebreak}};%
\node[paragraphtext, anchor=north west, text width=312.0pt, align=justify] at (67.0,760.5){\setlength{\baselineskip}{19.8pt} \par
\taua {সব্রতা, রুতবা, বাচঁম বদত অর্থাৎ, আমরা যেন\linebreak}};%
\node[paragraphtext, anchor=north west, text width=312.0pt, align=justify] at (67.0,742.5){\setlength{\baselineskip}{19.8pt} \par
\taua {চিন্তানায়ক হলেন ক্লড হেনরী দ্য সাঁ সিঁমো।\linebreak}};%
\node[paragraphtext, anchor=north west, text width=312.0pt, align=justify] at (67.0,724.5){\setlength{\baselineskip}{19.8pt} \par
\taua {সুঙ্গ যুগের স্থান থেকে খনন করে তুলে আনা হয়।\linebreak}};%
\node[paragraphtext, anchor=north west, text width=312.0pt, align=justify] at (67.0,706.5){\setlength{\baselineskip}{19.8pt} \par
\taua {নারদের সঙ্গীত মকরন্দ গ্রন্থ হল বর্তমান হিন্দুস্তানি\linebreak}};%
\node[paragraphtext, anchor=north west, text width=312.0pt, align=justify] at (67.0,688.5){\setlength{\baselineskip}{19.8pt} \par
\taua {এবং অর্জিত জ্ঞানকে কাজে লাগায়। যং\linebreak}};%
\node[paragraphtext, anchor=north west, text width=312.0pt, align=justify] at (67.0,670.5){\setlength{\baselineskip}{19.8pt} \par
\taua {পরেছিলেন বলে জানা গেছে। তাঁর স্থলাভিষিক্ত হন\linebreak}};%
\node[paragraphtext, anchor=north west, text width=312.0pt, align=justify] at (67.0,652.5){\setlength{\baselineskip}{19.8pt} \par
\taua {করে। তামিলনাড়ুর অনেক মৌসুমী নদীর মতো,\linebreak}};%
\node[headertext, anchor=north west, text width=159.0pt, align=left] at (60.0,1269.5){\setlength{\baselineskip}{19.8pt} \par
\taua {প্রাক্তন ছাত্র এবং\linebreak}};%
\node[headertext, anchor=north west, text width=159.0pt, align=left] at (60.0,1251.5){\setlength{\baselineskip}{19.8pt} \par
\taua {শুষ্ক এবং কম বৃষ্টিপাত\linebreak}};%
\node[headertext, anchor=north west, text width=159.0pt, align=left] at (60.0,1233.5){\setlength{\baselineskip}{19.8pt} \par
\taua {‘পাথম পসালি’, ‘সোরগম’,\linebreak}};%
\node[headertext, anchor=north west, text width=159.0pt, align=left] at (60.0,1215.5){\setlength{\baselineskip}{19.8pt} \par
\taua {অল্পবয়সী কমলাদেবীকে\linebreak}};%
\node[headertext, anchor=north west, text width=159.0pt, align=left] at (60.0,1197.5){\setlength{\baselineskip}{19.8pt} \par
\taua {রাসায়নিক এবং\linebreak}};%
\node[headertext, anchor=north west, text width=159.0pt, align=left] at (60.0,1179.5){\setlength{\baselineskip}{19.8pt} \par
\taua {পেট গ্রহনের পর এই\linebreak}};%
\node[paragraphtext, anchor=north west, text width=314.0pt, align=justify] at (403.0,594.5){\setlength{\baselineskip}{19.8pt} \par
\taua {সুতরাং এর অর্থ এখানে ইথনের মতো কোনও সি\linebreak}};%
\node[paragraphtext, anchor=north west, text width=314.0pt, align=justify] at (403.0,576.5){\setlength{\baselineskip}{19.8pt} \par
\taua {দিতেই পারছি না গুছিয়ে যে ব্যাপারটা লিখবো\linebreak}};%
\node[paragraphtext, anchor=north west, text width=314.0pt, align=justify] at (403.0,558.5){\setlength{\baselineskip}{19.8pt} \par
\taua {যুদ্ধের পর অশোক বৌদ্ধ ধর্মে ধর্মান্তরিত হয়ে\linebreak}};%
\node[paragraphtext, anchor=north west, text width=314.0pt, align=justify] at (403.0,540.5){\setlength{\baselineskip}{19.8pt} \par
\taua {আন্তর্জাতিক কর স্বচ্ছতা প্রসঙ্গে তাঁদের দেশের\linebreak}};%
\node[paragraphtext, anchor=north west, text width=314.0pt, align=justify] at (403.0,522.5){\setlength{\baselineskip}{19.8pt} \par
\taua {না প্রচুর গরম লাগছে আমি এই চাদরটি কিনে\linebreak}};%
\node[paragraphtext, anchor=north west, text width=314.0pt, align=justify] at (403.0,504.5){\setlength{\baselineskip}{19.8pt} \par
\taua {উভয় দূতাবাস ধ্বংস হয়েছে। সবরকমের\linebreak}};%
\node[paragraphtext, anchor=north west, text width=314.0pt, align=justify] at (403.0,486.5){\setlength{\baselineskip}{19.8pt} \par
\taua {এই এলাকা, বরফে ঘেরা পাহাড়ী অঞ্চলে, জঙ্গলের\linebreak}};%
\node[paragraphtext, anchor=north west, text width=314.0pt, align=justify] at (403.0,468.5){\setlength{\baselineskip}{19.8pt} \par
\taua {পেতে শুরু করেন। তিনি মিস কানাডা খেতাব\linebreak}};%
\node[paragraphtext, anchor=north west, text width=314.0pt, align=justify] at (403.0,450.5){\setlength{\baselineskip}{19.8pt} \par
\taua {সব কিছুর পরম আশ্রয়। ষষ্ঠ স্নায়ু সাধারণভাবে\linebreak}};%
\node[headertext, anchor=north west, text width=586.0pt, align=left, text=black] at (107.0,1123.5){\setlength{\baselineskip}{40.7pt} \par
\verylarge {অফ অবজারভেশনাল সায়েন্সেস\linebreak}};%
\node[paragraphtext, anchor=north west, text width=339.0pt, align=justify] at (390.0,1048.5){\setlength{\baselineskip}{19.8pt} \par
\taua {ব্যয় করতে পারে বা জেলের সময়ও হতে পারে৷ চেতন\linebreak}};%
\node[paragraphtext, anchor=north west, text width=339.0pt, align=justify] at (390.0,1030.5){\setlength{\baselineskip}{19.8pt} \par
\taua {হলো। এ প্রসঙ্গে তিনি বারাণসীর ল্যাংড়া আম,\linebreak}};%
\node[paragraphtext, anchor=north west, text width=339.0pt, align=justify] at (390.0,1012.5){\setlength{\baselineskip}{19.8pt} \par
\taua {থাকে। লাক্কোম জলপ্রপাত, যা লাক্কম জলপ্রপাত নামেও\linebreak}};%
\node[paragraphtext, anchor=north west, text width=339.0pt, align=justify] at (390.0,994.5){\setlength{\baselineskip}{19.8pt} \par
\taua {চলচ্চিত্রে, একজন পুলিশ কর্মীর ভূমিকায় অভিনয়\linebreak}};%
\node[paragraphtext, anchor=north west, text width=339.0pt, align=justify] at (390.0,976.5){\setlength{\baselineskip}{19.8pt} \par
\taua {খুররমের গঠনমূলক বছরগুলিতে মুঘল দরবারকে ঘিরে\linebreak}};%
\node[paragraphtext, anchor=north west, text width=339.0pt, align=justify] at (390.0,958.5){\setlength{\baselineskip}{19.8pt} \par
\taua {শিশু দিবস পালন করা হয়। হিন্দুধর্মে, কিংবদন্তী বলে\linebreak}};%
\node[paragraphtext, anchor=north west, text width=339.0pt, align=justify] at (390.0,940.5){\setlength{\baselineskip}{19.8pt} \par
\taua {পাদরতাং পিনষ্টূ মাম্ চৈতন্য চরিতামৃত কুড়ি দশমিক\linebreak}};%
\node[paragraphtext, anchor=north west, text width=326.0pt, align=justify] at (394.0,905.5){\setlength{\baselineskip}{19.8pt} \par
\taua {আরবি শব্দ 'বানান' থেকে এসেছে যার অর্থ হল\linebreak}};%
\node[paragraphtext, anchor=north west, text width=326.0pt, align=justify] at (394.0,887.5){\setlength{\baselineskip}{19.8pt} \par
\taua {করতে অস্বীকার করেন। তিনি কালারস টিভি\linebreak}};%
\node[paragraphtext, anchor=north west, text width=326.0pt, align=justify] at (394.0,869.5){\setlength{\baselineskip}{19.8pt} \par
\taua {এটি বেতুল জেলার ভৈঁসদেহী তহসিলের অধীনে পড়ে\linebreak}};%
\node[paragraphtext, anchor=north west, text width=332.0pt, align=justify] at (392.0,711.5){\setlength{\baselineskip}{19.8pt} \par
\taua {উনআশি দিন ধরে এই বীজগুলির যত্ন নেয়। গিরনার\linebreak}};%
\node[paragraphtext, anchor=north west, text width=332.0pt, align=justify] at (392.0,693.5){\setlength{\baselineskip}{19.8pt} \par
\taua {এবং তালে তাল মেলাই এভাবেই চারটে দিনের\linebreak}};%
\node[paragraphtext, anchor=north west, text width=332.0pt, align=justify] at (392.0,675.5){\setlength{\baselineskip}{19.8pt} \par
\taua {নামে একটি মালায়ালাম চলচ্চিত্রে অভিনয়\linebreak}};%
\node[paragraphtext, anchor=north west, text width=330.0pt, align=justify] at (401.0,647.5){\setlength{\baselineskip}{19.8pt} \par
\taua {দলের প্রাক্তন অধিনায়ক সৌরভ গঙ্গোপাধ্যায়, পঙ্কজ\linebreak}};%
\node[paragraphtext, anchor=north west, text width=330.0pt, align=justify] at (401.0,629.5){\setlength{\baselineskip}{19.8pt} \par
\taua {করে। গঙ্গার সমভূমি এবং ভারতের আরব-সাগর\linebreak}};%
\node[paragraphtext, anchor=north west, text width=321.0pt, align=justify] at (395.0,422.5){\setlength{\baselineskip}{19.8pt} \par
\taua {এই রূপগুলোতে ভালোবাসা খুঁজে পায়। যেমন সূফী\linebreak}};%
\node[paragraphtext, anchor=north west, text width=321.0pt, align=justify] at (395.0,404.5){\setlength{\baselineskip}{19.8pt} \par
\taua {লিগ চালু হয়। ঊনিশশো পঁচিশ সালে, মন্দিরে\linebreak}};%
\node[paragraphtext, anchor=north west, text width=321.0pt, align=justify] at (395.0,386.5){\setlength{\baselineskip}{19.8pt} \par
\taua {নেতৃত্ব দেওয়ার সক্ষমতা তৈরী হচ্ছে না। সুতরাংঃ\linebreak}};%
\node[paragraphtext, anchor=north west, text width=321.0pt, align=justify] at (395.0,368.5){\setlength{\baselineskip}{19.8pt} \par
\taua {অংশ হাঁটুর ওপরে মাটিতে নিয়ে আসা। অপ্রচলিত\linebreak}};%
\node[paragraphtext, anchor=north west, text width=321.0pt, align=justify] at (395.0,350.5){\setlength{\baselineskip}{19.8pt} \par
\taua {হিসাবে মনোনীত হয়। ফরাসি পর্যটক ফ্রাঁসোয়া\linebreak}};%
\node[paragraphtext, anchor=north west, text width=321.0pt, align=justify] at (395.0,332.5){\setlength{\baselineskip}{19.8pt} \par
\taua {আলু লাগানোর যন্ত্র বের হয়েছে যাতে তাড়াতাড়ি\linebreak}};%
\node[paragraphtext, anchor=north west, text width=321.0pt, align=justify] at (395.0,314.5){\setlength{\baselineskip}{19.8pt} \par
\taua {বারে যোগ দেয়, যেখানে কালে তাকে উঙ্গলি\linebreak}};%
\node[paragraphtext, anchor=north west, text width=324.0pt, align=justify] at (57.0,457.5){\setlength{\baselineskip}{19.8pt} \par
\taua {মধ্যে একটি তৃতীয় রেলপথ অনুমোদিত করা হয়েছে।\linebreak}};%
\node[paragraphtext, anchor=north west, text width=324.0pt, align=justify] at (57.0,439.5){\setlength{\baselineskip}{19.8pt} \par
\taua {এবং 'গোলমাল: ফান আনলিমিটেড' সহ বেশ\linebreak}};%
\node[paragraphtext, anchor=north west, text width=324.0pt, align=justify] at (57.0,421.5){\setlength{\baselineskip}{19.8pt} \par
\taua {দ্বারা প্রকাশ করা হয়: সা-ষড়জা, রি-ঋষভ,\linebreak}};%
\node[paragraphtext, anchor=north west, text width=324.0pt, align=justify] at (57.0,403.5){\setlength{\baselineskip}{19.8pt} \par
\taua {থেকে মূলত রপ্তানির উদ্দেশ্যে লৌহ আকরিক\linebreak}};%
\node[paragraphtext, anchor=north west, text width=324.0pt, align=justify] at (57.0,385.5){\setlength{\baselineskip}{19.8pt} \par
\taua {কেটে নিয়ে জমা করে দেবে। চৈতন্য দাস একজন\linebreak}};%
\node[paragraphtext, anchor=north west, text width=324.0pt, align=justify] at (57.0,367.5){\setlength{\baselineskip}{19.8pt} \par
\taua {সঙ্গে ডেটিং শুরু করেন। তিনি আরও পার্শ্ব চরিত্রে\linebreak}};%
\node[paragraphtext, anchor=north west, text width=324.0pt, align=justify] at (57.0,349.5){\setlength{\baselineskip}{19.8pt} \par
\taua {করেছিলেন, দুর্ঘটনার পরপরই তাঁরা ব্লু গ্রাস\linebreak}};%
\node[paragraphtext, anchor=north west, text width=324.0pt, align=justify] at (57.0,331.5){\setlength{\baselineskip}{19.8pt} \par
\taua {হয় অঙ্কন হাত বা স্ট্রিং হাত। এর জন্য অন্য সংস্কৃত\linebreak}};%
\node[paragraphtext, anchor=north west, text width=324.0pt, align=justify] at (57.0,313.5){\setlength{\baselineskip}{19.8pt} \par
\taua {আশেপাশে খুব কম পর্যটক থাকায়, আপনি প্রশান্ত\linebreak}};%
\node[paragraphtext, anchor=north west, text width=324.0pt, align=justify] at (57.0,295.5){\setlength{\baselineskip}{19.8pt} \par
\taua {সমুদ্র খাত। আমার কেন্দ্রীয় মন্ত্রিসভার সহকর্মী ডক্টর\linebreak}};%
\node[paragraphtext, anchor=north west, text width=324.0pt, align=justify] at (57.0,277.5){\setlength{\baselineskip}{19.8pt} \par
\taua {অভিযুক্ত করা হয়েছে৷ দ্য ইংলিশ টিচার'-এর পর,\linebreak}};%
\node[paragraphtext, anchor=north west, text width=324.0pt, align=justify] at (57.0,259.5){\setlength{\baselineskip}{19.8pt} \par
\taua {এটির প্রচার বেশি হয়, এবং পাশ্চাত্যদেশবাসীদের\linebreak}};%
\node[paragraphtext, anchor=north west, text width=324.0pt, align=justify] at (57.0,241.5){\setlength{\baselineskip}{19.8pt} \par
\taua {করা হয়। পরবর্তীতে তিনি আইআইএসসি-তে\linebreak}};%
\node[paragraphtext, anchor=north west, text width=324.0pt, align=justify] at (57.0,223.5){\setlength{\baselineskip}{19.8pt} \par
\taua {ঘেরাও করে। পল্লবী পংক্তিটি বিভিন্ন লয়ে, রাগমের\linebreak}};%
\node[paragraphtext, anchor=north west, text width=324.0pt, align=justify] at (57.0,205.5){\setlength{\baselineskip}{19.8pt} \par
\taua {যে সবকিছু মাথা পেতে মেনে নেন না তাঁরা আমলাগণ\linebreak}};%
\node[paragraphtext, anchor=north west, text width=324.0pt, align=justify] at (57.0,187.5){\setlength{\baselineskip}{19.8pt} \par
\taua {একটি নকল শনাক্তকরণ পেন কিনতে পারেন যার\linebreak}};%
\node[paragraphtext, anchor=north west, text width=324.0pt, align=justify] at (57.0,169.5){\setlength{\baselineskip}{19.8pt} \par
\taua {ছয়-এ, খানকে পাঁচ বছরের জেল দেওয়া হয় এবং\linebreak}};%
\node[paragraphtext, anchor=north west, text width=324.0pt, align=justify] at (57.0,151.5){\setlength{\baselineskip}{19.8pt} \par
\taua {জালালাবাদে সেভ দ্য চিলড্রেন-এর অফিসে\linebreak}};%
\node[paragraphtext, anchor=north west, text width=324.0pt, align=justify] at (57.0,133.5){\setlength{\baselineskip}{19.8pt} \par
\taua {মানে দিনে তিন চারবার পায়খানা হচ্ছে নাকি কাল\linebreak}};%
\node[paragraphtext, anchor=north west, text width=324.0pt, align=justify] at (57.0,115.5){\setlength{\baselineskip}{19.8pt} \par
\taua {হত্যা করার পর, তিনি দিল্লিতে তার ক্ষমতা সুসংহত\linebreak}};%
\node[paragraphtext, anchor=north west, text width=165.0pt, align=left] at (759.0,331.5){\setlength{\baselineskip}{19.8pt} \par
\taua {কার্যকালের মেয়াদ এক\linebreak}};%
\node[paragraphtext, anchor=north west, text width=165.0pt, align=left] at (759.0,313.5){\setlength{\baselineskip}{19.8pt} \par
\taua {লিখিত আবখাজ সাহিত্য\linebreak}};%
\node[paragraphtext, anchor=north west, text width=165.0pt, align=left] at (759.0,295.5){\setlength{\baselineskip}{19.8pt} \par
\taua {করি, তাহলে নিশ্চিতরূপে\linebreak}};%
\node[paragraphtext, anchor=north west, text width=223.0pt, align=left] at (872.0,198.5){\setlength{\baselineskip}{19.8pt} \par
\taua {থায়ে নি কান্নুরাঙ্গু নামে একটি\linebreak}};%
\node[paragraphtext, anchor=north west, text width=223.0pt, align=left] at (872.0,180.5){\setlength{\baselineskip}{19.8pt} \par
\taua {ইজ পিঙ্ক’ এবং ‘তুফান’ চলচ্চিত্রে\linebreak}};%
\node[paragraphtext, anchor=north west, text width=223.0pt, align=left] at (872.0,162.5){\setlength{\baselineskip}{19.8pt} \par
\taua {যে সন্তানরা তারা ছোটোবেলা থেকে\linebreak}};%
\node[paragraphtext, anchor=north west, text width=223.0pt, align=left] at (872.0,144.5){\setlength{\baselineskip}{19.8pt} \par
\taua {সমভূমিতে অবস্থিত। পুদুচেরি\linebreak}};%
\node[paragraphtext, anchor=north west, text width=223.0pt, align=left] at (872.0,126.5){\setlength{\baselineskip}{19.8pt} \par
\taua {এলাকায় সাইকেল চালানো জনপ্রিয়\linebreak}};%
\end{tikzpicture}%
\end{center}%
\end{document}