\documentclass[11pt]{article}%
\usepackage[T1]{fontenc}%
\usepackage[utf8]{inputenc}%
\usepackage{lmodern}%
\usepackage{textcomp}%
\usepackage{lastpage}%
\usepackage{tikz}%
\usepackage{fontspec}%
\usepackage{polyglossia}%
\usepackage[paperwidth=595pt,paperheight=842pt,margin=0pt]{geometry}%
%
\setmainlanguage{bengali}%
\setotherlanguage{english}%
\newfontfamily\bengalifont[Script=Bengali,Path=/home/vivek/Pager/fonts/bengali/Paragraph/]{NotoSansBengali_SemiCondensed-Light}%
\newfontfamily\headerfont[Script=Bengali,Path=/home/vivek/Pager/fonts/bengali/Header/]{NotoSerifBengali_ExtraCondensed-Black}%
\newfontfamily\paragraphfont[Script=Bengali,Path=/home/vivek/Pager/fonts/bengali/Paragraph/]{NotoSansBengali_SemiCondensed-Light}%

    \makeatletter
    \newcommand{\zettaHuge}{\@setfontsize\zettaHuge{200}{220}}
    \newcommand{\exaHuge}{\@setfontsize\exaHuge{165}{180}}
    \newcommand{\petaHuge}{\@setfontsize\petaHuge{135}{150}}
    \newcommand{\teraHuge}{\@setfontsize\teraHuge{110}{120}}
    \newcommand{\gigaHuge}{\@setfontsize\gigaHuge{90}{100}}
    \newcommand{\megaHuge}{\@setfontsize\megaHuge{75}{85}}
    \newcommand{\superHuge}{\@setfontsize\superHuge{62}{70}}
    \newcommand{\verylarge}{\@setfontsize\verylarge{37}{42}}
    \newcommand{\veryLarge}{\@setfontsize\veryLarge{43}{49}}
    \newcommand{\veryHuge}{\@setfontsize\veryHuge{62}{70}}
    \newcommand{\alphaa}{\@setfontsize\alphaa{60}{66}}
    \newcommand{\betaa}{\@setfontsize\betaa{57}{63}}
    \newcommand{\gammaa}{\@setfontsize\gammaa{55}{61}}
    \newcommand{\deltaa}{\@setfontsize\deltaa{53}{59}}
    \newcommand{\epsilona}{\@setfontsize\epsilona{51}{57}}
    \newcommand{\zetaa}{\@setfontsize\zetaa{47}{53}}
    \newcommand{\etaa}{\@setfontsize\etaa{45}{51}}
    \newcommand{\iotaa}{\@setfontsize\iotaa{41}{47}}
    \newcommand{\kappaa}{\@setfontsize\kappaa{39}{45}}
    \newcommand{\lambdaa}{\@setfontsize\lambdaa{35}{41}}
    \newcommand{\mua}{\@setfontsize\mua{33}{39}}
    \newcommand{\nua}{\@setfontsize\nua{31}{37}}
    \newcommand{\xia}{\@setfontsize\xia{29}{35}}
    \newcommand{\pia}{\@setfontsize\pia{27}{33}}
    \newcommand{\rhoa}{\@setfontsize\rhoa{24}{30}}
    \newcommand{\sigmaa}{\@setfontsize\sigmaa{22}{28}}
    \newcommand{\taua}{\@setfontsize\taua{18}{24}}
    \newcommand{\upsilona}{\@setfontsize\upsilona{16}{22}}
    \newcommand{\phia}{\@setfontsize\phia{15}{20}}
    \newcommand{\chia}{\@setfontsize\chia{13}{18}}
    \newcommand{\psia}{\@setfontsize\psia{11}{16}}
    \newcommand{\omegaa}{\@setfontsize\omegaa{6}{7}}
    \newcommand{\oomegaa}{\@setfontsize\oomegaa{4}{5}}
    \newcommand{\ooomegaa}{\@setfontsize\ooomegaa{3}{4}}
    \newcommand{\oooomegaaa}{\@setfontsize\oooomegaaa{2}{3}}
    \makeatother
    %
%
\begin{document}%
\normalsize%
\begin{center}%
\begin{tikzpicture}[x=1pt, y=1pt]%
\node[anchor=south west, inner sep=0pt] at (0,0) {\includegraphics[width=595pt,height=842pt]{/home/vivek/Pager/images_val/mg_or_000933_1.png}};%
\tikzset{headertext/.style={font=\headerfont, text=black}}%
\tikzset{paragraphtext/.style={font=\paragraphfont, text=black}}%
\node[paragraphtext, anchor=north west, text width=403.0pt, align=left] at (92.0,157.5){\setlength{\baselineskip}{19.8pt} \par
\taua {একটি ব্যাপকভাবে ব্যবহৃত বিশ্লেষণ সাধনী। মেডস্টোরি নামে\linebreak}};%
\node[paragraphtext, anchor=north west, text width=403.0pt, align=left] at (92.0,139.5){\setlength{\baselineskip}{19.8pt} \par
\taua {মুখোশের পাশাপাশি পুরুলিয়াতে আবার বিড়ি বাঁধার শিল্পও\linebreak}};%
\node[paragraphtext, anchor=north west, text width=403.0pt, align=left] at (92.0,121.5){\setlength{\baselineskip}{19.8pt} \par
\taua {মধ্যে একটি, যেখান থেকে জল নিয়ে খালের মাধ্যমে পুঢ়াল হ্রদ\linebreak}};%
\node[paragraphtext, anchor=north west, text width=403.0pt, align=left] at (92.0,103.5){\setlength{\baselineskip}{19.8pt} \par
\taua {গিয়ে তাদের ফিরিয়ে দেওয়ার বিনিময়ে যাত্রীদের নিয়ে আসেন।\linebreak}};%
\node[paragraphtext, anchor=north west, text width=403.0pt, align=left] at (92.0,85.5){\setlength{\baselineskip}{19.8pt} \par
\taua {পিএইচডি চলাকালীন কেমিক্যাল ইঞ্জিনিয়ারিংয়ের ছাত্রছাত্রীরা\linebreak}};%
\node[paragraphtext, anchor=north west, text width=403.0pt, align=left] at (92.0,67.5){\setlength{\baselineskip}{19.8pt} \par
\taua {তিনি ম্যাকগুইগানের মুখে ঘুঁষি মারলে, মারামারি করতে শুরু\linebreak}};%
\node[paragraphtext, anchor=north west, text width=423.0pt, align=justify] at (88.0,467.5){\setlength{\baselineskip}{19.8pt} \par
\taua {ঊনিশশো পঁয়ষট্টিতে কুমার ভারী জলের তড়িৎ-বিশিষ্ট উৎপাদন নিয়ে\linebreak}};%
\node[paragraphtext, anchor=north west, text width=423.0pt, align=justify] at (88.0,449.5){\setlength{\baselineskip}{19.8pt} \par
\taua {কনক্রিটের জঙ্গল তৈরি হচ্ছে উপযুক্ত মাঠই যদি না পায় তারা\linebreak}};%
\node[paragraphtext, anchor=north west, text width=423.0pt, align=justify] at (88.0,431.5){\setlength{\baselineskip}{19.8pt} \par
\taua {সহজানন্দ সরস্বতীজীর মতো অসংখ্য দেবতুল্য ব্যক্তিত্বেরও প্রেরণার\linebreak}};%
\node[paragraphtext, anchor=north west, text width=183.0pt, align=left] at (206.0,80.5){\setlength{\baselineskip}{16.5pt} \par
\Large {নগরীর সাথে উত্তরে জেলা স্তরের\linebreak}};%
\node[paragraphtext, anchor=north west, text width=246.0pt] at (329.0,398.5) {\taua{• মহিলা ছিলেন। আঠারোশো সাতান্ন}};%
\node[paragraphtext, anchor=north west, text width=246.0pt] at (329.0,380.5) {\taua{• করে। দ্য ইকোনোমিস্ট -এর প্রতিবেদন}};%
\node[paragraphtext, anchor=north west, text width=246.0pt] at (329.0,362.5) {\taua{• বা পান করতে অসুবিধা হবে। তিনি বেশ}};%
\node[paragraphtext, anchor=north west, text width=246.0pt] at (329.0,344.5) {\taua{• গেছে, সেই রাস্তা ধরে হাঁটতে থাকল সে।}};%
\node[paragraphtext, anchor=north west, text width=246.0pt] at (329.0,326.5) {\taua{• কোটি, পচাত্তর লক্ষ, ছেষট্টি হাজার,}};%
\node[paragraphtext, anchor=north west, text width=246.0pt] at (329.0,308.5) {\taua{• ব্রিটিশ ইস্ট ইন্ডিয়া কোম্পানির বিরুদ্ধে}};%
\node[paragraphtext, anchor=north west, text width=246.0pt] at (329.0,290.5) {\taua{• রয়েছে। কয়েকদিন বাদে শ্রীঙ্গেরি}};%
\node[paragraphtext, anchor=north west, text width=246.0pt] at (329.0,272.5) {\taua{• মেদিনীপুর, বাঁকুড়া, বীরভূ্ম, পুরুলিয়া,}};%
\node[paragraphtext, anchor=north west, text width=246.0pt] at (329.0,254.5) {\taua{• কৃষি-প্রক্রিয়ার কারণে বন উজাড় করা।}};%
\node[paragraphtext, anchor=north west, text width=246.0pt] at (329.0,236.5) {\taua{• পর তাঁকে চরম অশ্লীলতার দায়ে দোষী}};%
\node[paragraphtext, anchor=north west, text width=246.0pt] at (329.0,218.5) {\taua{• সাতচল্লিশ সালে ব্রিটিশরা ভারত ছেড়ে}};%
\node[paragraphtext, anchor=north west, text width=105.0pt, align=left] at (245.0,722.5){\setlength{\baselineskip}{19.8pt} \par
\taua {আনা হয়েছিল।\linebreak}};%
\end{tikzpicture}%
\end{center}%
\end{document}