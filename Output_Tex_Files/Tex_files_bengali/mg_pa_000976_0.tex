\documentclass[11pt]{article}%
\usepackage[T1]{fontenc}%
\usepackage[utf8]{inputenc}%
\usepackage{lmodern}%
\usepackage{textcomp}%
\usepackage{lastpage}%
\usepackage{tikz}%
\usepackage{fontspec}%
\usepackage{polyglossia}%
\usepackage[paperwidth=1656pt,paperheight=2339pt,margin=0pt]{geometry}%
%
\setmainlanguage{bengali}%
\setotherlanguage{english}%
\newfontfamily\bengalifont[Script=Bengali,Path=/home/vivek/Pager/fonts/bengali/Paragraph/]{NotoSerifBengali_ExtraCondensed-Regular}%
\newfontfamily\headerfont[Script=Bengali,Path=/home/vivek/Pager/fonts/bengali/Header/]{BalooDa2-SemiBold}%
\newfontfamily\paragraphfont[Script=Bengali,Path=/home/vivek/Pager/fonts/bengali/Paragraph/]{NotoSerifBengali_ExtraCondensed-Regular}%

    \makeatletter
    \newcommand{\zettaHuge}{\@setfontsize\zettaHuge{200}{220}}
    \newcommand{\exaHuge}{\@setfontsize\exaHuge{165}{180}}
    \newcommand{\petaHuge}{\@setfontsize\petaHuge{135}{150}}
    \newcommand{\teraHuge}{\@setfontsize\teraHuge{110}{120}}
    \newcommand{\gigaHuge}{\@setfontsize\gigaHuge{90}{100}}
    \newcommand{\megaHuge}{\@setfontsize\megaHuge{75}{85}}
    \newcommand{\superHuge}{\@setfontsize\superHuge{62}{70}}
    \newcommand{\verylarge}{\@setfontsize\verylarge{37}{42}}
    \newcommand{\veryLarge}{\@setfontsize\veryLarge{43}{49}}
    \newcommand{\veryHuge}{\@setfontsize\veryHuge{62}{70}}
    \newcommand{\alphaa}{\@setfontsize\alphaa{60}{66}}
    \newcommand{\betaa}{\@setfontsize\betaa{57}{63}}
    \newcommand{\gammaa}{\@setfontsize\gammaa{55}{61}}
    \newcommand{\deltaa}{\@setfontsize\deltaa{53}{59}}
    \newcommand{\epsilona}{\@setfontsize\epsilona{51}{57}}
    \newcommand{\zetaa}{\@setfontsize\zetaa{47}{53}}
    \newcommand{\etaa}{\@setfontsize\etaa{45}{51}}
    \newcommand{\iotaa}{\@setfontsize\iotaa{41}{47}}
    \newcommand{\kappaa}{\@setfontsize\kappaa{39}{45}}
    \newcommand{\lambdaa}{\@setfontsize\lambdaa{35}{41}}
    \newcommand{\mua}{\@setfontsize\mua{33}{39}}
    \newcommand{\nua}{\@setfontsize\nua{31}{37}}
    \newcommand{\xia}{\@setfontsize\xia{29}{35}}
    \newcommand{\pia}{\@setfontsize\pia{27}{33}}
    \newcommand{\rhoa}{\@setfontsize\rhoa{24}{30}}
    \newcommand{\sigmaa}{\@setfontsize\sigmaa{22}{28}}
    \newcommand{\taua}{\@setfontsize\taua{18}{24}}
    \newcommand{\upsilona}{\@setfontsize\upsilona{16}{22}}
    \newcommand{\phia}{\@setfontsize\phia{15}{20}}
    \newcommand{\chia}{\@setfontsize\chia{13}{18}}
    \newcommand{\psia}{\@setfontsize\psia{11}{16}}
    \newcommand{\omegaa}{\@setfontsize\omegaa{6}{7}}
    \newcommand{\oomegaa}{\@setfontsize\oomegaa{4}{5}}
    \newcommand{\ooomegaa}{\@setfontsize\ooomegaa{3}{4}}
    \newcommand{\oooomegaaa}{\@setfontsize\oooomegaaa{2}{3}}
    \makeatother
    %
%
\begin{document}%
\normalsize%
\begin{center}%
\begin{tikzpicture}[x=1pt, y=1pt]%
\node[anchor=south west, inner sep=0pt] at (0,0) {\includegraphics[width=1656pt,height=2339pt]{/home/vivek/Pager/images_val/mg_pa_000976_0.png}};%
\tikzset{headertext/.style={font=\headerfont, text=black}}%
\tikzset{paragraphtext/.style={font=\paragraphfont, text=black}}%
\node[paragraphtext, anchor=north west, text width=641.0pt, align=justify] at (174.0,2139.5){\setlength{\baselineskip}{47.300000000000004pt} \par
\veryLarge {মার্গের মধুসূদন কলেজ রঘুনাথপুর কলেজ\linebreak}};%
\node[paragraphtext, anchor=north west, text width=641.0pt, align=justify] at (174.0,2096.0){\setlength{\baselineskip}{47.300000000000004pt} \par
\veryLarge {আইআইটি কানপুরে কেমিক্যাল ইঞ্জিনিয়ারিংয়ের\linebreak}};%
\node[paragraphtext, anchor=north west, text width=641.0pt, align=justify] at (174.0,2052.5){\setlength{\baselineskip}{47.300000000000004pt} \par
\veryLarge {আপেক্ষিক ঐক্যর সঙ্গেও ব্যবহৃত হয়। বেশ কয়েকটি\linebreak}};%
\node[paragraphtext, anchor=north west, text width=641.0pt, align=justify] at (174.0,2009.0){\setlength{\baselineskip}{47.300000000000004pt} \par
\veryLarge {পশ্চিম চীনদেশের মৌঙ মাও রাজ্যের থেকে বারোশো\linebreak}};%
\node[paragraphtext, anchor=north west, text width=641.0pt, align=justify] at (174.0,1965.5){\setlength{\baselineskip}{47.300000000000004pt} \par
\veryLarge {অথবা গলায় ঘড়ঘড় আওয়াজ, শ্বাসকষ্ট এবং বুকে\linebreak}};%
\node[paragraphtext, anchor=north west, text width=641.0pt, align=justify] at (174.0,1922.0){\setlength{\baselineskip}{47.300000000000004pt} \par
\veryLarge {বলে থাকি এবং এচওএমও -কে নিউক্লিওফিলিক\linebreak}};%
\node[paragraphtext, anchor=north west, text width=641.0pt, align=justify] at (174.0,1878.5){\setlength{\baselineskip}{47.300000000000004pt} \par
\veryLarge {টিউমারের গৌণ ক্ষেত্রেও ঘটতে পারে। একইভাবে\linebreak}};%
\node[paragraphtext, anchor=north west, text width=641.0pt, align=justify] at (174.0,1835.0){\setlength{\baselineskip}{47.300000000000004pt} \par
\veryLarge {থেকে একজন হৃদয়হীন মানুষের সহজ রূপান্তরের\linebreak}};%
\node[paragraphtext, anchor=north west, text width=641.0pt, align=justify] at (174.0,1791.5){\setlength{\baselineskip}{47.300000000000004pt} \par
\veryLarge {জরুরী ছিল। অনেক হয়েছে ওর হাঁকাহাঁকি। এর ঊর্ধ্ব\linebreak}};%
\node[paragraphtext, anchor=north west, text width=693.0pt, align=justify] at (178.0,1566.5){\setlength{\baselineskip}{47.300000000000004pt} \par
\veryLarge {চ্যালেঞ্জের সম্মুখীন হলে সহায়ক শৈলীগুলি অন্তর্ভুক্ত\linebreak}};%
\node[paragraphtext, anchor=north west, text width=693.0pt, align=justify] at (178.0,1523.0){\setlength{\baselineskip}{47.300000000000004pt} \par
\veryLarge {ভারত অববাহিকার মানে হচ্ছে ভারতীয় প্লেট ও একটি\linebreak}};%
\node[paragraphtext, anchor=north west, text width=693.0pt, align=justify] at (178.0,1479.5){\setlength{\baselineskip}{47.300000000000004pt} \par
\veryLarge {যে ভূমিকায় রয়েছেন, এই অভিযানে অংশগ্রহণ করছেন,\linebreak}};%
\node[paragraphtext, anchor=north west, text width=693.0pt, align=justify] at (178.0,1436.0){\setlength{\baselineskip}{47.300000000000004pt} \par
\veryLarge {খরচ মেটাতে একাধিক ম্যাচ খেলে। ভারতের মধ্য\linebreak}};%
\node[paragraphtext, anchor=north west, text width=693.0pt, align=justify] at (178.0,1392.5){\setlength{\baselineskip}{47.300000000000004pt} \par
\veryLarge {একই অক্ষাংশে অবস্থান করা সত্ত্বেও শীতকালে কলকাতা\linebreak}};%
\node[paragraphtext, anchor=north west, text width=693.0pt, align=justify] at (178.0,1349.0){\setlength{\baselineskip}{47.300000000000004pt} \par
\veryLarge {জেলায় তিরানব্বই হাজার, আটশো পচাশি হেক্টর জমি\linebreak}};%
\node[paragraphtext, anchor=north west, text width=693.0pt, align=justify] at (178.0,1305.5){\setlength{\baselineskip}{47.300000000000004pt} \par
\veryLarge {হয়ে যেতে পারে, যার ফলে নাভির তন্ত্রী সংকোচন এবং\linebreak}};%
\node[paragraphtext, anchor=north west, text width=693.0pt, align=justify] at (178.0,1262.0){\setlength{\baselineskip}{47.300000000000004pt} \par
\veryLarge {ওয়ালেট -এর ব্যবহার কেমন ভাবে হচ্ছে আর এই\linebreak}};%
\node[paragraphtext, anchor=north west, text width=693.0pt, align=justify] at (178.0,1218.5){\setlength{\baselineskip}{47.300000000000004pt} \par
\veryLarge {একটি বড় দলের সঙ্গে হয়েছিল এবং মুঘল দরবারের\linebreak}};%
\node[paragraphtext, anchor=north west, text width=693.0pt, align=justify] at (178.0,1175.0){\setlength{\baselineskip}{47.300000000000004pt} \par
\veryLarge {কাছে যাওয়ার পরে, যুদ্ধে বীর যবনরা, ফুলের মানের শহর\linebreak}};%
\node[paragraphtext, anchor=north west, text width=316.0pt, align=left] at (528.0,901.5){\setlength{\baselineskip}{47.300000000000004pt} \par
\veryLarge {রিগ্রেশন বিশ্লেষণের\linebreak}};%
\node[paragraphtext, anchor=north west, text width=316.0pt, align=left] at (528.0,858.0){\setlength{\baselineskip}{47.300000000000004pt} \par
\veryLarge {সেগুলি একটি বন্ধ ব্যাগে\linebreak}};%
\node[paragraphtext, anchor=north west, text width=316.0pt, align=left] at (528.0,814.5){\setlength{\baselineskip}{47.300000000000004pt} \par
\veryLarge {এবং বন্য শুয়োর, স্লথ\linebreak}};%
\node[headertext, anchor=north west, text width=421.0pt, align=left] at (279.0,103.5){\setlength{\baselineskip}{47.300000000000004pt} \par
\veryLarge {ফর্ম্যাটটি কোনও সরকারি সংস্থার\linebreak}};%
\end{tikzpicture}%
\end{center}%
\end{document}