\documentclass[11pt]{article}%
\usepackage[T1]{fontenc}%
\usepackage[utf8]{inputenc}%
\usepackage{lmodern}%
\usepackage{textcomp}%
\usepackage{lastpage}%
\usepackage{tikz}%
\usepackage{fontspec}%
\usepackage{polyglossia}%
\usepackage[paperwidth=1528pt,paperheight=2084pt,margin=0pt]{geometry}%
%
\setmainlanguage{bengali}%
\setotherlanguage{english}%
\newfontfamily\bengalifont[Script=Bengali,Path=/home/vivek/Pager/fonts/bengali/Paragraph/]{Atma-Regular}%
\newfontfamily\headerfont[Script=Bengali,Path=/home/vivek/Pager/fonts/bengali/Header/]{Atma-Bold}%
\newfontfamily\paragraphfont[Script=Bengali,Path=/home/vivek/Pager/fonts/bengali/Paragraph/]{Atma-Regular}%

    \makeatletter
    \newcommand{\zettaHuge}{\@setfontsize\zettaHuge{200}{220}}
    \newcommand{\exaHuge}{\@setfontsize\exaHuge{165}{180}}
    \newcommand{\petaHuge}{\@setfontsize\petaHuge{135}{150}}
    \newcommand{\teraHuge}{\@setfontsize\teraHuge{110}{120}}
    \newcommand{\gigaHuge}{\@setfontsize\gigaHuge{90}{100}}
    \newcommand{\megaHuge}{\@setfontsize\megaHuge{75}{85}}
    \newcommand{\superHuge}{\@setfontsize\superHuge{62}{70}}
    \newcommand{\verylarge}{\@setfontsize\verylarge{37}{42}}
    \newcommand{\veryLarge}{\@setfontsize\veryLarge{43}{49}}
    \newcommand{\veryHuge}{\@setfontsize\veryHuge{62}{70}}
    \newcommand{\alphaa}{\@setfontsize\alphaa{60}{66}}
    \newcommand{\betaa}{\@setfontsize\betaa{57}{63}}
    \newcommand{\gammaa}{\@setfontsize\gammaa{55}{61}}
    \newcommand{\deltaa}{\@setfontsize\deltaa{53}{59}}
    \newcommand{\epsilona}{\@setfontsize\epsilona{51}{57}}
    \newcommand{\zetaa}{\@setfontsize\zetaa{47}{53}}
    \newcommand{\etaa}{\@setfontsize\etaa{45}{51}}
    \newcommand{\iotaa}{\@setfontsize\iotaa{41}{47}}
    \newcommand{\kappaa}{\@setfontsize\kappaa{39}{45}}
    \newcommand{\lambdaa}{\@setfontsize\lambdaa{35}{41}}
    \newcommand{\mua}{\@setfontsize\mua{33}{39}}
    \newcommand{\nua}{\@setfontsize\nua{31}{37}}
    \newcommand{\xia}{\@setfontsize\xia{29}{35}}
    \newcommand{\pia}{\@setfontsize\pia{27}{33}}
    \newcommand{\rhoa}{\@setfontsize\rhoa{24}{30}}
    \newcommand{\sigmaa}{\@setfontsize\sigmaa{22}{28}}
    \newcommand{\taua}{\@setfontsize\taua{18}{24}}
    \newcommand{\upsilona}{\@setfontsize\upsilona{16}{22}}
    \newcommand{\phia}{\@setfontsize\phia{15}{20}}
    \newcommand{\chia}{\@setfontsize\chia{13}{18}}
    \newcommand{\psia}{\@setfontsize\psia{11}{16}}
    \newcommand{\omegaa}{\@setfontsize\omegaa{6}{7}}
    \newcommand{\oomegaa}{\@setfontsize\oomegaa{4}{5}}
    \newcommand{\ooomegaa}{\@setfontsize\ooomegaa{3}{4}}
    \newcommand{\oooomegaaa}{\@setfontsize\oooomegaaa{2}{3}}
    \makeatother
    %
%
\begin{document}%
\normalsize%
\begin{center}%
\begin{tikzpicture}[x=1pt, y=1pt]%
\node[anchor=south west, inner sep=0pt] at (0,0) {\includegraphics[width=1528pt,height=2084pt]{/home/vivek/Pager/images_val/mg_kn_000420_0.png}};%
\tikzset{headertext/.style={font=\headerfont, text=black}}%
\tikzset{paragraphtext/.style={font=\paragraphfont, text=black}}%
\node[paragraphtext, anchor=north west, text width=383.0pt, align=justify] at (384.0,1682.5){\setlength{\baselineskip}{47.300000000000004pt} \par
\veryLarge {মাহ্‌ল। প্রক্রিয়াটি সাধারণত\linebreak}};%
\node[paragraphtext, anchor=north west, text width=383.0pt, align=justify] at (384.0,1639.0){\setlength{\baselineskip}{47.300000000000004pt} \par
\veryLarge {অ্যাসিডে সব একক বন্ড\linebreak}};%
\node[paragraphtext, anchor=north west, text width=383.0pt, align=justify] at (384.0,1595.5){\setlength{\baselineskip}{47.300000000000004pt} \par
\veryLarge {অভিযোগ করেন যে টাইগার\linebreak}};%
\node[paragraphtext, anchor=north west, text width=383.0pt, align=justify] at (384.0,1552.0){\setlength{\baselineskip}{47.300000000000004pt} \par
\veryLarge {অনেকাংশেই বন্ধ হয়ে যায়।\linebreak}};%
\node[paragraphtext, anchor=north west, text width=383.0pt, align=justify] at (384.0,1508.5){\setlength{\baselineskip}{47.300000000000004pt} \par
\veryLarge {থেকে একজন হৃদয়হীন\linebreak}};%
\node[paragraphtext, anchor=north west, text width=383.0pt, align=justify] at (384.0,1465.0){\setlength{\baselineskip}{47.300000000000004pt} \par
\veryLarge {সময়ে একটি জাতীয় উশু\linebreak}};%
\node[paragraphtext, anchor=north west, text width=650.0pt, align=justify] at (116.0,1136.5){\setlength{\baselineskip}{47.300000000000004pt} \par
\veryLarge {অবস্থানকে দৃঢ় করেছিল। শহরের উচ্চ-পদস্ত\linebreak}};%
\node[paragraphtext, anchor=north west, text width=650.0pt, align=justify] at (116.0,1093.0){\setlength{\baselineskip}{47.300000000000004pt} \par
\veryLarge {অ্যান্ড নিউ মেটেরিয়ালস্ এবং নয়ডার সংস্থা\linebreak}};%
\node[paragraphtext, anchor=north west, text width=650.0pt, align=justify] at (116.0,1049.5){\setlength{\baselineskip}{47.300000000000004pt} \par
\veryLarge {আবেশকের ব্যবহার অপরিহার্য। ঊনিশশো\linebreak}};%
\node[paragraphtext, anchor=north west, text width=650.0pt, align=justify] at (116.0,1006.0){\setlength{\baselineskip}{47.300000000000004pt} \par
\veryLarge {পরীক্ষা নিরীক্ষা করতে পারেন, যেমন হল কভার\linebreak}};%
\node[paragraphtext, anchor=north west, text width=650.0pt, align=justify] at (116.0,962.5){\setlength{\baselineskip}{47.300000000000004pt} \par
\veryLarge {গ্রীষ্মকালীন রাজধানী হিসাবেও কাজ করেছিল।\linebreak}};%
\node[paragraphtext, anchor=north west, text width=387.0pt, align=justify] at (384.0,1376.5){\setlength{\baselineskip}{47.300000000000004pt} \par
\veryLarge {এবং কোড়গুর সর্বোচ্চ\linebreak}};%
\node[paragraphtext, anchor=north west, text width=387.0pt, align=justify] at (384.0,1333.0){\setlength{\baselineskip}{47.300000000000004pt} \par
\veryLarge {আপনি সেদিন বলছিলেন যে\linebreak}};%
\node[paragraphtext, anchor=north west, text width=387.0pt, align=justify] at (384.0,1289.5){\setlength{\baselineskip}{47.300000000000004pt} \par
\veryLarge {বিস্তৃত হতে এবং সমাজ\linebreak}};%
\node[paragraphtext, anchor=north west, text width=659.0pt, align=justify] at (109.0,660.5){\setlength{\baselineskip}{47.300000000000004pt} \par
\veryLarge {সুভাষচন্দ্র বসুর গান্ধীজির অহিংস প্রতিরোধের\linebreak}};%
\node[paragraphtext, anchor=north west, text width=659.0pt, align=justify] at (109.0,617.0){\setlength{\baselineskip}{47.300000000000004pt} \par
\veryLarge {অ্যাপেন্ডিসাইটিসের ব্যথা নাভির চারপাশে হাল্কা\linebreak}};%
\node[paragraphtext, anchor=north west, text width=659.0pt, align=justify] at (109.0,573.5){\setlength{\baselineskip}{47.300000000000004pt} \par
\veryLarge {দুই সালে ফ্রেড গেইসবার্গের ভারতীয় সঙ্গীতের\linebreak}};%
\node[paragraphtext, anchor=north west, text width=659.0pt, align=justify] at (109.0,530.0){\setlength{\baselineskip}{47.300000000000004pt} \par
\veryLarge {অবস্থিত একটি পুরু টেকসই ঝিল্লি। তিনি\linebreak}};%
\node[paragraphtext, anchor=north west, text width=663.0pt, align=justify] at (792.0,1536.5){\setlength{\baselineskip}{47.300000000000004pt} \par
\veryLarge {যোগাযোগ করতে পারেন বা নিজে নিজেই কাজের\linebreak}};%
\node[paragraphtext, anchor=north west, text width=663.0pt, align=justify] at (792.0,1493.0){\setlength{\baselineskip}{47.300000000000004pt} \par
\veryLarge {নরসিংহ রাও প্রতিবেশী রাষ্ট্র শ্রীলঙ্কায় একটি দুগ্ধ\linebreak}};%
\node[paragraphtext, anchor=north west, text width=663.0pt, align=justify] at (792.0,1449.5){\setlength{\baselineskip}{47.300000000000004pt} \par
\veryLarge {সড়কে বাইপাস ভাগ হয়ে যাওয়ার প্রায় পাঁচশো\linebreak}};%
\node[paragraphtext, anchor=north west, text width=655.0pt, align=justify] at (112.0,282.5){\setlength{\baselineskip}{47.300000000000004pt} \par
\veryLarge {বিষয়বস্তু নিয়ে বিক্রম চরিত্রে বচ্চন আরও\linebreak}};%
\node[paragraphtext, anchor=north west, text width=655.0pt, align=justify] at (112.0,239.0){\setlength{\baselineskip}{47.300000000000004pt} \par
\veryLarge {দিকের ভারতীয় ইংরেজি সাহিত্যের এক প্রধান\linebreak}};%
\node[paragraphtext, anchor=north west, text width=655.0pt, align=justify] at (112.0,195.5){\setlength{\baselineskip}{47.300000000000004pt} \par
\veryLarge {দেয় এবং ছয়ই জুলাই সেখানে পৌঁছায়। ভারতীয়\linebreak}};%
\node[paragraphtext, anchor=north west, text width=657.0pt, align=justify] at (792.0,1361.5){\setlength{\baselineskip}{47.300000000000004pt} \par
\veryLarge {হয়েছে। নামটি মুনিওয়ার শব্দ থেকে এসেছে।\linebreak}};%
\node[paragraphtext, anchor=north west, text width=657.0pt, align=justify] at (792.0,1318.0){\setlength{\baselineskip}{47.300000000000004pt} \par
\veryLarge {স্বাক্ষর করেন। নিজের অবস্থা দেখে শ্রীঙ্গেরি\linebreak}};%
\node[paragraphtext, anchor=north west, text width=657.0pt, align=justify] at (792.0,1274.5){\setlength{\baselineskip}{47.300000000000004pt} \par
\veryLarge {জন্য আল বাগাভী নামে পরিচিত। আশরাফিয়ার\linebreak}};%
\node[paragraphtext, anchor=north west, text width=657.0pt, align=justify] at (792.0,1231.0){\setlength{\baselineskip}{47.300000000000004pt} \par
\veryLarge {সন্ত্রাসমূলক কর্মকাণ্ডে সহায়তা এবং প্ররোচনা\linebreak}};%
\node[paragraphtext, anchor=north west, text width=650.0pt, align=justify] at (117.0,454.5){\setlength{\baselineskip}{47.300000000000004pt} \par
\veryLarge {জলপ্রপাতের কাছে নির্বিচারে চেক ড্যাম নির্মাণের\linebreak}};%
\node[paragraphtext, anchor=north west, text width=650.0pt, align=justify] at (117.0,411.0){\setlength{\baselineskip}{47.300000000000004pt} \par
\veryLarge {এবং স্লথ বিয়ার। ষাঁড়ের ব্যাঙ, সাধারণ ভারতীয়\linebreak}};%
\node[paragraphtext, anchor=north west, text width=650.0pt, align=justify] at (117.0,367.5){\setlength{\baselineskip}{47.300000000000004pt} \par
\veryLarge {সংস্থা, এলাকার উদীয়মান ক্রীড়া প্রতিভাদের\linebreak}};%
\node[paragraphtext, anchor=north west, text width=650.0pt, align=justify] at (114.0,900.5){\setlength{\baselineskip}{47.300000000000004pt} \par
\veryLarge {সিংহাসনে একজন জীবিত পুরুষ উত্তরাধিকারী\linebreak}};%
\node[paragraphtext, anchor=north west, text width=650.0pt, align=justify] at (114.0,857.0){\setlength{\baselineskip}{47.300000000000004pt} \par
\veryLarge {সর্বদাই একটি গন্তব্য। রাজ্যটির মধ্যে বৃহত্তম\linebreak}};%
\node[paragraphtext, anchor=north west, text width=650.0pt, align=justify] at (114.0,813.5){\setlength{\baselineskip}{47.300000000000004pt} \par
\veryLarge {উন পালু রোম্বা সুথামা?' ছিল সঙ্গীত পরিচালক\linebreak}};%
\node[paragraphtext, anchor=north west, text width=650.0pt, align=justify] at (114.0,770.0){\setlength{\baselineskip}{47.300000000000004pt} \par
\veryLarge {কেন্দ্রীয় কয়লা, খনি ও সংসদ বিষয়ক মন্ত্রী শ্রী\linebreak}};%
\node[paragraphtext, anchor=north west, text width=650.0pt, align=justify] at (114.0,726.5){\setlength{\baselineskip}{47.300000000000004pt} \par
\veryLarge {ঈশ্বর তাঁর কাছে না আসা পর্যন্ত তিনি কঠোর\linebreak}};%
\node[paragraphtext, anchor=north west, text width=661.0pt, align=justify] at (787.0,418.5){\setlength{\baselineskip}{47.300000000000004pt} \par
\veryLarge {ক্রীড়া প্রতিভাদের প্রচার ও লালন-পালনের জন্য\linebreak}};%
\node[paragraphtext, anchor=north west, text width=661.0pt, align=justify] at (787.0,375.0){\setlength{\baselineskip}{47.300000000000004pt} \par
\veryLarge {ঊনিশশো সত্তর এর দশকের শেষের দিকে একটি\linebreak}};%
\node[paragraphtext, anchor=north west, text width=661.0pt, align=justify] at (787.0,331.5){\setlength{\baselineskip}{47.300000000000004pt} \par
\veryLarge {অমলের প্রতি তাঁর ক্রমবর্ধমান অনুভূতির কথা বলা\linebreak}};%
\node[paragraphtext, anchor=north west, text width=661.0pt, align=justify] at (787.0,288.0){\setlength{\baselineskip}{47.300000000000004pt} \par
\veryLarge {চৌধুরীর বাংলা চলচ্চিত্র অন্তহীনে উপস্থিত হন।\linebreak}};%
\node[paragraphtext, anchor=north west, text width=663.0pt, align=justify] at (114.0,1205.5){\setlength{\baselineskip}{47.300000000000004pt} \par
\veryLarge {বলা সরঞ্জাম ব্যবহার করতে পারেন। এই যুদ্ধে\linebreak}};%
\node[paragraphtext, anchor=north west, text width=665.0pt, align=justify] at (793.0,1674.5){\setlength{\baselineskip}{47.300000000000004pt} \par
\veryLarge {যে সজীব নরম মেমব্রেন বা ঝিল্লি থাকে তাকে\linebreak}};%
\node[paragraphtext, anchor=north west, text width=665.0pt, align=justify] at (793.0,1631.0){\setlength{\baselineskip}{47.300000000000004pt} \par
\veryLarge {সঙ্গে দুটি ঈদ (ঈদ-উল-ফিতর এবং\linebreak}};%
\node[paragraphtext, anchor=north west, text width=665.0pt, align=justify] at (793.0,1587.5){\setlength{\baselineskip}{47.300000000000004pt} \par
\veryLarge {লোকটি মঙ্গলবার থেকে হাসপাতালে ছিল যখন সে\linebreak}};%
\node[paragraphtext, anchor=north west, text width=416.0pt, align=left] at (1035.0,860.5){\setlength{\baselineskip}{47.300000000000004pt} \par
\veryLarge {এমনকি পেলোটন বলা হয়।\linebreak}};%
\node[paragraphtext, anchor=north west, text width=416.0pt, align=left] at (1035.0,817.0){\setlength{\baselineskip}{47.300000000000004pt} \par
\veryLarge {অফিসার লেফটেন্যান্ট হারমান\linebreak}};%
\node[paragraphtext, anchor=north west, text width=416.0pt, align=left] at (1035.0,773.5){\setlength{\baselineskip}{47.300000000000004pt} \par
\veryLarge {সেতু দ্বারা বাঘমতি নদী\linebreak}};%
\node[paragraphtext, anchor=north west, text width=416.0pt, align=left] at (1035.0,730.0){\setlength{\baselineskip}{47.300000000000004pt} \par
\veryLarge {দিলীপ তাঁর সহ-অভিনেতা এবং\linebreak}};%
\node[paragraphtext, anchor=north west, text width=416.0pt, align=left] at (1035.0,686.5){\setlength{\baselineskip}{47.300000000000004pt} \par
\veryLarge {ঊনত্রিশে জুনের অন্যান্য\linebreak}};%
\node[paragraphtext, anchor=north west, text width=416.0pt, align=left] at (1035.0,643.0){\setlength{\baselineskip}{47.300000000000004pt} \par
\veryLarge {মধু শর্মা অভিনীত একটি\linebreak}};%
\node[paragraphtext, anchor=north west, text width=416.0pt, align=left] at (1035.0,599.5){\setlength{\baselineskip}{47.300000000000004pt} \par
\veryLarge {এই তহবিলের শেয়ারগুলি\linebreak}};%
\node[paragraphtext, anchor=north west, text width=416.0pt, align=left] at (1035.0,556.0){\setlength{\baselineskip}{47.300000000000004pt} \par
\veryLarge {তিনি বের করেন। পরে সপ্তদশ\linebreak}};%
\node[paragraphtext, anchor=north west, text width=416.0pt, align=left] at (1035.0,512.5){\setlength{\baselineskip}{47.300000000000004pt} \par
\veryLarge {শেষ ছয়টি মুক্তিপ্রাপ্ত ছবি\linebreak}};%
\node[paragraphtext, anchor=north west, text width=690.0pt, align=justify] at (776.0,243.5){\setlength{\baselineskip}{47.300000000000004pt} \par
\veryLarge {কুড়ি খ্রিষ্টাব্দের ছাব্বিশে আগস্ট কলকাতায় পঁচাশি\linebreak}};%
\node[paragraphtext, anchor=north west, text width=690.0pt, align=justify] at (776.0,200.0){\setlength{\baselineskip}{47.300000000000004pt} \par
\veryLarge {জিততে পারে। একটি স্কিমা ওয়ালিচি গাছের\linebreak}};%
\node[paragraphtext, anchor=north west, text width=690.0pt, align=justify] at (776.0,156.5){\setlength{\baselineskip}{47.300000000000004pt} \par
\veryLarge {শহর দখল করে, কিন্তু রানী ছদ্মবেশে পালিয়ে যান।\linebreak}};%
\node[headertext, anchor=north west, text width=391.0pt, align=left] at (911.0,950.5){\setlength{\baselineskip}{66.0pt} \par
\alphaa {একটি অভ্যাস। চীন\linebreak}};%
\node[paragraphtext, anchor=north west, text width=652.0pt, align=left] at (791.0,1085.5){\setlength{\baselineskip}{47.300000000000004pt} \par
\veryLarge {পুনরুদ্ধার করতে চান তবে তারা তাঁর জন্য কাজ\linebreak}};%
\node[paragraphtext, anchor=north west, text width=652.0pt, align=left] at (791.0,1042.0){\setlength{\baselineskip}{47.300000000000004pt} \par
\veryLarge {ওয়েব সিরিজে দেখা গেছে এবং হটস্টারে\linebreak}};%
\node[paragraphtext, anchor=north west, text width=228.0pt, align=left] at (1227.0,1152.5){\setlength{\baselineskip}{40.7pt} \par
\verylarge {যিনি কুমারের কণ্ঠে\linebreak}};%
\node[paragraphtext, anchor=north west, text width=684.0pt, align=justify] at (776.0,1993.5){\setlength{\baselineskip}{47.300000000000004pt} \par
\veryLarge {ভি. মধুসূধন রাওয়ের চলচ্চিত্র প্রতিষ্ঠানে যোগ দেন।\linebreak}};%
\node[paragraphtext, anchor=north west, text width=684.0pt, align=justify] at (776.0,1950.0){\setlength{\baselineskip}{47.300000000000004pt} \par
\veryLarge {দিল্লি। রেইডের নেতৃত্ব দেন ডলোইস, যারা রেইড\linebreak}};%
\node[paragraphtext, anchor=north west, text width=684.0pt, align=justify] at (776.0,1906.5){\setlength{\baselineskip}{47.300000000000004pt} \par
\veryLarge {চাখেসাং, চ্যাং, দিমাসা কাচারি, খিয়ামনিউনগান,\linebreak}};%
\node[paragraphtext, anchor=north west, text width=692.0pt, align=justify] at (780.0,1852.5){\setlength{\baselineskip}{47.300000000000004pt} \par
\veryLarge {ক্লাব প্রতিষ্ঠিত হয়। এটি ঢাকা সায়দাবাদ বাসস্ট্যান্ড\linebreak}};%
\node[paragraphtext, anchor=north west, text width=692.0pt, align=justify] at (780.0,1809.0){\setlength{\baselineskip}{47.300000000000004pt} \par
\veryLarge {করেছি এইভাবে আমি বলি তখন আমার সেটা খুবই\linebreak}};%
\node[paragraphtext, anchor=north west, text width=692.0pt, align=justify] at (780.0,1765.5){\setlength{\baselineskip}{47.300000000000004pt} \par
\veryLarge {উইক এমন অন্যান্য দলিল বা সম্পত্তি নিদর্শন পত্র\linebreak}};%
\node[paragraphtext, anchor=north west, text width=644.0pt, align=left] at (121.0,1906.5){\setlength{\baselineskip}{47.300000000000004pt} \par
\veryLarge {করে, যেগুলি সাধারণত বিদেশী বন্দুকধারীদের\linebreak}};%
\node[paragraphtext, anchor=north west, text width=644.0pt, align=left] at (121.0,1863.0){\setlength{\baselineskip}{47.300000000000004pt} \par
\veryLarge {খ্রিষ্টীয় প্রথম শতাব্দী থেকে পশ্চিম ভারতের\linebreak}};%
\node[headertext, anchor=north west, text width=563.0pt, align=left] at (173.0,1775.5){\setlength{\baselineskip}{68.2pt} \par
\superHuge {কেউ এমন আচরণ করতে\linebreak}};%
\node[headertext, anchor=north west, text width=322.0pt, align=left] at (109.0,1983.5){\setlength{\baselineskip}{66.0pt} \par
\alphaa {শোনা যাচ্ছে।\linebreak}};%
\end{tikzpicture}%
\end{center}%
\end{document}