\documentclass[11pt]{article}%
\usepackage[T1]{fontenc}%
\usepackage[utf8]{inputenc}%
\usepackage{lmodern}%
\usepackage{textcomp}%
\usepackage{lastpage}%
\usepackage{tikz}%
\usepackage{fontspec}%
\usepackage{polyglossia}%
\usepackage[paperwidth=1704pt,paperheight=2230pt,margin=0pt]{geometry}%
%
\setmainlanguage{bengali}%
\setotherlanguage{english}%
\newfontfamily\bengalifont[Script=Bengali,Path=/home/vivek/Pager/fonts/bengali/Paragraph/]{NotoSansBengali_Condensed-ExtraLight}%
\newfontfamily\headerfont[Script=Bengali,Path=/home/vivek/Pager/fonts/bengali/Header/]{NotoSansBengali-Black}%
\newfontfamily\paragraphfont[Script=Bengali,Path=/home/vivek/Pager/fonts/bengali/Paragraph/]{NotoSansBengali_Condensed-ExtraLight}%

    \makeatletter
    \newcommand{\zettaHuge}{\@setfontsize\zettaHuge{200}{220}}
    \newcommand{\exaHuge}{\@setfontsize\exaHuge{165}{180}}
    \newcommand{\petaHuge}{\@setfontsize\petaHuge{135}{150}}
    \newcommand{\teraHuge}{\@setfontsize\teraHuge{110}{120}}
    \newcommand{\gigaHuge}{\@setfontsize\gigaHuge{90}{100}}
    \newcommand{\megaHuge}{\@setfontsize\megaHuge{75}{85}}
    \newcommand{\superHuge}{\@setfontsize\superHuge{62}{70}}
    \newcommand{\verylarge}{\@setfontsize\verylarge{37}{42}}
    \newcommand{\veryLarge}{\@setfontsize\veryLarge{43}{49}}
    \newcommand{\veryHuge}{\@setfontsize\veryHuge{62}{70}}
    \newcommand{\alphaa}{\@setfontsize\alphaa{60}{66}}
    \newcommand{\betaa}{\@setfontsize\betaa{57}{63}}
    \newcommand{\gammaa}{\@setfontsize\gammaa{55}{61}}
    \newcommand{\deltaa}{\@setfontsize\deltaa{53}{59}}
    \newcommand{\epsilona}{\@setfontsize\epsilona{51}{57}}
    \newcommand{\zetaa}{\@setfontsize\zetaa{47}{53}}
    \newcommand{\etaa}{\@setfontsize\etaa{45}{51}}
    \newcommand{\iotaa}{\@setfontsize\iotaa{41}{47}}
    \newcommand{\kappaa}{\@setfontsize\kappaa{39}{45}}
    \newcommand{\lambdaa}{\@setfontsize\lambdaa{35}{41}}
    \newcommand{\mua}{\@setfontsize\mua{33}{39}}
    \newcommand{\nua}{\@setfontsize\nua{31}{37}}
    \newcommand{\xia}{\@setfontsize\xia{29}{35}}
    \newcommand{\pia}{\@setfontsize\pia{27}{33}}
    \newcommand{\rhoa}{\@setfontsize\rhoa{24}{30}}
    \newcommand{\sigmaa}{\@setfontsize\sigmaa{22}{28}}
    \newcommand{\taua}{\@setfontsize\taua{18}{24}}
    \newcommand{\upsilona}{\@setfontsize\upsilona{16}{22}}
    \newcommand{\phia}{\@setfontsize\phia{15}{20}}
    \newcommand{\chia}{\@setfontsize\chia{13}{18}}
    \newcommand{\psia}{\@setfontsize\psia{11}{16}}
    \newcommand{\omegaa}{\@setfontsize\omegaa{6}{7}}
    \newcommand{\oomegaa}{\@setfontsize\oomegaa{4}{5}}
    \newcommand{\ooomegaa}{\@setfontsize\ooomegaa{3}{4}}
    \newcommand{\oooomegaaa}{\@setfontsize\oooomegaaa{2}{3}}
    \makeatother
    %
%
\begin{document}%
\normalsize%
\begin{center}%
\begin{tikzpicture}[x=1pt, y=1pt]%
\node[anchor=south west, inner sep=0pt] at (0,0) {\includegraphics[width=1704pt,height=2230pt]{/home/vivek/Pager/images_val/mg_hi_000605_0.png}};%
\tikzset{headertext/.style={font=\headerfont, text=black}}%
\tikzset{paragraphtext/.style={font=\paragraphfont, text=black}}%
\node[paragraphtext, anchor=north west, text width=764.0pt, align=justify] at (59.0,1508.5){\setlength{\baselineskip}{47.300000000000004pt} \par
\veryLarge {হ্যান্ডলুম পছন্দ করছে। বাহাদুর শাহ যোধপুরের অজিত সিং\linebreak}};%
\node[paragraphtext, anchor=north west, text width=764.0pt, align=justify] at (59.0,1465.0){\setlength{\baselineskip}{47.300000000000004pt} \par
\veryLarge {সূত্রগুলি জম্মু ও কাশ্মীরকে ভারত-অধিকৃত কাশ্মীর কিংবা\linebreak}};%
\node[paragraphtext, anchor=north west, text width=764.0pt, align=justify] at (59.0,1421.5){\setlength{\baselineskip}{47.300000000000004pt} \par
\veryLarge {যে কারণে এর নামটিকে প্রায়শই ছোট করে শুধু সিলমিক\linebreak}};%
\node[paragraphtext, anchor=north west, text width=764.0pt, align=justify] at (59.0,1378.0){\setlength{\baselineskip}{47.300000000000004pt} \par
\veryLarge {আবার গঙ্গানারায়ণ সিংয়ের নেতৃত্বে ব্রিটিশদের বিরুদ্ধে\linebreak}};%
\node[paragraphtext, anchor=north west, text width=764.0pt, align=justify] at (59.0,1334.5){\setlength{\baselineskip}{47.300000000000004pt} \par
\veryLarge {জন্য নিজের জীবন উৎসর্গ করেছিলেন, তিনি এই ধ্রুপদী\linebreak}};%
\node[paragraphtext, anchor=north west, text width=764.0pt, align=justify] at (59.0,1291.0){\setlength{\baselineskip}{47.300000000000004pt} \par
\veryLarge {সর্বাধিক সংখ্যক ডুয়েট গেয়েছেন কুমার। প্রথম কুমারগুপ্তও\linebreak}};%
\node[paragraphtext, anchor=north west, text width=764.0pt, align=justify] at (59.0,1247.5){\setlength{\baselineskip}{47.300000000000004pt} \par
\veryLarge {শারীরিক ব্যায়াম অন্তর্ভুক্ত রয়েছে। রাশিয়া এবং ইউরোপে\linebreak}};%
\node[paragraphtext, anchor=north west, text width=764.0pt, align=justify] at (59.0,1204.0){\setlength{\baselineskip}{47.300000000000004pt} \par
\veryLarge {ভাল বন্ধু ছিলেন। দ্বিপরমাণুক অণু এইচএফ\linebreak}};%
\node[paragraphtext, anchor=north west, text width=764.0pt, align=justify] at (59.0,1160.5){\setlength{\baselineskip}{47.300000000000004pt} \par
\veryLarge {খুব খুশি যে মানুষজন বিপুল সংখ্যায় সোশ্যাল মিডিয়া\linebreak}};%
\node[paragraphtext, anchor=north west, text width=764.0pt, align=justify] at (59.0,1117.0){\setlength{\baselineskip}{47.300000000000004pt} \par
\veryLarge {টাইগার্স, নিউইয়র্ক ইয়াঙ্কিস, হিউস্টন অ্যাস্ট্রস এবং\linebreak}};%
\node[paragraphtext, anchor=north west, text width=764.0pt, align=justify] at (59.0,1073.5){\setlength{\baselineskip}{47.300000000000004pt} \par
\veryLarge {খুশি, নাচ দেখতে দেখতে আমি ভাবছিলাম জনমানবশূণ্য\linebreak}};%
\node[paragraphtext, anchor=north west, text width=764.0pt, align=justify] at (59.0,1030.0){\setlength{\baselineskip}{47.300000000000004pt} \par
\veryLarge {প্যান্টিক লেক এন্ট্রান্স, সেল এবং বেয়ারনসডেল এলাকায়\linebreak}};%
\node[paragraphtext, anchor=north west, text width=764.0pt, align=justify] at (59.0,986.5){\setlength{\baselineskip}{47.300000000000004pt} \par
\veryLarge {নিজেদেরকে উল্লম্বভাবে চালিত করে। ইতালি থেকে মার্বেল,\linebreak}};%
\node[paragraphtext, anchor=north west, text width=764.0pt, align=justify] at (59.0,943.0){\setlength{\baselineskip}{47.300000000000004pt} \par
\veryLarge {কুকিজ বা হানুক্কার জন্য প্রাসঙ্গিক চা সেট নিন। অন্য\linebreak}};%
\node[paragraphtext, anchor=north west, text width=764.0pt, align=justify] at (69.0,865.5){\setlength{\baselineskip}{47.300000000000004pt} \par
\veryLarge {নাম্বুল, সেকমাই, চাকপি, থৌবাল এবং খুগা। বিভিন্ন ক্ষেত্রে\linebreak}};%
\node[paragraphtext, anchor=north west, text width=764.0pt, align=justify] at (69.0,822.0){\setlength{\baselineskip}{47.300000000000004pt} \par
\veryLarge {করা হয়। এটি প্রায় দেড়শোটি ইভেন্টের সমন্বয়ে গঠিত এবং\linebreak}};%
\node[paragraphtext, anchor=north west, text width=764.0pt, align=justify] at (69.0,778.5){\setlength{\baselineskip}{47.300000000000004pt} \par
\veryLarge {পাওয়া যাবে কাঁকড়াবিছে, মাকড়সা, নানা প্রজাতির পতঙ্গ\linebreak}};%
\node[paragraphtext, anchor=north west, text width=764.0pt, align=justify] at (69.0,735.0){\setlength{\baselineskip}{47.300000000000004pt} \par
\veryLarge {জ্ঞানং বিজ্ঞান সহিতং যজ্ঞনাত্বা মোক্ষসে শুভাত এই শ্লোক\linebreak}};%
\node[paragraphtext, anchor=north west, text width=764.0pt, align=justify] at (69.0,691.5){\setlength{\baselineskip}{47.300000000000004pt} \par
\veryLarge {জুড়ে কয়েক পশলা বষ্টির সম্ভাবনা আইআইটি দিল্লি এং\linebreak}};%
\node[paragraphtext, anchor=north west, text width=764.0pt, align=justify] at (69.0,648.0){\setlength{\baselineskip}{47.300000000000004pt} \par
\veryLarge {ভোঁসলে দ্বারা বর্তমান মন্দিরটি নির্মিত হয়েছিল। এখানে\linebreak}};%
\node[paragraphtext, anchor=north west, text width=764.0pt, align=justify] at (69.0,604.5){\setlength{\baselineskip}{47.300000000000004pt} \par
\veryLarge {ব্যক্তিগত পরিষেবার ব্যবস্থা অব্যাহত ছিল। সংগঠিত\linebreak}};%
\node[paragraphtext, anchor=north west, text width=764.0pt, align=justify] at (69.0,561.0){\setlength{\baselineskip}{47.300000000000004pt} \par
\veryLarge {বিক্রয়যোগ্য পণ্য এবং তার শিল্পের বর্ণনা দেবে। আর\linebreak}};%
\node[paragraphtext, anchor=north west, text width=764.0pt, align=justify] at (69.0,517.5){\setlength{\baselineskip}{47.300000000000004pt} \par
\veryLarge {যে রাঁচি স্টেডিয়াম পয়তিরিশ হাজার জনের আসন ধারনের\linebreak}};%
\node[paragraphtext, anchor=north west, text width=764.0pt, align=justify] at (69.0,474.0){\setlength{\baselineskip}{47.300000000000004pt} \par
\veryLarge {প্রধান চরিত্র পাণ্ডবদের কাহিনী বর্ণনা করে। গ্রীষ্মের মরসুমে\linebreak}};%
\node[paragraphtext, anchor=north west, text width=738.0pt, align=justify] at (871.0,953.5){\setlength{\baselineskip}{47.300000000000004pt} \par
\veryLarge {হবে। নাগাল্যান্ডের সতেরোটি প্রধান উপজাতি হল\linebreak}};%
\node[paragraphtext, anchor=north west, text width=738.0pt, align=justify] at (871.0,910.0){\setlength{\baselineskip}{47.300000000000004pt} \par
\veryLarge {সর্বস্ব দিয়া আমার জীবনধারা নিয়ন্ত্রিত করিয়াছিলেন সেই\linebreak}};%
\node[paragraphtext, anchor=north west, text width=738.0pt, align=justify] at (871.0,866.5){\setlength{\baselineskip}{47.300000000000004pt} \par
\veryLarge {লাইন দ্বারা সংযুক্ত। এর চূড়ান্ত লক্ষ্য হল বোধি (জাগরণ)\linebreak}};%
\node[paragraphtext, anchor=north west, text width=738.0pt, align=justify] at (871.0,823.0){\setlength{\baselineskip}{47.300000000000004pt} \par
\veryLarge {পাঁচটি চলচ্চিত্রে কাজ করেছেন। মুলাঙ্কাডাকম-এ একটি\linebreak}};%
\node[paragraphtext, anchor=north west, text width=738.0pt, align=justify] at (871.0,779.5){\setlength{\baselineskip}{47.300000000000004pt} \par
\veryLarge {পাঁচ মিনিট বাকি থাকতে পেনাল্টি থেকে ইংল্যান্ডকে দুই\linebreak}};%
\node[paragraphtext, anchor=north west, text width=738.0pt, align=justify] at (871.0,736.0){\setlength{\baselineskip}{47.300000000000004pt} \par
\veryLarge {সালের 'রামজি রাও স্পিকিং' চলচ্চিত্রে কাজের সূত্রে\linebreak}};%
\node[paragraphtext, anchor=north west, text width=738.0pt, align=justify] at (871.0,692.5){\setlength{\baselineskip}{47.300000000000004pt} \par
\veryLarge {এবং এটিই প্রতিক্ষেপণের চাপ নেয়। এটি মুঘল\linebreak}};%
\node[paragraphtext, anchor=north west, text width=738.0pt, align=justify] at (871.0,649.0){\setlength{\baselineskip}{47.300000000000004pt} \par
\veryLarge {হিন্দ ফৌজের পরাক্রমের জয়গান গাইছে। নদীর\linebreak}};%
\node[paragraphtext, anchor=north west, text width=738.0pt, align=justify] at (871.0,605.5){\setlength{\baselineskip}{47.300000000000004pt} \par
\veryLarge {ভয়ঙ্করভাবে ভুল দিকে চলে যায় যখন রাগিনী নিজের\linebreak}};%
\node[paragraphtext, anchor=north west, text width=771.0pt, align=justify] at (70.0,410.5){\setlength{\baselineskip}{47.300000000000004pt} \par
\veryLarge {দিক থেকে সর্বকালের সেরা টেস্ট ব্যাটসম্যান বলে মনে করা\linebreak}};%
\node[paragraphtext, anchor=north west, text width=771.0pt, align=justify] at (70.0,367.0){\setlength{\baselineskip}{47.300000000000004pt} \par
\veryLarge {ছিলেন, তার সঙ্গে অমিত শাহ ছিলেন এবং ওনার\linebreak}};%
\node[paragraphtext, anchor=north west, text width=771.0pt, align=justify] at (70.0,323.5){\setlength{\baselineskip}{47.300000000000004pt} \par
\veryLarge {জাহাজের দ্রুত উন্নতি ঘুড়ির প্রতি আগ্রহ হ্রাস করে। প্রকৃত\linebreak}};%
\node[paragraphtext, anchor=north west, text width=771.0pt, align=justify] at (70.0,280.0){\setlength{\baselineskip}{47.300000000000004pt} \par
\veryLarge {প্রকরণগুলির গণনায় সহায়তা করার জন্য কম্পিউটারের\linebreak}};%
\node[paragraphtext, anchor=north west, text width=771.0pt, align=justify] at (70.0,236.5){\setlength{\baselineskip}{47.300000000000004pt} \par
\veryLarge {বিকল্প প্রদান করে৷ পরিবর্তে, সপ্তাহান্তে বা বিশেষ শপিং\linebreak}};%
\node[paragraphtext, anchor=north west, text width=771.0pt, align=justify] at (70.0,193.0){\setlength{\baselineskip}{47.300000000000004pt} \par
\veryLarge {এই মামলায় মোট সত্তর জন নিযুক্ত রয়েছে। দ্বিতীয়\linebreak}};%
\node[paragraphtext, anchor=north west, text width=730.0pt, align=justify] at (880.0,546.5){\setlength{\baselineskip}{47.300000000000004pt} \par
\veryLarge {পচাত্তর দানা। তিনি শ্রোতাদের কাছ থেকে আলোচনার\linebreak}};%
\node[paragraphtext, anchor=north west, text width=730.0pt, align=justify] at (880.0,503.0){\setlength{\baselineskip}{47.300000000000004pt} \par
\veryLarge {অন্তর বৈশিষ্ট্য প্রদর্শন করলেও বৈশিষ্ট্যের সঙ্গে সংযুক্ত\linebreak}};%
\node[paragraphtext, anchor=north west, text width=730.0pt, align=justify] at (880.0,459.5){\setlength{\baselineskip}{47.300000000000004pt} \par
\veryLarge {সবচেয়ে বড় ব্যাংক সঞ্চয় রয়েছে বলে জানা গেছে। এটি\linebreak}};%
\node[paragraphtext, anchor=north west, text width=730.0pt, align=justify] at (880.0,416.0){\setlength{\baselineskip}{47.300000000000004pt} \par
\veryLarge {নিরাপত্তা এবং তথ্য আদানপ্রদান ও ইন্টারঅপারেবিলিটি\linebreak}};%
\node[paragraphtext, anchor=north west, text width=730.0pt, align=justify] at (880.0,372.5){\setlength{\baselineskip}{47.300000000000004pt} \par
\veryLarge {সম্মানিত করেন। খলজিদের দ্বারা উৎখাত না হওয়া পর্যন্ত\linebreak}};%
\node[paragraphtext, anchor=north west, text width=730.0pt, align=justify] at (880.0,329.0){\setlength{\baselineskip}{47.300000000000004pt} \par
\veryLarge {পুরুষ ও একজন মহিলা। যদিও ক্যানোগুলি একসময়\linebreak}};%
\node[paragraphtext, anchor=north west, text width=730.0pt, align=justify] at (880.0,285.5){\setlength{\baselineskip}{47.300000000000004pt} \par
\veryLarge {লেখক আহমদ ছফা রচিত প্রথম উপন্যাস। ওয়াইজে\linebreak}};%
\node[paragraphtext, anchor=north west, text width=730.0pt, align=justify] at (880.0,242.0){\setlength{\baselineskip}{47.300000000000004pt} \par
\veryLarge {আইসিঙের এমজিএম অ্যানিমেশন স্টুডিওর অংশ\linebreak}};%
\node[paragraphtext, anchor=north west, text width=730.0pt, align=justify] at (880.0,198.5){\setlength{\baselineskip}{47.300000000000004pt} \par
\veryLarge {দুর্নীতির বিষয়ে আরও সতর্ক করে সরকারি কর্মকর্তাদের\linebreak}};%
\node[paragraphtext, anchor=north west, text width=746.0pt, align=justify] at (863.0,1139.5){\setlength{\baselineskip}{47.300000000000004pt} \par
\veryLarge {সম্ভবত প্রসারিত হয়। মূল রচনা কেন্দ্র করে, সংক্ষিপ্ত ও\linebreak}};%
\node[paragraphtext, anchor=north west, text width=746.0pt, align=justify] at (863.0,1096.0){\setlength{\baselineskip}{47.300000000000004pt} \par
\veryLarge {জানিয়েছেন। প্রধান ভারতীয় সংস্থা এবং বহুজাতিক\linebreak}};%
\node[paragraphtext, anchor=north west, text width=746.0pt, align=justify] at (863.0,1052.5){\setlength{\baselineskip}{47.300000000000004pt} \par
\veryLarge {করেন। ঊনিশশো আটাত্তর সালে বালাচন্দর পরিচালিত\linebreak}};%
\node[paragraphtext, anchor=north west, text width=746.0pt, align=justify] at (863.0,1009.0){\setlength{\baselineskip}{47.300000000000004pt} \par
\veryLarge {জন্য অস্ত্রোপচারের আগে ওষুধের প্রয়োজন হয়। অন্যান্য\linebreak}};%
\node[paragraphtext, anchor=north west, text width=328.0pt, align=left] at (659.0,2179.5){\setlength{\baselineskip}{47.300000000000004pt} \par
\veryLarge {এবং কর্ণাটকের বাদামি\linebreak}};%
\node[paragraphtext, anchor=north west, text width=328.0pt, align=left] at (659.0,2136.0){\setlength{\baselineskip}{47.300000000000004pt} \par
\veryLarge {যে আমরা আবার তিন\linebreak}};%
\node[paragraphtext, anchor=north west, text width=328.0pt, align=left] at (659.0,2092.5){\setlength{\baselineskip}{47.300000000000004pt} \par
\veryLarge {একটি বিষয়ে মনোনিবেশ\linebreak}};%
\node[paragraphtext, anchor=north west, text width=328.0pt, align=left] at (659.0,2049.0){\setlength{\baselineskip}{47.300000000000004pt} \par
\veryLarge {সমিতি, কমিটি, সংঘ,\linebreak}};%
\node[paragraphtext, anchor=north west, text width=328.0pt, align=left] at (659.0,2005.5){\setlength{\baselineskip}{47.300000000000004pt} \par
\veryLarge {কঠিন হতে পারে।\linebreak}};%
\node[headertext, anchor=north west, text width=1353.0pt, align=left, text=black] at (152.0,1771.5){\setlength{\baselineskip}{82.5pt} \par
\megaHuge {পারেনি। কিন্তু পরের দিন সকালে পুলিশ\linebreak}};%
\node[paragraphtext, anchor=north west, text width=268.0pt, align=left] at (701.0,1637.5){\setlength{\baselineskip}{47.300000000000004pt} \par
\veryLarge {করে ক্ষারক দহন\linebreak}};%
\node[paragraphtext, anchor=north west, text width=522.0pt, align=left] at (1084.0,1497.5){\setlength{\baselineskip}{42.900000000000006pt} \par
\kappaa {সনামাহী, লেইমারেন, ওকনারেল, থাংনারেল,\linebreak}};%
\node[paragraphtext, anchor=north west, text width=522.0pt, align=left] at (1084.0,1454.0){\setlength{\baselineskip}{42.900000000000006pt} \par
\kappaa {মিলে অযোধ্যা যেতে চাইছি তো যাইহোক,\linebreak}};%
\node[paragraphtext, anchor=north west, text width=522.0pt, align=left] at (1084.0,1410.5){\setlength{\baselineskip}{42.900000000000006pt} \par
\kappaa {বক্সিং প্রতিযোগিতায়, যা অনেকটা সান্দার\linebreak}};%
\node[paragraphtext, anchor=north west, text width=522.0pt, align=left] at (1084.0,1367.0){\setlength{\baselineskip}{42.900000000000006pt} \par
\kappaa {তাঁর পরিবারের বিভিন্ন সদস্যকে অধিষ্ঠিত\linebreak}};%
\node[paragraphtext, anchor=north west, text width=522.0pt, align=left] at (1084.0,1323.5){\setlength{\baselineskip}{42.900000000000006pt} \par
\kappaa {বা বর্ষাঋতুর আবির্ভাব ঘটায়। কারাতের\linebreak}};%
\node[paragraphtext, anchor=north west, text width=522.0pt, align=left] at (1084.0,1280.0){\setlength{\baselineskip}{42.900000000000006pt} \par
\kappaa {হয়। ব্রহ্মচারিণীকে মুক্তি বা মোক্ষ এবং শান্তি\linebreak}};%
\node[paragraphtext, anchor=north west, text width=522.0pt, align=left] at (1084.0,1236.5){\setlength{\baselineskip}{42.900000000000006pt} \par
\kappaa {লো লো আল্ফ়া সমস্যায় যাচ্ছে। সেই\linebreak}};%
\node[headertext, anchor=north west, text width=248.0pt, align=left] at (0.0,2079.5){\setlength{\baselineskip}{47.300000000000004pt} \par
\veryLarge {প্ল্যাটফর্ম শ্রী\linebreak}};%
\node[headertext, anchor=north west, text width=248.0pt, align=left] at (0.0,2036.0){\setlength{\baselineskip}{47.300000000000004pt} \par
\veryLarge {একটি সার্ভ যা\linebreak}};%
\node[headertext, anchor=north west, text width=697.0pt, align=left] at (502.0,128.5){\setlength{\baselineskip}{47.300000000000004pt} \par
\veryLarge {হরিয়ানা, পঞ্জাব এবং হিমাচল প্রদেশের সান্নিধ্য এবং\linebreak}};%
\node[headertext, anchor=north west, text width=697.0pt, align=left] at (502.0,85.0){\setlength{\baselineskip}{47.300000000000004pt} \par
\veryLarge {যাওয়ার এবং পরে ফিরে আসার পথ প্রদান করে।\linebreak}};%
\node[paragraphtext, anchor=north west, text width=350.0pt, align=left] at (863.0,1195.5){\setlength{\baselineskip}{42.900000000000006pt} \par
\kappaa {তিরাশি হাজার পাঁচশো একান্ন\linebreak}};%
\node[paragraphtext, anchor=north west, text width=773.0pt, align=left] at (838.0,1506.5){\setlength{\baselineskip}{47.300000000000004pt} \par
\veryLarge {তেত্তিরিশ খ্রীষ্টাব্দে শ্রীহট্ট জেলার হবিগঞ্জ মহকুমার বানিয়াচং\linebreak}};%
\node[paragraphtext, anchor=north west, text width=773.0pt, align=left] at (838.0,1463.0){\setlength{\baselineskip}{47.300000000000004pt} \par
\veryLarge {দশকের মাঝামাঝি থেকে উনিশশো সত্তরের দশকের\linebreak}};%
\node[paragraphtext, anchor=north west, text width=773.0pt, align=left] at (838.0,1419.5){\setlength{\baselineskip}{47.300000000000004pt} \par
\veryLarge {পাঁচ, ভায়াকম আঠারো, শিমারো, এমএক্সপ্লেয়ার সহ বিভিন্ন\linebreak}};%
\node[paragraphtext, anchor=north west, text width=773.0pt, align=left] at (838.0,1376.0){\setlength{\baselineskip}{47.300000000000004pt} \par
\veryLarge {এই এলাকায় অফিসের জায়গা খুঁজছে এমন আইটি\linebreak}};%
\node[paragraphtext, anchor=north west, text width=773.0pt, align=left] at (838.0,1332.5){\setlength{\baselineskip}{47.300000000000004pt} \par
\veryLarge {মাসে অনুষ্ঠিত হয়। সোয়াজিল্যান্ডের রাজা, তৃতীয় মস্বাতী,\linebreak}};%
\node[paragraphtext, anchor=north west, text width=773.0pt, align=left] at (838.0,1289.0){\setlength{\baselineskip}{47.300000000000004pt} \par
\veryLarge {প্রতিযোগিতার নামকরণ করা হয়েছে। জয়সলমের পর্যন্ত\linebreak}};%
\node[paragraphtext, anchor=north west, text width=773.0pt, align=left] at (838.0,1245.5){\setlength{\baselineskip}{47.300000000000004pt} \par
\veryLarge {বিজেপি পার্টি অফিসকে ভাঙচুর করে এবং মারপিট করে\linebreak}};%
\node[paragraphtext, anchor=north west, text width=773.0pt, align=left] at (838.0,1202.0){\setlength{\baselineskip}{47.300000000000004pt} \par
\veryLarge {লুই মাউন্টব্যাটেন, বার্মার প্রথম আর্ল মাউন্টব্যাটেন,\linebreak}};%
\end{tikzpicture}%
\end{center}%
\end{document}